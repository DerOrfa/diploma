\documentclass[halfparskip,a4paper,12pt,headinclude, footexclude]{scrreprt}
\usepackage{german}
\usepackage[latin1]{inputenc}
\usepackage{amsmath}
\usepackage{amsfonts}
\usepackage{graphicx}
\usepackage{listings}
\usepackage{newlfont}
\usepackage{hyperref}
\usepackage{hyphenat}
\usepackage[german]{varioref}

\usepackage{bibgerm}
\usepackage[numbers]{natbib}
\usepackage{url}
\usepackage[german]{nomencl}
\usepackage[german]{babel}
\usepackage{fancyhdr}

\usepackage{xspace}
\graphicspath{{pics/}}

\addto\captionsgerman{\renewcommand{\figurename}{Abb.}}

\renewcommand*{\figureformat}{\figurename~\thefigure\autodot}

\pagestyle{fancy}
\lhead{}
\chead{}
\rfoot{Seite \thepage}
\cfoot{}
\lfoot{3D--Volumendaten auf Consumer--Hardware}
\renewcommand{\footrulewidth}{0.5pt}
\textheight=620pt

\newcommand{\fachw}[1]{\emph{#1}} %textsf
\newcommand{\code}[1]{\texttt{#1}}
\newcommand{\menge}[1]{\mathbb{#1}}
\newcommand{\elem}[1]{\in \menge{#1}}
\newcommand{\elemof}[2]{\in \menge{#1}^#2}
\newcommand{\eRaum}[1]{$#1 \elemof{R}{3}$\xspace}
\newcommand{\eR}[1]{$#1 \elem{R}$\xspace}
\newcommand{\pp}{\ldots}
\newcommand{\degrees}[1]{\ensuremath{#1^\circ}}
\newcommand{\gdw}{\leftrightarrow}

\newcommand{\abb}[1]{\figurename \vref{#1}}

\newcommand\email{\begingroup \urlstyle{rm}\Url}

\newcommand{\inclfigure}[5]{
        \begin{figure}[#1]
          \begin{center}
            \fbox{\includegraphics[keepaspectratio=true,width=#3]{#2}}
          \end{center}
          \caption{#4}
          \label{#5}
        \end{figure}
}

\newcommand{\inclfigures}[5]{
  \begin{figure}[hbt]
    \begin{center}
      \fbox{
        \includegraphics[keepaspectratio=true,height=#3]{#1}
        \includegraphics[keepaspectratio=true,height=#3]{#2}
      }
    \end{center}
    \caption{#4}
    \label{#5}
  \end{figure}
}

\newcommand{\inclfiguress}[6]{
  \begin{figure}[hbt]
    \begin{center}
      \fbox{
        \includegraphics[keepaspectratio=true,height=#4]{#1}
        \includegraphics[keepaspectratio=true,height=#4]{#2}
        \includegraphics[keepaspectratio=true,height=#4]{#3}
      }
    \end{center}
    \caption{#5}
    \label{#6}
  \end{figure}
}

\lstset{language=C++,tabsize=4,frame=single,numbers=left, framerule=0.4pt,frameround=tttt,basicstyle=\tiny}

%% Title Page
%\title{Entwicklung eines Systems zur Visualisierung von 3D--Volumendaten auf Consumer--Hardware}
%\author{Enrico Reimer\\
%{Max-Planck-Institut f�r Kognitions- und Neurowissenschaften,}\\
%{Stephanstrasse 1a, 04103 Leipzig, Deutschland}
%%\thanks{Danke an alle :-)}
%}



\makenomenclature
\renewcommand{\nomname}{Glossar}

\fancypagestyle{plain}{               % change plain style, e.g. 1st page of chapter
 \fancyhf{}                           % clear all plain header and footer fields
 \renewcommand{\footrulewidth}{0pt}   % put a 0.5pt thick line over the footer
 \renewcommand{\headrulewidth}{0pt}   % put no line under the header 
}

\linespread{1.15}
\begin{document}
\nomenclature{GL-Raum}{virtueller, durch eine \fachw{OpenGL}-Sicht
  dargestellter Raum, in den zu zeichnende Objekte platziert werden.
  Der GL-Raum ist theoretisch unendlich, es wird jedoch nur ein, durch
  die Sicht definierter Teil, gezeichnet. }

\nomenclature{Texturraum}{virtueller Raum innerhalb des GL-Raumes, der
  durch eine dreidimensionale Textur geformt wird. Hier auch
  vereinfachend als Datenraum bezeichnet.}

\nomenclature{GUI}{``graphical user interface'' bzw. ``Grafische Benutzungsschnittstelle''\\
  System das grafisch mit dem Nutzer �ber Fenster, Schaltfl�chen,
  Dialogboxen u.�. interagiert. Wird allgemein als intuitiver als die
  Interaktion mittels Kommandos angesehen. }


\nomenclature{Kontext}{Hier meist \fachw{OpenGL}-Kontext.\\
  Identifiziert eine Instanz des \fachw{OpenGL}-Renderers. Der Kontext
  ist fest mit reservierten Ressourcen des Systems, sowie dem Fenster
  in das der Renderer zeichnet, verbunden.}



\nomenclature{Widget}{Kurzform von ``Window Gadget''\\
  Frei �bersetzt: ``Dingsbums f�r Fenster''.  Bezeichnet in der
  Programmierung allgemein Funktionen oder Objekte, die zum Darstellen
  in Grafischen Nutzerschnittstellen verwendet werden.}

%\nomenclature{WM}{Window Manager bzw. Fenstermanager\\
%System, das die Darstellung der Fenster einer Grafischen
%Nutzerschnittstelle beeinflusst und die Eingabeger�te (Tastatur, Maus
%usw.) dem jeweils aktiven Fenster zuordnet. Sowohl der �u�ere Rahmen
%als auch die Position der prim�rer Fenster werden vom WM kontrolliert.}

\nomenclature{voxel}{Kurzform f�r Volumenpixel\\
  Dreidimensionale Version des Pixels. Ein Voxel repr�sentiert ein
  Datum in einem Volumendatensatz. Voxel sind hier nicht
  gezwungenerma�en kubisch. }

\nomenclature{CBS}{Max-Planck-Institut f�r Kognitions- und
Neurowissenschaften\\
(engl.: Max Planck Institute for Human Cognitive and Brain Sciences ) }


\nomenclature{vista}{Die Vista Bibliothek\\
  Eine Funktionsbibliothek f�r die wissenschaftliche Arbeit mit
  graphischen Daten im weitesten Sinne. Im CBS vor allem f�r den
  Datenaustausch verwendet. (siehe auch Abschnitt \vref{sec:volumetex}
  und Quellenverweis \cite{vista})}


\nomenclature{mesa}{auch Mesa3D\\
  Vor allem im UNIX-Bereich verbreiteter \fachw{Softwarerenderer} mit
  modularer Architektur. (siehe auch Abschnitt \vref{def:renderer} und
  Quellenverweis \cite{mesa}) }

\nomenclature{OpenGL}{Open Graphics Library\\
  Urspr�nglich von \cite{sgi} als IrisGL entwickelte
  Softwareschnittstelle f�r Vektorgrafik. Inzwischen als OpenGL
  standardisiert.\\
  (siehe auch Kapitel \vref{opengl})}

\nomenclature{API}{Application Programming Interface\\
  Die Schnittstelle, die ein Softwaresystem anderen Programmen zur
  Verf�gung stellt.}

\nomenclature{ARB}{Architecture Review Board\\
  Hier das Standardisierungsgremium des \fachw{OpenGL}\hyp{}Standards. \\
  (siehe auch Kapitel \vref{opengl})}

\nomenclature{DVR}{direct volume rendering\\
  Renderingmethode bei der Volumendaten direkt mittels
  \fachw{Raytracing} abgebildet werden.  (siehe auch Abschnitt
  \vref{desing:volume}) } 

\nomenclature{SF}{``surface fitting'' bzw.
  Oberfl�chenrendering\\
  Renderingmethode bei der Volumendaten indirekt mittels
  Polygonisierung dargestellt werden.  (siehe auch Abschnitt
  \vref{desing:volume})}

\nomenclature{Raytracing}{auch Strahlverfolgung\\
  Ein Rendering-Verfahren zur Erzeugung meist fotorealistisch
  wirkender Bilder. F�r jedes Pixel der Zielbildes wird mindestens ein
  Strahl in die Szene gesandt. Die ``Bedingungen'', die der Strahl
  dort ``vorfindet'' bestimmen die Farbe des Zielpixels.}

\nomenclature{MRI}{Magnetresonanztomografie, auch
  Kernspintomografie oder Kernspin\\
  Ein bildgebendes Verfahren zur Darstellung von Strukturen im Inneren
  des K�rpers, das auf dem magnetischen Moment von Atomkernen beruht.\\
  (siehe auch Abschnitt \vref{MRI}) }

\nomenclature{fMRI}{funktionelle Magnetresonanztomografie
  oder Kernspintomografie\\
  kernspintomografisches Verfahren das unter Verwendung des
  BOLD\hyp{}Verfahrens �ber die Sauerstoffanreicherung R�ckschl�sse auf die
  Aktivit�t des entsprechenden Gebietes zieht.\\ (siehe auch Abschnitt
  \vref{MRI}) }

\nomenclature{BOLD}{Blood Oxygen Level Dependency\\
  Von \citet{Ogawa:bold} entwickeltes kernspintomografisches Verfahren,
  das die 1935 von Linus Pauling und Charles D. Coryell entdeckte
  paramagnetische Eigenschaft von H�moglobin ausnutzt, um die
  Sauerstoffanreicherung im Blut zum messen.\\ (siehe auch Abschnitt
  \vref{MRI}) } 

\nomenclature{Renderer}{vom englischen ``to render''
  d.h. ``machen''\\
  Hier ist damit die Generierung von Pixelbildern aus abstrakten
  Eingabedaten gemeint. Als \fachw{Renderer} werden Systeme
  bezeichnet, diese Funktion ausf�hren. }

\nomenclature{rastern}{Abbilden beliebiger Bilddaten in
  diskreten Puffern\\
  Bezeichnet eine Operation, die meist stetige zweidimensionale Bilder
  in zweidimensionale Pixelpuffer (z.B. Bildpuffer einer Grafikkarte)
  abbildet, in diesem Fall ist $f:\menge{R}^2 \mapsto \menge{N}^2$.\\
  (siehe auch Abschnitt \vref{opengl:arch}) }

\nomenclature{MRT}{Magnetresonanztomografie\\
  (siehe auch MRI)}

\nomenclature{overhead}{Daten, die nicht prim�r zu den Nutzdaten
  z�hlen, sondern als Hilfsdaten zur �bermittlung oder Speicherung
  ben�tigt werden.}

\nomenclature{dual-core}{Prozessoren mit zwei Rechenwerken\\
Dual-Core-Prozessoren werden von vielen Herstellern als Weg gesehen,
die Leistung von Desktopsystemen ohne Erh�hung der Taktrate zu
steigern. Sowohl AMD, als auch Intel haben Dual-Core-Prozessoren angek�ndigt oder  bereits
 im Angebot.}

\nomenclature{GPU}{Graphic Processing Unit oder Visual Processing Unit (VPU)\\
Prozessor, der auf graphische Operationen wie
z.B. \fachw{OpenGL}\hyp{}Operationen spezialisiert ist. 
(siehe auch Abschnitt \vref{opengl})
}

\nomenclature{Direct3D}{3D-Modul von DirectX\\
\fachw{OpenGL} nachempfundene, aber objektorientierte Schnittstelle
f�r die Programmierung dreidimensionaler Systeme. Stark an MS-Windows
gebunden, und daher nur dort verf�gbar.
}

\nomenclature{Qt}{``The Qt application development framework'' \\
Eine umfangreiche und sehr m�chtige Systembibliothek mit
Widgetsystem, die f�r zahlreiche Plattformen verf�gbar ist. 
Die API ist plattformunabh�ngig.
}
%%%% Generate Title Page %%%%%%%%%%%%%%%%%%%%%%%%%%%%%%%%%%%%%%%%%%%%%%%%%%%%%%%%%%% 
 
\begin{titlepage}
%  \sffamily
        \begin{center}
          \vspace*{-15mm}
    \normalsize
    Hochschule f�r Technik, Wirtschaft und Kultur Leipzig (FH)\\
    Fachbereich Informatik, Mathematik und Naturwissenschaften\\
    Studiengang Informatik
    
    \vspace{28mm}
    
    \LARGE Konzeption und Implementierung eines\\ Systems zur
    Visualisierung und Bearbeitung von 3D--Volumendaten auf
    Consumer--Hardware

    \vspace{24mm}
    
    \LARGE
    \textsc{Diplomarbeit}
    
    \vspace{6mm}
    
    \Large
    eingereicht von\\
    \vspace{1mm}
    \Large
    Enrico Reimer\\
    
    \vspace{50mm}       
    
    \normalsize
    \begin{tabular}{l@{\hspace{5mm}}l}
      Datum: & 28.1.2005 \\
      Matrikelnummer: & 29936 (00IN) \\
      eMail: & \email{enni_@t-Online.de} \\
      Gutachter (HTWK): & Prof. Dr.-Ing. Frank Jaeger \\
      Betreuer (CBS): & PD Dr. Gabriele Lohmann\\
    \end{tabular}
        \end{center}
        \normalfont
\end{titlepage}

%\maketitle
\thispagestyle{plain}
\section*{Erkl�rung}

Hiermit erkl�re ich an Eides statt, dass ich die vorliegende
Diplomarbeit selbst�ndig und nur mit den angegebenen Quellen und
Hilfsmitteln angefertigt habe. Alle Stellen, die den Quellen w�rtlich
oder inhaltlich entnommen wurden, sind als solche kenntlich gemacht
worden. Diese Arbeit hat in gleicher oder �hnlicher Form noch keiner
Pr�fungsbeh�rde vorgelegen.  \vspace{1.0\baselineskip}
\begin{center}
  \begin{tabular}{lr}
    \hspace{0.35\textwidth} & \hspace{0.35\textwidth} \\
    Leipzig, den 28.1.2005 & Enrico Reimer \\
  \end{tabular}
\end{center}

\vspace{9.0\baselineskip}
\section*{Danksagung}

Diese Arbeit entstand in Kooperation mit dem Max-Planck-Institut f�r
Kognitions- und Neurowissenschaften. An dieser Stelle m�chte ich mich
bei allen Kollegen f�r die Bereitstellung des Themas und der
Arbeitsmittel sowie f�r die fortw�hrende Unterst�tzung w�hrend der
Bearbeitung bedanken!

Au�erdem gilt mein Dank Herrn Prof.Dr. Jaeger von der HTWK, der
sich die Zeit nahm, Vorabversionen des Textes zu lesen, und der mit
fundierten Hinweisen zur Verbesserung der Diplomarbeit beitrug.

Eine gro�e Hilfe sind auch meine Beta-Leser, vor allem Nico und
Johannes gewesen.  Ohne ihre ausf�hrliche Fehlersuche und die
zahlreichen Tipps s�he meine Arbeit heute mit Sicherheit anders aus.

\begin{flushright} Enrico Reimer \end{flushright}
\begin{abstract} 

\begin{flushright}\begin{small}
\emph{Erkennen hei�t: zu der halben Wirklichkeit der Sinnenerfahrung die Wahrnehmung des Denkens hinzuf�gen, auf da� ihr Bild vollst�ndig werde.}\\
\textsc{J.W. von Goethe}
\end{small}\end{flushright}

  Neben der Genauigkeit des Messinstruments ist die schnelle und
  korrekte Interpretation des Gemessenen von zentraler Bedeutung f�r
  das Messen selbst. Moderne bildgebende Messsysteme erzeugen heute
  gro�e Mengen oft dreidimensionaler graphischer Informationen die von
  Fachleuten interpretiert werden m�ssen. Diese Diplomarbeit
  besch�ftigt sich mit dem Entwurf und der Implementierung eines
  Systems zur Visualisierung dreidimensionaler Messdaten, wie sie
  beispielsweise Kernspintomografen liefern.
\end{abstract} 

%\bibliographystyle{gerplain}
\bibliographystyle{plainnat_e}
%\bibliographystyle{jr-dinat}


\fancypagestyle{plain}{               % change plain style, e.g. 1st page of chapter
 \fancyhf{}                           % clear all plain header and footer fields
 \fancyhf[RF]{Seite \thepage}
 \fancyhf[LF]{3D--Volumendaten auf Consumer--Hardware}
                                      % put page number on center of footer
 \renewcommand{\footrulewidth}{0.5pt} % put a 0.5pt thick line over the footer
 \renewcommand{\headrulewidth}{0pt}   % put no line under the header 
}

\tableofcontents

\chapter{Einleitung}

Der Nutzen einer Messmethode h�ngt in gro�em Ma�e davon ab, wie gut
die erhaltenen Daten interpretiert und angewandt werden k�nnen.  In
der modernen Medizin liefern Kernspintomografen nicht nur punktgenaue
Informationen �ber anatomische Details unseres K�rpers, sie lassen
auch Erkenntnisse �ber nat�rliche Vorg�nge in ihm zu.  So ist durch
die funktionelle Kernspintomografie inzwischen messbar, welcher �u�ere
Reiz in welchem Ausma� zu erh�hten Aktivit�ten in bestimmten Teilen
des menschlichen Gehirns f�hrt. Obwohl sich die Messtechnik und die
Methoden zur automatischen Verbesserung der Daten rasant
weiterentwickeln, m�ssen diese Informationen noch immer von Menschen
interpretiert werden, welche oft ihre Probleme mit den komplexen
Informationen haben.  Es ist daher eine zugleich schwierige, aber auch
wichtige Aufgabe, die dreidimensionalen Messdaten (Volumendaten) f�r
den menschlichen Betrachter entsprechend aufzuarbeiten und geeignet
darzustellen.

Deshalb befasst sich diese Diplomarbeit mit dem Entwurf und der
Entwicklung eines sowohl einfachen, als auch flexiblen Systems zur
Visualisierung und Bearbeitung von Volumendaten.

\section{Motivation}

Die Bildgebungsverfahren der Kernspintomografie (MRI) und der
funktionellen Kernspintomografie (fMRI) haben in den letzten Jahren
gro�e technische Fortschritte gemacht. Die Aufl�sung von
Kernspintomografen steigt st�ndig, so dass immer genauere Daten
ermittelt, und Aktivit�ten im Gehirn immer genauer gemessen werden
k�nnen. Auch die mathematischen Modelle zur Erzeugung und Aufbereitung
der ermittelten Daten werden rapide weiterentwickelt. Ferner stehen im
Bereich der interaktiven Visualisierung und Verarbeitung der Daten
einige umfangreiche, professionelle Softwaresysteme zur Verf�gung.

\subsection{Verf�gbare Mittel der Visualisierung}

Diese Programme sind jedoch oft �berdimensioniert und bieten meist mehr
Funktionalit�ten als erforderlich. Neben der unn�tigen Erschwerung f�r
den Anwender erh�ht diese �bertriebene Komplexit�t den Bedarf an
Ressourcen und verteuert die Softwareprodukte gleichzeitig.
Au�erdem kommt es oft vor, dass trotz des hohen Funktionsumfanges
gerade das eine Feature, das dringend gebraucht wird, fehlt.  Eine
Anpassung ist meist nicht nur aus lizenzrechtlichen Gr�nden
problematisch, sondern oft auch aufgrund des Umfangs dieser Systeme
schlicht unm�glich. Was fehlt, ist ein einfaches, flexibles und
effektives System, das eine m�glichst direkte und intuitive
Visualisierung der Volumendaten erlaubt, und sich auf die
wesentlichsten Funktionen beschr�nkt. Gleichzeitig sollte es auf
g�ngiger PC\hyp{}Hardware fl�ssig und stabil betrieben werden k�nnen und
sich leicht anpassen bzw. erweitern lassen.

\subsection{Objekterkennung durch Segmentierung}

Die Erkennung von Strukturen ist eine der �blichsten Aufgaben bei der
Interpretation der Daten bildgebender Messsysteme. Die durch die
Kernspintomografie generierten Volumendaten bilden in diesem
Zusammenhang keine Ausnahme. Aus diesem Grunde finden beim
Max-Planck-Institut f�r Kognitions- und Neurowissenschaften (CBS)
mehrere Algorithmen zur Objekterkennung Anwendung. Diese hoch
spezialisierten Algorithmen lassen jedoch keinen Nutzereingriff zu.
Etwaige Erkennungsfehler m�ssen im Nachhinein m�hsam von Hand
korrigiert werden. Es w�re effektiver diese Fehler im Entstehen zu
unterbinden, d.h. den Rechner schon im Ansatz zu korrigieren, bevor
sich der gemachte Fehler ausweitet. 

\section{Zielsetzung}
Inhalt dieser Diplomarbeit ist der Entwurf und die Realisierung eines
Programms zur gemeinsamen Visualisierung anatomischer und
funktioneller Volumendaten mittels einer geeigneten
Darstellungsstrategie. Das dabei entwickelte Programm soll neben der
Visualisierung der Volumendaten Werkzeuge f�r eine interaktive
Objekterkennung innerhalb dieser Daten bieten.  Es wird daher n�tig
sein, Methoden zur interaktiven Segmentierung zu entwerfen, bei der
sich Anwender und Rechner gegenseitig unterst�tzen.  Dieses System
soll auf zurzeit g�ngigen B�ro\hyp{}PCs m�glichst ohne zus�tzliche
Kosten einsetzbar sein.  Daraus folgt, dass seine Anforderungen an die
Hardware m�glichst gering sein sollten und es vor allem ohne
Spezialhardware auskommen muss. Die Erweiterung und Anpassung dieses
Programms soll m�glichst leicht, und eine Portierung auf andere
Plattformen mit geringem Aufwand m�glich sein.

\section{Aufbau der Arbeit}

\begin{itemize}
\item Kapitel 2 beschreibt die allgemeinen Grundlagen
  dreidimensionaler Messungen und der Visualisierung der so
  ermittelten dreidimensionalen Messdaten auf zweidimensionalen
  Anzeigesystemen wie z.B. Bildschirmen.
 
\item Kapitel 3 behandelt den Entwurf, und die Implementierung eines
  auf \fachw{OpenGL} basierenden allgemeinen Basissystems zur
  Visualisierung dreidimensionaler Szenen. Es werden auch die
  n�tigen mathematischen Grundlagen dieser Visualisierung besprochen.
  
\item Kapitel 4 beschreibt den Aufbau einer, auf den in Kapitel 3
  angestellten �berlegungen basierenden Basisbibliothek. Diese
  Bibliothek wird die Implementierung und sp�tere Pflege des
  eigentlichen Programms erleichtern.
  
\item Kapitel 5 schildert die theoretischen Grundlagen einer auf
  ``Wasserscheiden'' basierenden Segmentierung im Raum, w�hrend Kapitel
  6 zwei Implementierungen dieses Verfahrens behandelt.
  
\item Kapitel 7 behandelt die auf den vorherigen Arbeiten basierende
  Implementation des eigentlichen Visualisierungs- und
  Bearbeitungssystems und Kapitel 8 schlie�t diese Arbeit mit einer
  Zusammenfassung ab.

\end{itemize}

\chapter{Grundlagen und Vor�berlegungen}

\section{Die Magnetresonanztomografie}
\label{MRI}

Der folgende kurze Abriss der Geschichte der Magnetresonanztomografie
basiert haupt\-s�chlich auf einer Zusammenstellung von \citet{mri} und
Informationen aus der \citet{wikipedia:mri}.

Felix Bloch und Edward Purcell entdeckten 1946 unabh�ngig voneinander
das Prinzip der kernmagnetischen Resonanz (NMR). Wie so oft in der
Grundlagenforschung wurde anfangs keine Anwendung f�r diesen
physikalischen Effekt gesehen, und die Entdeckung blieb relativ
unbeachtet. Erst ab 1950 wurde die NMR wieder aufgegriffen und f�r
chemische und physikalische Analysemethoden weiterentwickelt.  1952
wurde Felix Bloch und Edward Purcell f�r ihre Entdeckung den
Nobelpreis f�r Physik verliehen.

Eine Anwendung in der Medizin fand die NMR erst 1971, als Raymond
Damadian nachweisen konnte, dass sich die Magnetresonanzen von
gesundem Gewebe und von Tumoren unterscheiden.  1972 wurde die auf
R�ntgenstrahlen basierende Computertomografie (CT) eingef�hrt. Das
hatte zwar nichts mit der NMR zu tun, zeigte aber, dass Krankenh�user
durchaus gewillt waren, gro�e Geldmengen f�r medizinische
Untersuchungsger�te auszugeben. Noch im selben Jahr leitete Paul
Lauterbur von der CT ein auf NMR basierendes Verfahren ab und konnte
erste Bilder von kleinen Gewebeproben erzeugen.  Der Grundstein der
heutigen MRI wurde jedoch erst 1975 von Richard Ernst gelegt, als
dieser ein auf der Fourier\hyp{}Transformation basierendes
NMR\hyp{}Bildgebungsverfahren vorstellte.  Anders als die CT hat die
MRI keine bekannten Nebenwirkungen und kann relativ problemlos, auch
in kurzen Zeitabst�nden, an jedem Teil des K�rpers durchgef�hrt
werden.

Aufgrund des hohen technischen und mathematischen Aufwands konnten
lange Zeit nur kleine Bilder angefertigt werden.  Erst die rasante
Entwicklung der Rechentechnik erm�glichte es ab 1980 mittels Ernsts
Technik innerhalb von 5 Minuten komplette Schnittbilder des K�rpers zu
erzeugen. Bis 1986 wurde diese Zeit weiter auf 5 Sekunden reduziert.
Au�erdem konnten erstmals mehrere Schnitte nacheinander abgebildet und
zu den heute �blichen Volumendaten zusammengefasst werden.  1991
erhielt Richard Ernst f�r seine Arbeit den Nobelpreis f�r Chemie.

Die ein Jahr sp�ter entwickelte funktionelle MRI (fMRI) verwendet das
von \citet{Ogawa:bold} publizierte BOLD\hyp{}Verfahren um die
Sauerstoffanreicherung von Blut kernspintomografisch zu erfassen. Da
sich das Verh�ltnis von oxigeniertem zu deoxigeniertem Blut zusammen
mit der Erh�hung der Aktivit�t in bestimmten Gehirnregionen �ndert,
l�sst sich diese Aktivit�t zeitnah messen.  Diese gemessenen
Aktivit�ten lassen sich dann in direkten zeitlichen Zusammenhang mit
�u�eren Reizen bringen.  Mit Hilfe der fMRI kann somit eine
dreidimensionale Aktivit�tskarte des Gehirns angefertigt werden.


\section{Visualisierung dreidimensionaler Daten}

In den letzten Jahren ist die Computergrafik zu einem allt�glichen
Werkzeug in der Wissenschaft geworden. Vor allem die computergest�tzte
Visualisierung, im Folgenden allgemein als \fachw{Renderings}
bezeichnet, ist f�r die Darstellung gro�er und komplexer Datenmengen
ein gern verwendetes Mittel. Dies trifft besonders auf die
Visualisierung dreidimensionaler Daten zu. Die zwei wichtigsten
Methoden des \fachw{Renderings} dreidimensionaler Daten sind das
\fachw{volume rendering} und der \fachw{Schnitt}.

\subsection{\fachw{Volume rendering}} 

Da dreidimensionale Datens�tze meist das Ergebnis einer Messung an
einem nat�rlichen dreidimensionalen Objekt sind, also dieses Objekt in
einer bestimmten Weise repr�sentieren, liegt es nahe dieses Objekt
auch in seiner nat�rlichen Form darzustellen. Zu diesem Zweck wird
jedes Datum an dem Punkt im Anzeigeraum abgebildet, an dem es im
reellen Raum gemessen wurde.  Diese nur in Bereich der
computergest�tzten Visualisierung verf�gbare Methode wird als
\fachw{volume rendering} bezeichnet. Sie liefert ein naturgetreues
Abbild des untersuchten Objektes, das f�r den Betrachter leicht zu
erfassen ist.


\subsection{Der \fachw{Schnitt}} 

Bei der optischen Mikroskopie wird oft aus dem zu untersuchenden
Objekt eine d�nne Scheibe herausgeschnitten. Diese Scheibe wird
stellvertretend f�r das Objekt untersucht, da sich eine d�nne Scheibe
leichter durchleuchten l�sst als das ganze Objekt.  Innerhalb der
Probe kommt es au�erdem kaum zu �berlagerungen, da sie so d�nn ist.

In der Regel werden mehrere Schnitte angefertigt, um so die innere
Struktur des Objektes zu bestimmen. Um den mathematischen Aufwand f�r
R�ckschl�sse aus der Position im Schnitt auf die Position in der Probe
klein zu halten, werden Schnitte in der Regel parallel zu einer der
Koordinatenebenen durchgef�hrt, sie stehen also senkrecht auf einer
der Koordinatenachsen. Auf dieser Achse l�sst sich die Tiefe des
untersuchten Schnittes leicht ablesen. Die anderen beiden Achsen
entsprechen genau denen aus dem zweidimensionalen Koordinatensystem
des Schnittes.


%\section{Untersuchung einiger aktuell verf�gbarer Visualisierungstools}

\section{Visualisierung mittels \fachw{OpenGL}}
\label{opengl}

Der Einsatz von \fachw{OpenGL} auf herk�mmlichen PCs spielt in der
Umsetzung des Visualisierungssystems eine zentrale Rolle. Es zahlt
sich daher aus diesen Bereich der Computergrafik etwas genauer zu
betrachten.  Kommende �berlegungen und Entscheidungen in Entwurf und
Implementierung basieren auf diesen Informationen.

\subsection{\fachw{OpenGL} als universelle
  Visualisierungsschnittstelle}

\fachw{OpenGL} ist eine Softwareschnittstelle zu einem beliebigen
grafischen Anzeigesystem, im Folgenden \fachw{Renderer} genannt. Das
System folgt dabei einem Client\hyp{}Server\hyp{}Konzept, wobei der
\fachw{Renderer} den Server darstellt und die Grafikanwendung den
Client. F�r die meisten gebr�uchlichen Sprachen sind
\fachw{OpenGL}-Funktionsbibliotheken verf�gbar. Die Funktionen, die
diese Bibliotheken zur Verf�gung stellen, bilden die standardisierte
Programmierschnittstelle (API) von \fachw{OpenGL}. Diese
\fachw{OpenGL-API} wird oft vereinfachend als \fachw{OpenGL}
bezeichnet. Da \fachw{OpenGL} in hohem Ma�e standardisiert ist, lassen
sich verschiedenste \fachw{Renderer} zur Darstellung verwenden, ohne
dass die Anwendung neu kompiliert werden muss.

Der \fachw{OpenGL}\hyp{}Standard wird von einen Standardisierungsgremium,
dem ``Architecture Review Board'' (ARB) gepflegt und �berwacht, und
die \fachw{OpenGL\hyp{}API} steht auf allen �blichen Plattformen zur
Verf�gung.  Die Referenzdokumentation, das ``Bluebook'' wird
regelm��ig aktualisiert vom ARB herausgegeben. Zur Zeit ist die von
\citet{ref:openGL} �berarbeitete Version 1.5 am gebr�uchlichsten.
Anders als z.B. bei \fachw{Direct3D} wird durch die Verwendung von
\fachw{OpenGL} sowohl auf Client- als auch auf Serverseite die Bindung
an einen Hersteller oder an eine Plattform vermieden. Das ARB
spezifiziert genau die Vorbedingungen und die Nachbedingungen jeder
API\hyp{}Funktion. Es wird jedoch bewusst auf Aussagen zu ihrer
Implementierung verzichtet. Inzwischen ist ein gro�er Teil der
Funktionalit�t von \fachw{OpenGL} direkt in die Hardware von
Grafikkarten �bernommen worden. Einige Anbieter verwenden auch spezielle
Multimediabefehlss�tze moderner Prozessoren. \fachw{OpenGL} stellt somit eine
in vielen F�llen hardwarebeschleunigte, jedoch in jedem Fall
verl�ssliche und standardisierte Schnittstelle bereit.

\subsection{Der \fachw{Renderer}}
\label{def:renderer}

Gezeichnet wird bei der \fachw{OpenGL}\hyp{}Architektur immer durch
den \fachw{Renderer}, der Hauptbestandteil des Servers ist. Server und
\fachw{Renderer} werden deshalb oft vereinfachend gleich gesetzt. Die
meisten Renderer sind mit Grafikkarten verwoben, da das Gezeichnete in
den meisten F�llen auf Bildschirmen angezeigt werden soll.  Durch die
�bernahme von m�glichst vielen Bestandteilen eines Renderers direkt in
die Hardware der Grafikkarte soll eine m�glichst hohe Leistung beim
Zeichnen erreicht werden. Aufgrund der hohen Komplexit�t der
umfangreichen \fachw{OpenGL\hyp{}API} ist dies sehr aufwendig, und
vollst�ndig hardwareimplementierte Renderer sind nur in wenigen teuren
Spezialgrafikkarten zu finden. Die meisten Hersteller gehen statt
dessen einen Kompromiss ein, indem sie seltener verwendete oder zu
komplexe Teile des \fachw{Renderers} in den Treiber verlagern.  Diese
vor allem im Consumer\hyp{}Bereich �bliche L�sung hat auch den
Vorteil, dass die Grafikkarte relativ leicht weitere
Grafikschnittstellen unterst�tzen kann. Alle Grafikschnittstellen
basieren auf sehr �hnlichen Grundkonzepten, so dass sie viele
Funktionalit�ten gemeinsam haben. Dieser gemeinsame Nenner wird von
den Herstellern in ihre Hardware implementiert, w�hrend der Treiber
als Mittler fungiert und Unterschiede ausgleicht.

Allgemein lassen sich \fachw{Renderer} in die zwei Gruppen
\fachw{Softwarerenderer} und \fachw{Hardwarerenderer} unterteilen.
Nur komplett in Software implementierte \fachw{Renderer} werden
gemeinhin als \fachw{Softwarerenderer} bezeichnet. Als
\fachw{Hardwarerenderer} gelten dagegen �blicherweise schon Renderer
die nur den wichtigsten Teil ihrer Funktionalit�t in Hardware
ausf�hren. Da die Implementation komplexer geometrischer Algorithmen
in Hardware sehr viel aufwendiger ist als in Software, galt der Preis
einer Grafikkarte lange Zeit als Ma� des Anteils
hardwareimplementierter Funktionalit�t in dieser Grafikkarte.

Mit der Massenproduktion im Consumer\hyp{}Markt ist der Anteil der
Implementationskosten f�r den \fachw{Renderer} und somit der Anteil
der Entwicklungskosten einer Grafikkarte gemessen an den Produktkosten
jedoch erheblich geschrumpft. Consumer\hyp{}3D\hyp{}Grafikkarten kann
so trotz hoher Entwurfskosten immer mehr Funktionalit�t direkt in die
Hardware integriert werden. Gleichzeitig entstehen neue Priorit�ten,
welche Funktionalit�t in Hardware verf�gbar sein sollte.  Da
Consumer\hyp{}3D\hyp{}Grafikkarten praktisch ausschlie�lich auf Spiele
ausgelegt sind, wird vielen Funktionen professioneller
\fachw{Renderer} zugunsten anderer, f�r Spiele wichtigerer Funktionen
weniger Bedeutung beigemessen. Zum Beispiel werden im professionellen
Bereich oft Drahtgitter und einfarbige Fl�chen dargestellt. Im
Gegensatz dazu liegt der Fokus bei Spielen vor allem bei texturierten
Fl�chen.  Dies zeigt sich im Verhalten der verschiedenen Grafikkarten.
W�hrend Consumer\hyp{}\fachw{Renderer} zum Beispiel beim Zeichnen von
Drahtgittermodellen extrem schlechte Leistungen zeigen, zeichnen sie
die eigentlich aufw�ndigeren texturierten Fl�chen mit enormer
Geschwindigkeit.

\fachw{Hardwarerenderer} f�r den Consumer\hyp{}Markt werden zur Zeit
fast ausschlie�lich von \citet{nvidia} oder \citet{ati} geliefert.
Die von nVidia zentral angebotenen Treiber sind schon l�ngere Zeit f�r
verschiedene 32Bit und 64Bit Plattformen verf�gbar. Im
Unix\hyp{}Bereich waren daher bisher vor allem
nVidia\hyp{}\fachw{Renderer} verbreitet. ATI verbessert jedoch seit
kurzem seine bisher schwache Unterst�tzung von
nicht\hyp{}Windows\hyp{}Betriebssystemen und bietet inzwischen auch
64-Bit\hyp{}Treiber an.

\fachw{Softwarerenderer} sind meist mit dem Betriebssystem gelieferte
Ersatzrenderer, die die Funktionsf�higkeit \fachw{OpenGL}\hyp{}basierter
Anwendungen sicherstellen sollen.  Neben den herstellerspezifischen
\fachw{Softwarerendereren} ist vor allem auf Unix\hyp{}Systemen das
unter der \citet{MIT:license} stehende Mesa \cite{mesa} stark verbreitet.

Mesa zeichnet sich vor allem dadurch aus, dass es den Renderer
nocheinmal in eine Schnittstelle (auf die die \fachw{OpenGL\hyp{}API}
aufsetzt) und einen ausf�hrenden Teil unterteilt. Die
softwareimplementierte Funktionalit�t des ausf�hrenden Teils steht
dabei als Satz leicht austauschbarer Funktionszeiger zur Verf�gung.
Einige ``Treibermodule'' f�r Grafikkarten, die selbst entweder keine
oder nur eine unzureichende \fachw{OpenGL}\hyp{}Schnittstelle haben, machen
sich das zu nutze.  Ist es bei einer Grafikkarte m�glich, eine
bestimmte Funktionalit�t hardwarebeschleunigt zu implementieren, dann
ersetzen sie den Zeiger auf die entsprechende reine Software\hyp{}Funktion
durch den Zeiger auf ihre eigene, ganz oder teilweise
hardwarebeschleunigte Funktion. Dies ist z. B. oft bei den f�r das
Rastern zust�ndigen Funktionen m�glich, da hier nur noch
zweidimensional gezeichnet wird.  In der Regel kann jede moderne
Grafikkarte zweidimensionale Zeichenoperationen, wie z.B. das Zeichnen
von Polygonen u.�., hardwarebeschleunigt ausf�hren.  Mesa ist also
nicht einfach nur ein \fachw{Softwarerenderer}, sondern kann
auch ein Mittel sein, ``schwacher'' oder inkompatibler Hardware
mittels Kompatibilit�tsschicht entgegen zu kommen.

\subsection{Die Architektur von \fachw{OpenGL}}
\label{opengl:arch}

Die \fachw{OpenGL\hyp{}API} abstrahiert den \fachw{Renderer} zu einer
Zustandsmaschine, die auf gegebenen Ortsvektoren aus $R^3$
Transformationen mittels mehrerer Matrizen anwendet.  Auf diesem
mathematischen Wege werden jegliche Tranformationen des zu zeichnenden
Ortsvektors $\left( \menge{R}^3 \mapsto \menge{R}^3 \right)$ sowie
perspektivische Verzerrung und Projektion auf die Projektionsfl�che $\left(
  \menge{R}^3 \mapsto \menge{R}^2 \right)$ realisiert. Die Projektionsfl�che
kann vereinfachend mit der Oberfl�che des Bildschirms gleichgesetzt
werden. Da die meisten Darstellungsmedien diskrete Koordinatensysteme
verwenden, werden die Ergebnisse der Projektion gegebenenfalls noch
gerastert $\left( \menge{R}^2 \mapsto \menge{N}^2 \right)$.  Die so
ermittelten Vektoren aus $\menge{N}^2$ werden dann verwendet, um
Farbinformationen in das entsprechende Darstellungsmedium zu
schreiben.  Diese Farbinformationen k�nnen entweder direkter oder
indirekter Natur sein.  Direkte Farbinformationen werden aus einem
vorbereiteten n-dimensionalen Puffer, einer Textur, gelesen.  Mittels,
durch Texturkoordinaten bestimmten Matrixoperationen $\left(
  \menge{R}^n \mapsto \menge{R}^3 \right)$ werden diese Farbwerte dann
in den Raum projiziert.  Indirekte Farbinformationen sind durch
lineare Funktionen interpolierte Farbwerte, wobei diese Funktionen
durch den Ortsvektor parametrisiert werden.  Dazu kommen noch die so
genannten Shader, kleine Programme, die �hnlich einer Textur den
Farbwert f�r jedes einzelne Pixel bestimmen.   Da diese
Shaderprogramme  als Algorithmus f�r jeden einzelnen dargestellten
Punkt ausgef�hrt werden m�ssen, sind sie sehr rechenzeitaufwendig und
werden deshalb �blicherweise direkt vom Grafikchip ausgef�hrt.  Die
durch diese drei Methoden bestimmten Farbwerte lassen sich mittels
\fachw{Multitexturing} beliebig kombinieren.


\chapter{Allgemeiner Entwurf}
\label{cha:design}

\section{Auswahl der Visualisierungsmethode}

Die Visualisierung der Volumendaten ist zentraler Bestandteil des zu
entwerfenden Systems, daher spielt die Auswahl der Methode der
Visualisierung eine entscheidende Rolle.  Im Folgenden sollen die Vor-
und Nachteile der beiden vorgestellen Visualisierungsmethoden
\fachw{Schnitt} und \fachw{volume rendering} betrachtet, und daraus
die Geeignetere ermittelt werden.

\subsection{Das \fachw{volume rendering}}
\label{desing:volume}

Wie bereits erw�hnt, werden beim \fachw{volume rendering} die
dreidimensional bestimmten Daten auch wieder dreidimensional
abgebildet $\left( \mathbb{R}^3_{reel} \mapsto \mathbb{R}^3_{virt}
\right)$. Diese ``naturgetreue'' Abbildung ist sehr intuitiv, da der
Betrachter seine, aus der reellen Welt stammenden, Begrifflichkeiten
und Denkweisen ohne Abstraktion �bernehmen kann und sich so schnell
``zurechtfindet''. 

Es existieren zwei �bliche Arten des \fachw{volume renderings}.  Beide
werden auf Volumendaten angewendet, wobei jedem Datum ein Voxel in
der dreidimensionalen Szene entspricht. 

Das direkte \fachw{volume rendering} (DVR) wie es z.B. von
\citet{Levoy:1990:ERT} beschrieben wird, basiert auf dem
\fachw{Raytracing}. Der Vorteil dieses Verfahrens ist, dass es keine
Fl�chen ben�tigt, sondern die Voxelhaufen der Szene direkt darstellen
kann.  Es ist somit m�glich transparente, oder nicht scharf
abgegrenzte K�rper nat�rlich darzustellen. Direktes \fachw{volume
  rendering} ist jedoch sehr aufwendig, da prinzipiell jedes einzelne
Datum bzw.  jeder einzelne Voxel ber�cksichtigt werden muss. Es gibt
zwar Ans�tze dies hardwarebeschleunigt auszuf�hren (
\citet{Roettger:2003:hwdvr}), jedoch ist interaktives Arbeiten selbst
dann nur mit extrem leistungsf�higer Hardware m�glich.  Der in dieser
Arbeit geplante breite Einsatz am Arbeitsplatz w�re derzeit nicht
realisierbar.

Ein weiterer Aspekt ist, dass sich bei diesem Verfahren, unabh�ngig
vom Darstellungsmedium, immer mehrere Daten �berlagern werden. Auch
wenn verdeckte Daten (z.B. bei Transparenten K�rpern) mit in die
Darstellung einflie�en findet ein Informationsverlust bzw. eine
Verf�lschung statt. Es kann nur ``in das Objekt hineingeblickt''
werden, wenn irrelevante Daten ganz oder teilweise ausgeblendet
werden. Wenn also die Informationen selektiv reduziert werden, und so
der ``Blick'' auf die relevanten Daten freigeben wird.  Dazu muss
selbstverst�ndlich erst einmal bekannt sein, welche Daten irrelevant sind.
Dass relevante Daten relevante Daten �berlagern kann jedoch auch so
nicht verhindert werden.

Die zweite Methode, das Oberfl�chenrendering (SF), basiert auf einer
bin�ren Segmentierung des Datenraums. F�r jedes Voxel wird bestimmt,
ob es Teil des Objektes ist, oder nicht. Transparente oder diffuse
Objekte sind nicht m�glich. Auch hier findet eine selektive
Reduzierung der Daten statt. Zwischen den auf diese Weise getrennten
Voxelmengen wird durch den \fachw{marching cubes}-Algorithmus
\cite{Lorensen:1987:MCH} eine Polygonfl�che erzeugt. Diese Fl�che
entspricht der Oberfl�che des K�rpers. Sie kann konventionell, also
auch hardwarebeschleunigt gerendert werden. In dieser Arbeit geht es
jedoch um Daten von Messungen innerer Strukturen, w�hrend die
Oberfl�che g�nzlich uninteressant ist. Auch die zweite Art des
\fachw{volume renderings} ist daher ungeeignet.

Letztendlich gibt es praktisch gesehen kein Medium f�r wirkliche
dreidimensionale Darstellung.  Auch die menschliche Netzhaut, quasi
als letzte Instanz, ist nur zweidimensional. Jegliche r�umliche
Informationen werden vom Gehirn ``interpoliert''.  Bei der �blichen
Darstellung auf einem Bildschirm oder auf Papier gehen ohnehin
s�mtliche Tiefeninformation verloren. Eine verlustfreie direkte
Darstellung von Volumendaten ist daher kaum m�glich. Zusammenfassend
kann gesagt werden, dass das \fachw{volume rendering} f�r die genannte
Aufgabe der interaktiven Visualisierung von Volumendaten entweder zu
langsam oder zu verlustreich ist.


\subsection{Der \fachw{Schnitt}}

Der \fachw{Schnitt} ist eine in der wissenschaftlichen Praxis bew�hrte
und beliebte Methode. Er bietet auf relativ einfache Weise Einblick in
das untersuchte Objekt. Die dabei stattfindende Reduzierung der
sichtbaren ``Daten'' auf einen kleinen Ausschnitt kann durchaus als
Reduktion der Informationen verstanden werden.  Anders als beim
\fachw{volume rendering} ist diese Reduktion jedoch nicht selektiv,
sondern unterliegt einer einfachen linearen Formel. Das ist kein
Nachteil, da, wie bereits er�rtert, die selektive Reduktion ohnehin
nicht zweckm��ig ist.  Durch das Anfertigen mehrerer Schnitte, lassen
sich die vorher ``aussortierten'' Daten gezielt sukzessiv erg�nzen.
Hierbei werden �berlagerungen durch die Ausschlie�lichkeit der Methode
von vornherein vermieden. Elemente, die bereits Teil eines vorher
angefertigten Schnittes sind, k�nnen nicht mehr Teil des Objektes
sein.  Eine entsprechend geordnete Menge von Schnitten l�sst sich
leicht auch dreidimensional interpretieren, obwohl der einzelne
Schnitt nur zweidimensionale Informationen liefert.

Das Konzept des Schnittes l�sst sich leicht in die Computergrafik
�bertragen, indem eine gedachte Ebene durch den Datensatz und damit
durch das Objekt gelegt wird. Werden anschlie�end alle Daten
angezeigt, die auf dieser Ebene liegen, entspricht das einem
virtuellen \fachw{Schnitt} durch das Objekt.  Dabei kann immer nur ein
Datum an einer bestimmten Stelle der Ebene liegen. �berlagerungen, und
damit ein Verlust von Information bei der Darstellung werden auf diese
Weise von vornherein ausgeschlossen. Der \fachw{Schnitt} eignet sich
zudem besser f�r die Darstellung auf zweidimensionalen Medien wie
Bildschirmen oder Papier.

Der Zeichenaufwand f�r einen \fachw{Schnitt} ist erheblich geringer
als beim \fachw{volume rendering}, da nur ein Bruchteil der Daten
gezeichnet wird. Zur Bestimmung der zu zeichnenden Punkte einer Ebene
reicht die einfache lineare Gleichung $\vec{r} = \vec{r_0} + \lambda
\vec{u} + \mu \vec{v}$ aus.  Dabei beschreiben \eRaum{\vec{r_0}} den
Ortsvektor eines Punktes auf der Ebene, \eRaum{\vec{u}} und
\eRaum{\vec{v}} Vektoren auf der Ebene, und \eR{\lambda} und \eR{\mu}
beliebige skalare Faktoren.  S�mtliche Punkte, deren Ortsvektor
\eRaum{\vec{r}} diese Gleichung erf�llt, geh�ren zu der Ebene und
m�ssen somit gezeichnet werden.

Zwar geht auch bei diesem Verfahren die direkte Information �ber die
Tiefe verloren, sie l�sst sie sich aber relativ leicht ermitteln. Die
Tiefe eines jeden angezeigten Datums entspricht genau der Tiefe seines
Ortsvektors \eRaum{\vec{r}}, der sich wiederum leicht mittels obiger
Gleichung bestimmen l�sst.  Analog zum physischen Anfertigen mehrerer
Schnitte, l�sst sich die Schnittebene durch Anpassung der Parameter
$\vec{r_0}$, $\vec{u}$ und $\vec{v}$ verlagern. Es lassen sich auch
beliebig viele \fachw{Schnitte} gleichzeitig beliebig in den Datenraum
legen.

Wie bereits erw�hnt, ist der Schnitt in der wissenschaftlichen Praxis
bekannt.  F�r die Zielgruppe (Mediziner) ist der \fachw{Schnitt} eine
vertraute Darstellungsform.  Er kann deshalb innerhalb dieses Umfeldes
als mindestens so intuitiv wie \fachw{volume rendering} angesehen
werden und ist gleichzeitig �bersichtlicher. Folglich ist der
\fachw{Schnitt} sowohl aus ergonomischer Sicht, als auch aus Sicht der
technischen Umsetzung besser geeignet.

\section{Die Nutzerschnittstelle (\fachw{GUI})}
\label{sec:GUI}

Der Entwurf der Nutzerschnittstelle ist entscheidend f�r die
praktische Anwendbarkeit und Akzeptanz eines Programms.  Dies gilt
besonders f�r Visualisierungstools. Neben der bereits besprochenen
Intuitivit�t und Effektivit�t der Visualisierung an sich, ist auch die
Intuitivit�t und Effektivit�t des Programms in das sie eingebettet
ist, von gro�er Bedeutung.

\subsection{Anforderungen}
Es sollen Schnitte aus Volumendaten dargestellt werden. Aus der Praxis
ist bekannt, dass meist mehrere Schnitte zur Anwendung kommen. Es ist
daher sinnvoll, mehrere Schnitte gleichzeitig darzustellen.  Daraus
ergibt sich jedoch das Problem, dass sich mehrere Schnittebenen
gegenseitig �berlagern k�nnten, was wieder Informationsverlust zur
Folge hat.  Au�erdem schr�nkt die gleichzeitige Anzeige mehrerer
\fachw{Schnitte} die �bersichtlichkeit stark ein. Aus diesem Grunde
ist es notwendig, dass jeder \fachw{Schnitt} seine eigene Sicht also
sein eigenes gesondertes Fenster hat. Diese Sichten sollen die
M�glichkeit bieten, andere Schnitte aus- bzw.  einzublenden, was
jedoch keinen Einfluss auf deren eigene Sicht haben darf. Eine
zentrale Verwaltung aller Schnitte sowie ein �bersichtsfenster, das
die Lage aller Schnitte im Raum und untereinander veranschaulicht,
verbessert die �bersichtlichkeit zus�tzlich.  Gleichzeitig muss das
gesamte System angemessen schnell sein und m�glichst geringe
Anforderungen an die Hardware stellen, um fl�ssiges interaktives
Arbeiten zu erm�glichen.

\subsection{Umsetzung des Schnittkonzeptes}
Die Schnitte selbst sind zweidimensional und k�nnen damit theoretisch
ohne Informationsverlust (abgesehen von Rastering) auf dem Bildschirm
abgebildet werden. Dies wird erreicht, indem die Sichtlinie, also die
Linie an der entlang der Nutzer auf den Schnitt blickt, als Normale
der Schnittebene verwendet wird.  Die Folge dieser Ausrichtung ist,
dass die Schnittfl�che parallel zur Projektionsfl�che ausgerichtet
wird.


\subsubsection{Bestimmung der Schnittebene}
\label{calc:schnitt}

Die Schnittfl�che wird durch Berechnung der Ortsvektoren ihrer vier
Ecken bestimmt. Dazu sind au�er der Sichtlinie noch zwei weitere
Parameter notwendig: der Aufspannwinkel und die Sichtsenkrechte. Der
Aufspannwinkel der Sicht ist der Winkel zwischen dem oberen und dem
unteren Rand des Sichtfeldes. Er betr�gt in der Regel ca. 30� und ist
vergleichbar mit der Brennweite einer Kamera.  Die Sichtsenkrechte
definiert Rollbewegungen der Sicht um die Sichtlinie. Sie zeigt aus
Sicht des Betrachters per Definition immer senkrecht nach oben.
W�hrend die Rollbewegungen direkt vom Nutzer bestimmt werden, leitet
sich der Aufspannwinkel von der Gr��e und Form des Sichtfensters sowie
dem Abstand der Kamera zur Schnittfl�che ab. Aus diesen drei
Parametern lassen sich die Eckvektoren der Schnittfl�che wie folgt
berechnen:

\begin{enumerate}
  
\item Der Sichtvektor \eRaum{\vec{s}} (steht senkrecht auf der
  Schnittfl�che und zeigt zur Kamera) hat die gleiche Richtung wie die
  Sichtlinie, und seine L�nge wird vom Nutzer indirekt als Entfernung
  der Kamera vom Schnitt bestimmt.
  
\item Die direkt vom Nutzer durch Rollbewegungen der Kamera bestimmte
  Sichtsenkrechte ist der Vektor \eRaum{\vec{u}}. Sie steht senkrecht
  auf dem Sichtvektor.
  
\item Der Winkel \eR{\delta} der Diagonalen des Sichtfeldes zu seiner
  Senkrechten wird nach der Formel $\delta=\arctan(b/h)$ berechnet.
  Die Breite $b$ und die H�he $h$ des Sichtfeldes sind die in das
  Koordinatensystem des Raumes umgerechneten Ma�e des Sichtfensters.

\item Die L�nge der Hypotenuse des dabei Aufgespannten Dreiecks \eR{c}
  ergibt sich aus $c=\frac{\sin(\alpha)}{\cos(\delta)}*|s|$, wobei
  \eR{\alpha} der Aufspannwinkel der Sicht ist.
  
\item Die Eckvektoren der Schnittfl�che lassen sich nun durch
  Multiplikation der normierten Sichtsenkrechte mit $c$ und der
  entsprechenden Rotation um die Sichtlinie ermitteln. Dies geschieht
  nach folgender Vorschrift.

\begin{center}
$
\vec{c}=
\begin{pmatrix}
\cos{(\degrees{360}-\delta)}\\ 
\cos{(\delta)}\\ 
\cos{(\degrees{180}-\delta)}\\ 
\cos{(\degrees{180}+\delta)}\\
\end{pmatrix}
$, $
\vec{c'}=
\begin{pmatrix}
1\\ 
1\\ 
1\\ 
1\\
\end{pmatrix}-\vec{c}
$ und $
\vec{s}=
\begin{pmatrix}
\sin{(\degrees{360}-\delta)}\\ 
\sin{(\delta)}\\ 
\sin{(\degrees{180}-\delta)}\\ 
\sin{(\degrees{180}+\delta)}\\
\end{pmatrix}
$\\ 
mit  $\vec{c} \elemof{R}{4}$, $\vec{c'} \elemof{R}{4}$ und $\vec{s} \elemof{R}{4}$ in
\[
\vec{e_n}=
\begin{pmatrix}
 s_x^2c'_n+c_n   & s_ys_xc'_n+s_zs_n & s_zs_xc'_n-s_ys_n \\ 
 s_xs_yc'_n-s_zs_n & s_y^2c'_n+c_n   & s_zs_yc'_n-s_xs_n \\ 
 s_xs_zc'_n+s_ys_n & s_ys_zc'_n-s_xs_n & s_z^2c'_n+c_n \\ 
\end{pmatrix} \cdot \vec{u}
\]
\end{center}

Dabei bestimmt \eRaum{\vec{e_n}} den Ortsvektor der n-ten Ecke der
Schnittfl�che (es gilt $0 \leq n < 4$ mit $n \elem{N}$).  Zu beachten
ist, dass aus dem Sichtfeld nur die indirekten Gr��en L�nge und Winkel
der (Fenster-)Diagonalen zur Bestimmung der Diagonalen der Schnittfl�che
verwendet werden. Die Senkrechte des Sichtfensters stimmt nicht
gezwungenerma�en mit der Sichtsenkrechten �berein, sie dient nur der
Bestimmung des Winkels $\delta$. Die Schnittfl�che wird wie beschrieben
ausschlie�lich durch die Rotation der Sichtsenkrechten um den
Sichtvektor geformt. Die Werte $c$ und $\delta$ sind dabei lediglich
Parameter, die daf�r sorgen, dass Sichtfenster und Schnittfl�che die
gleiche Form haben.

\end{enumerate}

\abb{Sampleview1} verdeutlicht die Wirkung dieses Vorgehens. Das linke
Bild zeigt den \fachw{Schnitt}, wie er in der Schnittsicht dargestellt
wird.  Das rechte Bild zeigt den �berblick, der den \fachw{Schnitt}
selbst, und seine Lage im (Daten)Raum visualisiert.

\inclfigures{Overview1_2}{Overview1_1}{5cm}{Einzelner \fachw{Schnitt} mit �bersicht}{Sampleview1}

Der Betrachter blickt, wie im linken Bild zu erkennen, immer senkrecht
auf den \fachw{Schnitt} beziehungsweise der Bildschirm liegt zu jedem
Zeitpunkt parallel zur Projektionsfl�che.  Jegliche perspektivische
Verzerrungen werden auf diese Weise vermieden. Die physische Analogie
dazu ist eine Kamera, vor die ein Schirm montiert ist, der genau das
Sichtfeld der Kamera abdeckt. Wird die Kamera bewegt, bewegt sich der
Schirm mit. Die relative Lage der Kamera zum Schirm �ndert sich
daher nicht. Die Schnittebene �bernimmt die Rolle des Schirms.  Wenn
der Betrachter seine Sicht mittels Maus bewegt, bewegt er die
Schnittebene, also den \fachw{Schnitt} durch den Raum der
Volumendaten.

Die Schnittsicht erscheint immer zweidimensional, lediglich die
�bersicht verdeutlicht die Lage des \fachw{Schnittes} im Datenraum. Es
findet eine strenge Trennung zwischen den quasi zweidimensionalen
prim�ren Informationen des \fachw{Schnittes} und seinen dreidimensionalen
sekund�ren Eigenschaften, wie zum Beispiel seiner Lage, statt. Dies
erh�ht die �bersichtlichkeit, da der Anwender beim Arbeiten in der
Schnittsicht nicht durch ``unwesentliche'' Informationen abgelenkt
wird. Um ggf. die Lage des \fachw{Schnittes} oder andere sekund�re
Informationen in Erfahrung zu bringen, reicht ein Blick auf das
�bersichtsfenster.


\subsubsection{Abbildung der Volumendaten auf dem \fachw{Schnitt}}

Welches Datum aus dem Datenraum an welcher Position auf dem
\fachw{Schnitt} gezeichnet wird, h�ngt von dessen Lage ab und wird
durch die schon erw�hnte Formel $\vec{r} = \vec{r_0} + \lambda \vec{u}
+ \mu \vec{v}$ bestimmt. Der \fachw{Schnitt} wird in Wirklichkeit
nicht durch eine unendliche Schnittebene, sondern durch ein
rechteckiges Schnittpolygon realisiert.  Die Skalierungsfaktoren
$\lambda$ und $\mu$ der obigen Ebenengleichung sind also nicht
beliebig. Ihre Wertebereiche werden durch die Abmessungen des Polygons
bestimmt.  Dieses Abmessungen wiederum werden so angepasst, dass das
Polygon die Schnittsicht genau ausf�llt.

Bewegt sich der Betrachter im GL\hyp{}Raum, so �ndert er nicht seine
Sicht auf die Schnittebene, denn die bewegt sich immer entsprechend
mit.  Stattdessen �ndert er die Lage des Schnittpolygons im Datenraum,
und damit die Parameter der obigen Formel.  Auf diesem Wege bestimmt
er welche Daten aus dem Datenraum, d.h. dem Volumendatensatz,
dargestellt werden. Die Berechnung ist zwar schon um einiges einfacher
als beim \fachw{volume rendering} aber immernoch recht umfangreich.
Sie l�sst sich jedoch auf den \fachw{Renderer} �bertragen, der sie
meist effektiver bew�ltigen kann.

\fachw{OpenGL} zeichnet Fl�chen polygonorientiert. Nur die Punkte des
GL\hyp{}Raumes, die zum zu zeichnenden Polygon geh�ren, f�r die also
obige Gleichung zutrifft, werden �berhaupt beim Zeichnen
ber�cksichtigt.  Wird einer dieser Punkte gezeichnet, kann die
verwendete Farbe wie schon erkl�rt unter Anderem aus einem
n-dimensionalen Puffer bestimmt werden. Es bietet sich also an, die
dreidimensionalen Messdaten direkt in einem solchen Puffer abzulegen.
Aus den Texturkoordinaten des zu zeichnenden Punktes \eRaum{\vec{r}}
bestimmt der \fachw{Renderer} mittels einer einfachen Transformation
$\menge{R} \mapsto \menge{N}$ den Index des Farbwerteintrages im
Texturpuffer. Die Texturkoordinaten eines Punktes werden Analog zu
seinen Raumkoordinaten �ber die Formel $\vec{r_tex} = \vec{r_{0_tex}}
+ \lambda \vec{u} + \mu \vec{v}$ bestimmt. Das Schnittpolygon
existiert daher nicht nur im GL\hyp{}Raum, sondern gleichzeitig auch im
Texturraum des \fachw{Renderers} und damit im Datenraum der in den
Texturspeicher geladenen Messdaten. Um die Lage des Schnittpolygons im
Texturraum zu bestimmen, werden seinen Eckpunkten zus�tzlich zu den
Raumkoordinaten Texturkoordinaten zugewiesen.  Je nach Lage des
Schnittpolygon im Texturraum werden die Daten bzw. Farbwerte aus dem
Daten- bzw.  Texturraum auf dem Schnittpolygon abgebildet, die es
gerade darin ``schneidet''.

Durch Gleichsetzen des Koordinatensystems des GL\hyp{}Raumes mit dem
des Texturraumes lassen sich Texturkoordinaten ohne Transformation im
GL\hyp{}Raum abbilden und anders herum. Es existiert daher ein virtueller
Datenraum innerhalb des GL\hyp{}Raumes. Dieser Datenraum wird in der
�bersicht von \abb{Sampleview1} durch den schwarzen Quader angedeutet,
wirklich sichtbar k�nnen aber nur die Volumendaten werden, die auf
einer Schnittebene liegen.

S�mtliche angef�hrten Rechnungen werden vom \fachw{Renderer}, also im
Idealfall hardwarebeschleunigt, ausgef�hrt. Der \fachw{Renderer}
bestimmt die zu zeichnenden Punkte. Er ermittelt den Index der
Farbwerte f�r das Zeichnen und greift direkt auf diese Werte (die
Messwerte) im Texturspeicher zu. Nur die Raum- und Texturkoordinaten
der Eckpunkte des Schnittes m�ssen von der Anwendung bestimmt werden.
Da GL- und Texturraum gleichgesetzt sind, lassen sich die
Raumkoordinaten auch als Texturkoordinaten werwenden. Die vier
GL\hyp{}Raum\hyp{}Koordinaten des Schnittpolygons k�nnen wie in
Abschnitt \vref{calc:schnitt} beschrieben, ermittelt werden.  Durch
die Reduzierung der n�tigen anwendungsseitigen Operationen auf die
Bestimmung dieser Eckpunkte l�sst sich die Darstellung des
\fachw{Schnittes} sehr schnell und ressourcenschonend ausf�hren.


\subsubsection{Mehrere Schnitte}

\inclfigures{Overview2_2}{Overview2_3}{4cm}{Schnitte A und B}{Sampleview2:Schnitte}
\inclfigure{hbt}{Overview2_1}{10cm}{�bersicht zu den zwei Schnitten A und B}{Sampleview2:Overview}

\abb{Sampleview1} stellt mit zwei Fenstern nur die
einfachste Variante der Visualisierung dar. Wie in den Anforderungen
beschrieben, wird das System beliebig viele Schnitte gleichzeitig in
mehreren Sichten darstellen k�nnen. \abb{Sampleview2:Schnitte} zeigt zum Beispiel die Verwendung zweier
Schnitte. Jedem Schnitt wird dabei eine Schnittsicht zugeordnet.

Die Schnitte (\abb{Sampleview2:Schnitte}) k�nnen unabh�ngig voneinander
positioniert werden. Jede Schnittsicht zeigt nur ihre eigene
Schnittebene, andere Schnitte sind ausgeblendet. Die dazugeh�rige
�bersicht (\abb{Sampleview2:Overview}) zeigt dagegen beide Schnitte, ihre Lage
zueinander, sowie ihre Lage innerhalb des Datenraumes.

Aus der Verwendung mehrerer Fenster, und damit mehrerer Instanzen des
Renderers ergibt sich zus�tzlicher Verwaltungsaufwand. Die �nderung
der Lage einer Schnittfl�che muss allen Instanzen des Renderers, die
diese Schnittebene darstellen, mitgeteilt werden. Dies betrifft
mindestens die ihr zugeordnete Sicht und das �bersichtsfenster.

Zudem m�ssten sich alle Instanzen die Volumendaten teilen, damit diese
gro�e Datenmenge nicht unn�tig vervielfacht werden muss. Auch die
Schnittfl�che selbst darf bei der Anzeige in einem zus�tzlichen
Fenster m�glichst nicht kopiert werden. Ihre Parameter werden bei der
Anzeige in einer zus�tzlichen Instanz nicht ver�ndert. Eine Kopie, die
als weitere Schnittfl�che an den Renderer �bermittelt werden m�sste,
wird daher eigentlich nicht ben�tigt. \fachw{OpenGL} bietet dazu die
M�glichkeit, dass sich mehrere Instanzen des Renderers interne Daten
wie Texturen und zu zeichnende Objekte untereinander teilen. Ihre
anwendungsseitigen Container fallen aber wie alle anderen gemeinsam
verwendeten Daten in den Verantwortungsbereich der Anwendung. In der
Implementation m�ssen daher Mittel zum Teilen von Ressourcen und zur
Kommunikation integriert werden.

\subsection{Der Cursor}

Als Cursor wird in der IT allgemein die Bearbeitungsposition innerhalb
einer Menge von Daten bezeichnet. Dies gilt auch f�r die hier
verwendeten Volumendaten. Der Cursor liegt naturgem�� meist im Zentrum
der Betrachtung, und kann so effektiv besonders ver�nderliche
Informationen �bermitteln. Dabei sollte er aber auch nicht �berladen
werden.

\subsubsection{Positionsbestimmung}
\label{cursor:pos}

Die wichtigste Information, die ein Cursor darstellt ist die der
Position. Visualisierungssysteme verwenden meist die Position des
Mauszeigers im Darstellungsfenster, um daraus die Position des Cursors
zu ermitteln. Die �bertragung von Bildschirmkoordinaten ist jedoch bei
dreidimensionalen Systemen nicht so trivial wie bei zweidimensionalen
Systemen. Das Problem liegt dabei darin begr�ndet, dass die bei der
Projektion des Raumes auf den Bildschirm ($\mathbb{R}^3 \mapsto
\mathbb{N}^2$) verloren gegangene Tiefeninformation bei der
umgekehrten Projektion der Bildschirmkoordinaten in den Raum
($\mathbb{N}^2 \mapsto \mathbb{R}^3$) fehlt. Aus den
Bildschirmkoordinaten l�sst sich die Raumkoordinate daher nicht
eindeutig bestimmen, vielmehr liefert $\mathbb{N}^2 \mapsto
\mathbb{R}^3$ einen senkrecht auf der Projektionsfl�che stehenden
Strahl von m�glichen Koordinaten. \fachw{OpenGL} bietet zwar eine
entsprechende Funktion zur umgekehrten Projektion an, dieser muss aber
eine Tiefeninformation in der Form $z=\frac{a}{b}$ �bergeben werden.
Dabei sind $a$ der Abstand des gesuchten Punktes von der
Projektionsebene, und $b$ der Abstand des Horizonts von der
Projektionsebene. Die Ma�zahl $z \elem{R}$ hat dabei nichts mit der
z-Achse des Raumes zu tun, sie gibt lediglich ein L�ngenverh�ltnis an,
und bestimmt so einen Punkt auf dem erw�hnten Strahl. F�r $z$ sind
also nur Werte zwischen $0$ und $1$ sinnvoll.  Ein $z>1$ w�rde
bedeuten, dass der gesuchte Punkt hinter dem Horizont liegt, und ein
$z<0$, dass er vor der Projektionsfl�che liegt.  In beiden F�llen ist
er nicht sichtbar und damit nutzlos.  Die Bestimmung dieses
Verh�ltnisses bleibt Aufgabe der Anwendung, und stellt damit das
eigentliche Problem bei der �bertragung von Bildschirmkoordinaten in
den Raum dar.

Durch die Verwendung des Schnittkonzeptes ist die Bestimmung der
Verh�ltnisszahl $z$ relativ einfach. Der Nutzer bewegt sich zwar im
Grunde frei im Raum, jedoch bleibt er immer auf der Schnittebene. Da
nur die auf der Schnittebene liegende Volumendaten sichtbar werden,
sind nur die auf der Schnittebene liegende Koordinaten f�r den
Anwender von Interesse. Der gesuchte Wert $a$ ($b$ ist vorgegeben)
entspricht also genau dem Abstand der Projektionsebene am ausgew�hlten
Punkt zur Schnittebene.  Auch $a$ ist weitestgehend konstant, da die
Schnittebene, wie bereits dargelegt, parallel zur Projektionsebene
liegt, und nur selten ihren Abstand zu dieser �ndert.  Auf diese Weise
kann auch bei dieser Operation die eigentliche Rechenlast (die
Projektion $\mathbb{N}^2 \mapsto \mathbb{R}^3$) auf den Renderer
�bertragen werden, w�hrend die dazu n�tige Verh�ltniszahl $z$
weitestgehend von Konstanten abh�ngt, und deshalb nur selten bestimmt
werden muss.


\subsubsection{Darstellung}

\inclfigure{hbt}{Cursor}{9cm}{Beispiel eines Cursors}{Samplecursor}

Cursor werden in der Regel als Trennmarkierung zwischen einzelnen
Dateneinheiten, oder als Markierung einer einzelnen Dateneinheit
dargestellt. Bei Visualisierungssystemen ist der Cursor oft ein
schlichtes Fadenkreuz.  Das hier entworfene System verwendet jedoch
eine r�umliche Darstellung, bei der weder die Betrachtungsrichtung,
noch die Skalierung offensichtlich sind.  Es liegt daher nahe, dem
Cursor noch zus�tzlich zur Position Informationen �ber Lage und
Skalierung mitzugeben. Dazu wird der Cursor als W�rfel entworfen,
dessen Kantenl�nge einen bekannten Wert, z.B. eine Einheit im
Datenraum, hat. Auf diese Weise fungiert er gleichzeitig als Ma�stab,
als Markierung des aktuell ausgew�hlten Datenelementes und auch als
Abgrenzung zu den Nachbarn dieses Elementes. Wird der W�rfel fest an
den Koordinatenachsen des Raumes ausgerichtet, kann er au�erdem die
eventuelle ``Schieflage'' des betrachteten Schnittes durch seine
eigene ``Schieflage'' (im Verh�ltnis zum Schnitt) andeuten.
\abb{Samplecursor} zeigt einen solchen Cursor in der Mitte der Anzeige
eines schief im Raum liegenden Schnittes.

Die Position des Cursors im Raum, und damit auch im Datenraum) wird in
diesem Beispiel in der Statuszeile angezeigt. Der Cursor selbst liegt
schief und verdeutlicht so die ``Schieflage'' des Schnittes. Der
Schnitt selbst ist stark vergr��ert dargestellt, was auch durch die
relative Gr��e des mit einem Millimeter Kantenl�nge eigentlich sehr
kleinen Cursors erkennbar ist.

\section{Segmentierung der Volumendaten}

Wie in der Zielsetzung beschrieben, soll das Visualisierungssystem die
M�glichkeit zur interaktiven Objekterkennung bieten. Dabei sollen
sich Anwender und Rechner gegenseitig unterst�tzen. Indem dem Rechner
h�here Erkennungsaufgaben abgenommen werden, l�sst sich die notwendige
Rechenzeit reduzieren.  Au�erdem wird das Erkennungssystem auf diese
Weise flexibler. Es l�sst sich somit auf eine breitere Auswahl von
Eingabedaten anwenden und reagiert toleranter auf unerwartete Werte.
Die, der Interaktiven Segmentierung vorangestellte automatische
Aufbereitung besteht in einer Vorsegmentierung der Daten. Dabei sucht
der Rechner kleinere Strukturen, die auch automatisch sicher zu
erkennen sind, und markiert sie. Bei der, darauf folgenden
Interaktiven Segmentierung sollen diese vorher erkannten Unterobjekte
vom Nutzer sinnvoll zusammengesetzt werden.

\subsection{Die automatische Vorsegmentierung}

F�r die Vorsegmentierung der Daten wird die von
\citet{digabel-lantuejoul78} entwickelte
\fachw{watershed}\hyp{}Transformation verwendet werden. Sie
interpretiert zweidimensionale Bilder als zweidimensionale Graphen,
und trennt in den daraus gebildeten Wertgebirgen die ``Senken'' und
``Gipfel''. Soll die \fachw{watershed}\hyp{}Transformation auf
dreidimensionale Bilder, also Volumendaten, angewendet werden, m�ssten
lediglich die verwendeten zweidimensionalen Graphen durch eine dritte
Dimension erweitert werden. Obwohl sie urspr�nglich f�r
zweidimensionale Bilddaten entworfen wurde, l�sst sich
\fachw{watershed}\hyp{}Transformation auf diese Weise leicht auf
dreidimensionale Bilder �bertragen. Es existiert ein gro�e Anzahl von
Algorithmen, die eine \fachw{watershed}\hyp{}Transformation umsetzen.
F�r diese Arbeit wurde ein rekursiver Wurzelsuchalgorithmus nach
\citet{oai:CiteSeerPSU:114309} ausgew�hlt. Dieser Algorithmus bietet
eine Kostenfunktion, die ben�tigt wird, wenn der Graph eine erh�hte
Konnektivit�t erhalten soll. Diese Problematik wird in Kapitel
\vref{cha:watershed_impl} ausf�hrlich behandelt werden. Die
Wurzelsuche l�sst sich au�erdem leicht parallelisieren, und kann somit
von zuk�nftigen \fachw{dual-core}\hyp{}PC\hyp{}Systemen profitieren.
Das Ergebnis dieser Transformation sind dreidimensionale Bilder, in
denen alle Punkte, die zu einem Unterobjekt geh�ren, den gleichen
Wert (Index) haben. Diese Strukturen m�ssen im Rahmen der folgenden
interaktiven Segmentierung vom Nutzer zusammengef�gt werden.

\subsection{Visualisierung der interaktiven Segmentierung}

Bei der Interaktiven Segmentierung spielt das Visualisierungssystem
eine wichtige Rolle, denn der Nutzer muss interaktiv mitverfolgen
k�nnen, welches Unterobjekt er gerade ausgew�hlt hat, wo dieses Objekt
liegt, und wie es aussieht. Daher muss das ausgew�hlte Unterobjekt
dreidimensional dargestellt werden. Der Schnitt als Prim�re
``Arbeitsfl�che'' des Nutzers und Dom�ne des Cursors ist jedoch
prinzipiell zweidimensional.  Das gew�hlte Unterobjekt lie�e sich zwar
trotzdem dreidimensional darin zeichnen, jedoch w�rde es einen Teil
des Schnittes verdecken. Ein Unterobjekt wird aktiv, wenn der Cursor
sich in ihm befindet. Folglich w�hre dieser zu jeden Zeitpunkt von der
Oberfl�che des aktiven Unterobjektes verdeckt. Aus diesen Gr�nden wird
die Darstellung des aktiven Unterobjektes in die �bersicht
ausgelagert. In der Schnittansicht wird das Unterobjekt dagegen
lediglich angedeutet. Die vom Nutzer zusammengesetzten Objekte werden
ausschlie�lich in der �bersicht dargestellt.

Diese Visualisierung der Segmentierung muss ausreichend schnell
erfolgen, um fl�ssiges Arbeiten zu erm�glichen.  Aufgrund der
Vorbereitung der Daten durch die Vorsegmentierung muss der Rechner
w�hrend der eigentlichen Segmentierung nur noch das vom Nutzer
ausgew�hlte Unterobjekt aus dem Ergebnisspool der Vorsegmentierung
ausw�hlen und darstellen. Werden die Darstellungen der Unterobjekte
gepuffert (die Unterobjekte �ndern sich nie), l�sst sich die
Visualisierung der Segmentierung l�sst sich daher problemlos mit der
geforderten Geschwindigkeit umsetzen. Dessenungeachtet ist auch hier
ein effektives System f�r den Datenaustausch zwischen den Sichten
notwendig ist.



\chapter{Implementation der Basisbibliothek}

\section{Vorbemerkungen zur Basisbibliothek}
\label{sec:why_sgl}
Ziel dieser Diplomarbeit ist die Entwicklung eines m�glichst
praxisnahen Visualisierungstools. Da sich in der Praxis die
Anforderungen oft �ndern, sollte dieses Programm leicht zu pflegen und
zu erweitern sein.  

Visualisierungssysteme sind naturgem�� �u�erst komplex und
umfangreich.  Das erschwert deren Pflege und jegliche nachtr�gliche
Anpassungen wird zu einer potenziellen Fehlerquelle.  Jedoch f�llt
beim Betrachten der Anforderungen an solche Systeme auf, dass viele
grundlegende Probleme und Aufgaben regelm��ig in gleicher, oder leicht
abgewandelter Form wiederkehren.  So ben�tigen z.B. alle
3D\hyp{}Visualisierungssysteme eine Kamera, Texturen und grundlegende
geometrische Objekte wie etwa Fl�chen.  All diese Programme m�ssen
au�erdem sowohl mit dem Nutzer, als auch mit ihrer Umgebung
interagieren.  Es ist daher sinnvoll, zuerst ein m�glichst allgemeines
Zeichensystem zu entwerfen, das diese Grundvoraussetzungen erf�llt.
Dieses System wird in einer Basisbibliothek implementiert, und steht
auf diese Weise darauf aufbauenden Systemen zur Verf�gung.  Ausgehend
von dieser Grundlage lassen sich dann leicht L�sungen f�r einen
konkreten Fall ableiten. Die Basisbibliothek wird objektorientiert
implementiert.  Sp�tere Spezialisierung sind daher problemlos �ber das
Mittel der Vererbung zu erreichen. Da die Basisbibliothek die
Grundlage dieser und zuk�nftiger Visualisierungsarbeiten darstellen
und dabei m�glichst breit einsetzbat sein soll, muss ihr Entwurf
besonders gr�ndlich und umfassend erfolgen.

\subsection{Gemeinsam genutzte Ressourcen}
\label{sec:ressharing}

Im Entwurf hat sich der Bedarf nach einer effektiven Ressourcenteilung
unter den verschiedenen Objekten gezeigt. Einerseits, um Speicherplatz
zu sparen, und andererseits, um �nderungen an einem mehrfach verwendeten
Objekt nicht unn�tig bei dessen Kopien reproduzieren zu m�ssen.
Die einfachste Methode dies zu erreichen, ist die Verwendung ein und
des selben Speicherbereiches f�r mehrere Objekte.

\subsubsection{Das Zeigerproblem}

In den Sprachen C und C++ werden zum gleichzeitigen Verwenden des
selben Speicherbereiches �blicherweise \fachw{Zeiger} verwendet.
\fachw{Zeiger} speichern statt der eigentlichen Daten nur die Adresse,
an der sich die Daten im Speicher befinden. Auf diesem Wege kann ohne
weiteren Aufwand von jedem beliebigen Teil eines Programms aus auf die
selben Daten zugegriffen werden. Dieses Konzept ist zwar sehr
effektiv, da es sich an der Hardware orientiert, f�hrt aber unter
Umst�nden zu gro�en Problemen. Wird der G�ltigkeitsbereich einer
Variablen verlassen, wird der f�r sie reservierte Speicherbereich
wieder freigegeben. Die Variable gilt damit als gel�scht. Im
Normalfall stellt dies kein Problem dar, da die Variable per
Definition die Einzige war, die auf diese Daten zugriff.  Da die
Variable nicht mehr existiert, werden die gel�schten Daten garantiert
nicht mehr ben�tigt. Verl�sst ein Zeiger seinen G�ltigkeitsbereich,
w�re es theoretisch kein Problem den Speicherbereich auf den er
``zeigt'' freizugeben. Zu diesem Zeitpunkt ist aber nicht bekannt, ob
dieser Speicherbereich nicht noch von einem anderen Zeiger, oder einer
regul�ren Variablen verwendet wird. Der von einem Zeiger verwendete
Speicher kann deshalb nur von Hand freigegeben werden, wenn
sichergestellt wurde, dass er nicht mehr verwendet wird. Es ist
offensichtlich, dass das sehr schwierig werden kann. Gerade in
komplexen Systemen geht die �bersicht �ber die Objekte im Speicher
schnell verloren. Oft ist es f�r den Entwickler schlicht unm�glich zu
garantieren, dass ein bestimmter von Zeigern verwendeter
Speicherbereich gel�scht werden kann. Dabei kann das Freigeben noch
verwendeten Speichers zu schweren Fehlern f�hren.

\subsubsection{Intelligente Zeiger}

Der Vorteil regul�rer Variablen ist die automatische Freigabe des
belegten Speichers bei Verlassen des G�ltigkeitsbereiches. Der Vorteil
von Zeigern liegt dagegen in der M�glichkeit einen Speicherbereich
mehrfach zu verwenden. Diese scheinbar gegens�tzlichen Eigenschaften
vereinen intelligente Zeiger. 

Intelligente Zeiger sind Objekte, die als regul�re Variable erzeugt
werden. Wie andere Variablen werden sie bei Verlassen ihres Kontextes
automatisch gel�scht. Den eigentlichen Zeiger halten sie als
Parameter. Wird ein solcher intelligenter Zeiger kopiert, wird nicht
nur der eigentliche Zeiger, also die Speicheradresse, kopiert.  Es
wird au�erdem ein interner Referenzz�hler f�r diese Speicheradresse
erh�ht. Verl�sst einer der Beiden seinen G�ltigkeitsbereich, so kann
sein Destruktor �ber diesen Referenzz�hler leicht ermitteln, ob der
von diesem intelligenten Zeiger verwendete Speicherbereich noch von
Anderen intelligenten Zeigern verwendet wird. Ist dies der Fall, wird
lediglich der Referenzz�hler um eins decrementiert. Damit dieses
System funktioniert, m�ssen sich die intelligenten Zeiger den
Referenzz�hler teilen. Ist der zu l�schende intelligente Zeiger
dagegen der Einzige, der diesen Speicherbereich verwendet, gibt er ihn
frei. Zudem existieren noch einige syntaktische Mittel, welche die
Arbeit mit intelligenten Zeigern einfacher und sicherer gestalten.
Die f�r normale Zeiger in C und C++ �blichen Operatoren zur
Dereferenzierung (\code{*} und \code{->}) k�nnen durch entsprechende
eigene Funktionen der Klasse \fachw{intelligenter Zeiger} �berladen
werden, um die ``Dereferenzierung'' der intelligenten Zeiger zu
vereinfachen.  Desweiteren besteht das Interface dieser Klasse aus
zwei Konstruktoren. Der Erste erwartet einen konventionellen Zeiger
und ihm sollte ausschlie�lich der R�ckgabewert einer Objekterzeugung
(\code{new}\hyp{}Operation) �bergeben werden. Der zweite Konstruktor
ist der Kopierkonstruktor. Auch er muss �berladen werden, da beim
Kopieren der Referenzz�hler erh�ht werden muss.  Nat�rlich darf nicht
von au�en auf interne Parameter des intelligenten Zeigers zugegriffen
werden k�nnen. Die einzige Stelle des Interfaces, an der noch
konventionelle Zeiger auftreten, ist beim Erzeugen eines neuen
referenzierten Objektes, also dem ersten Konstruktor. Die
\code{new}\hyp{}Operation, und damit das Erzeugen eines neuen
Referenzierten Objektes im Speicher lie�e sich auch direkt in den
Konstruktor des intelligenten Zeigers verlagern.  Dies wird jedoch
dadurch erschwert, dass die Parameter, die der Konstruktor des
referenzierten Objektes erwartet, unbekannt sind.  Aus diesem Grund
wird auf diese M�glichkeit verzichtet. Es reicht an dieser Stelle auch
aus, einfach ``aufzupassen''. Die genannten Anforderungen, und das
hier entworfene Verhalten basieren auf den Arbeiten von \citet{E&D-94}
sowie \citet{Col-94}. Sie werden von dem Datentyp \code{shared\_ptr}
umgesetzt. Diese generative Objektklasse aus den
Boost\hyp{}Bibliotheken \citep{boost:smartptr} wird helfen im Verlauf
dieser Arbeit ein robustes und leicht handhabbares System zur
gemeinsamen Benutzung von Speicherbereichen zu implementieren.


\subsection{Kommunikation}
\label{sec:signals}

Eine weitere, beim Entwurf deutlich gewordene Anforderung ist die
M�glichkeit zur Kommunikation zwischen einzelnen Teilen des Programms.
Darunter wird allgemein die M�glichkeit eines Programmteiles
verstanden, unter bestimmten Umst�nden die Ausf�hrung von Code in
einem ``entfernten'' anderen Teil des Programms auszul�sen. Also den
Prozessor zu veranlassen, an die Entsprechende Stelle im Programm und
nach Ausf�hrung des Codes wieder zur�ck zu springen. Auch wenn die
verwendete Programmiersprache C++ kein explizit daf�r ausgelegtes
Mittel kennt, gibt es unter Anderem die folgenden M�glichkeiten das
beschriebene Verhalten mit allgemeineren Mitteln zu modellieren.

\subsubsection{\fachw{Polling} gemeinsamer Variablen}

Die einfachste Methode zur Kommunikation zwischen Programmteilen
besteht darin, die zu �bermittelnde Botschaft in einem gemeinsam
verwendeten Speicherbereich abzulegen. Hat der Empf�nger die Nachricht
gelesen, l�scht er sie in der Regel, um zu verhindern, dass er sie
versehentlich doppelt liest. Das Problem dabei ist, dass eine
Nachricht dadurch nur einmal gelesen werden kann. die Versendung an
mehrere Empf�nger ist somit ausgeschlossen.  Weiterhin k�nnen die
potentiellen Empf�nger nicht wissen, ob eine Nachricht f�r sie
vorliegt. Sie m�ssen dies in regelm��igen Zeitintervallen selbst
pr�fen. Dies ist meist so implementiert, dass alle potentiellen
Empf�nger von einem zentralen System regelm��ig dazu veranlasst
werden, die Pr�fung und ggf. den der Nachricht entsprechenden Code
auszuf�hren. Diese als \fachw{Polling} bezeichneten Operationen kosten
zus�tzlich Prozessorzeit und verursachen auch zus�tzlichen
Verwaltungsaufwand, da eine Zentrale Verwaltungsstelle f�r die
Kommunikation n�tig wird.


\subsubsection{Triggering mittels \fachw{Callbacks}}

Die Programmiersprache C erm�glicht neben Zeigern auf Daten auch
Zeiger auf Funktionen. Durch dieses System ist es beispielsweise
m�glich einer Funktion den Zeiger auf eine Nachrichtenfunktion
mitzugeben, die diese dann nach Belieben aufrufen kann. Auf diese
Weise kann sie bei ihrem Aufrufer ``zur�ckrufen''. Die dabei
verwendeten Funktion werden deshalb auch als
\fachw{Callback}\hyp{}Funktionen bzw. als \fachw{Callbacks}
bezeichnet.  �bergibt ein potentieller Empf�nger einem Sender eine vom
Empf�nger bestimmte Nachrichtenfunktion, kann der Sender sie nach
Belieben aufrufen. Der Sender f�hrt daher vom Empf�nger bestimmten
Code aus. Im Gegensatz zum \fachw{Polling} von Nachrichten bestimmt
hier der Sender, wann der Code ausgef�hrt wird.  Da der Sender
``wei�'', wann eine Nachricht vorliegt, wird nur dann Code ausgef�hrt,
wenn es auch wirklich n�tig ist. Unter C++ werden \fachw{Callbacks}
oft als Objektklassen mit �berladenem \code{()}-Operator realisiert.
Dadurch verhalten sie sich syntaktisch wie Funktionen, obwohl sie im
Sinne der Sprache Objekte sind. Aus diesem Grunde k�nnen
\fachw{Callbacks}\hyp{}Objekte Parameter wie zum Beispiel einen Zeiger
auf den Empf�nger halten.  Obwohl \fachw{Callbacks} sehr effektiv
sind, werden sie bei gr��eren Systemen selten eingesetzt, denn Zeiger
auf Funktionen leiden unter �hnlichen Problemen wie Zeiger auf Daten.
Die Callbackfunktionen selbst werden zwar nur �u�erst selten gel�scht
(Programmcode wird nur unter besonderen Bedingungen freigegeben), aber
es kann nicht garantiert werden, dass der Empf�nger bzw. seine Daten
noch existieren. Die Aufrufe des \fachw{Callbacks} selbst k�nnten
demnach ins ``Leere'' laufen und u. U. sogar Schutzverletzungen
ausl�sen.  Das gleiche gilt f�r \fachw{Callback}\hyp{}Objekte wenn
diese ihre Empf�nger nicht �berwachen.  Zudem muss der Sender die
\fachw{Callback}\hyp{}Objekte verwalten. Er muss also �ber s�mtliche
\fachw{Callback}\hyp{}Objekt\hyp{}Klassen informiert sein und ihre
Instanzen sinnvoll verwalten k�nnen.

\subsubsection{Triggering mittels \fachw{Signal}\hyp{}\fachw{Slot}\hyp{}Architektur}

Die \fachw{Signal}\hyp{}\fachw{Slot}\hyp{}Architektur stellt eine formalisierte
Erweiterung von \fachw{Callback}\hyp{}Objekten dar. Der \fachw{Slot} ist
dabei ein \fachw{Callback}\hyp{}Objekt mit einigen zus�tzlich vereinbarten
Eigenschaften. Er wird �blicherweise als Parameter\hyp{}Objekt des
Empf�ngers implementiert, so dass er automatisch gel�scht wird, wenn
der Empf�nger gel�scht wird.  Durch einen �berladenen Destruktor kann
der \fachw{Slot} sich so bei seinem \fachw{Signal} abmelden. Das
verhindert, dass Empf�nger angesprochen werden, die nicht mehr
existieren. Aufgebaut wird eine
\fachw{Signal}\hyp{}\fachw{Slot}\hyp{}Verbindung indem der Empf�nger seinen
entsprechenden \fachw{Slot} bei dem \fachw{Signal}\hyp{}Objekt des Senders
anmeldet. \fachw{Signal}\hyp{}Objekte werden �hnlich wie
\fachw{Slot}\hyp{}Objekte in der Regel als Parameter des Senders
implementiert, um auch hier sicherzustellen, dass der Sender sich
gegebenenfalls bei den, mit ihm verbundenen, \fachw{Slots} abmeldet.
Ausgel�st wird das \fachw{Signal} �ber den Aufruf seines �berladenen
\code{()}-Operators. Dieser ruft wiederum die \code{()}-Operatoren aller
registrierten \fachw{Slots} auf. Diese l�sen dann die gew�nschten Aktionen im
Empf�nger aus. Die Vorteile diese Konzeptes liegen darin, dass
Empf�nger und Sender nichts voneinander wissen m�ssen, die Verbindung
aber dennoch stabil ist. Wie bei einfachen \fachw{Callbacks} kann die
Kommunikationsverbindung zur Laufzeit beliebig hergestellt und wieder
getrennt werden. Dabei fungieren die \fachw{Signal}- bzw.
\fachw{Slot}\hyp{}Objekt als standardisierte Vermittler, so dass beliebige
Objekte ``verbunden'' werden k�nnen. Auch die Verbindung mehrerer
Empf�nger bzw. deren \fachw{Slots} mit einem \fachw{Signal} stellt kein Problem
dar, da das \fachw{Signal}\hyp{}Objekt die ihm zugeordneten \fachw{Slots}
in einer dynamischen Liste halten kann.

\inclfigure{hbt}{classSGLObjBase_1_1CompilerMerker__coll__graph}{13cm}{Der \fachw{Slot} \code{SGLObjBase::CompilerMerker} und seine Beziehungen}{SGLObjBase::CompilerMerker:Bez}

Neben der Quasi\hyp{}Standardimplementation der
Boost\hyp{}Bibliotheken \citep{boost:signal} geh�rt die Umsetzung in
der \fachw{Qt\hyp{}Bibliothek} zu den verbreitetsten
\fachw{Signal}\hyp{}\fachw{Slot}\hyp{}Systemen f�r C++. \fachw{Qt}
verwendet allerdings ein weniger formalisiertes Konzept, das auf einem
\fachw{Precompiler}, Makros und einer zentralen Verwaltung von Sendern
und Empf�ngern basiert. Au�er f�r die Implementation der
Nutzerschnittstelle wird in dieser Arbeit ausschlie�lich die
Boost\hyp{}Implementation Anwendung finden.
\abb{SGLObjBase::CompilerMerker:Bez} zeigt zum Beispiel die
\fachw{Slot}\hyp{}Klasse \code{CompilerMerker}, deren einzige Instanz
das zu \fachw{SGLObjBase} geh�rende Funktionsobjekt
\code{compileNextTime} ist.

\section{Die wichtigsten Objektklassen}
\label{sec:object}

Die Basisbibliothek wird, wie im Entwurf festgelegt, objektorientiert
implementiert. Die dabei verwendete Hierarchie und Strukturierung
lehnt sich an die eines ``nat�rlichen'' Raumes an.

\subsection{Die Raumklasse \fachw{SGLSpace}}


\fachw{SGLSpace}, die Objektklasse zur Abstraktion des Raumes,
stellt die zentrale Schnittstelle und Verwalungsinstanz des Systems
dar. Sie erzeugt einen \fachw{OpenGL}\hyp{}Kontext, und initialisiert wenn 
n�tig den \fachw{OpenGL\hyp{}Renderer}. Sie ist ebenso f�r die
Kommunikation mit den Widgetsystem zust�ndig. Zur Anpassung an
verschiedene Widgetsysteme werden spezialisierte Raumklassen von der
allgemeinen Raumklasse abgeleitet.  Diese Ableitungen sind Teil
gesonderter Widget\hyp{}Adapter\hyp{}Bibliotheken.

\inclfigure{hbt}{classSGLSpace__inherit__graph}{5cm}{Klassendiagramm f�r SGLSpace}{SGLSpace:Class}
\inclfigure{hbt}{classSGLSpace__coll__graph}{15cm}{wichtige Member von SGLSpace}{SGLSpace:Bez}

S�mtliche zu zeichnende Objekte m�ssen sich bei der entsprechenden
Instanz der Raum\hyp{}Klasse registrieren, um in diesem ``Raum''
gezeichnet zu werden.  Die Raumklasse speichert intelligente Zeiger
auf diese Objekte in speziellen Listen. Neben den ``normalen''
Objekten verwaltet jede Rauminstanz noch einige spezielle Objekte die
es nur einmal pro Raum geben kann. Zum Beispiel eine Konsole f�r die
Ausgabe von Meldungen im Grafikfenster, Koordinatengitter mit ihren
Achsenbeschriftungen und nat�rlich eine Kamera. Auch die Modi des
Renderers werden zentral in der Raumklasse verwaltet. Jegliche interne
und externe Ereignisse werden von ihr interpretiert und die
entsprechenden Operationen, wie zum Beispiel die Umpositionierung der
Kamera, ausgef�hrt.  Bei Bedarf zeichnet sie die Szene danach komplett
neu. Dazu wird der Renderer in den entsprechenden Modus versetzt und
anschlie�end die Objektlisten aufgefordert, bei allen registrierten
Objekten die Funktion zum Zeichnen aufzurufen.

Jede Instanz der Raumklasse hat einen eigenen
\fachw{OpenGL}\hyp{}Kontext und ein eigenes Fenster, in das dieser
Kontex \fachw{rendert}. Die einzelnen Kontexte verschiedener Instanzen
der Raumklasse k�nnen sich aber trotzdem Daten im \fachw{Renderer}
teilen.  Objekte, die in einer Instanz erzeugt wurden, stehen so
automatischen allen anderen \fachw{OpenGL}\hyp{}Kontexten zum
Darstellen zur Verf�gung. Unabh�ngig davon muss ein Objekt in jeder
Instanz der Raumklasse registriert sein, in der es gezeichnet werden
soll. Ist ein Objekt in mehreren Instanzen der Raumklasse registriert,
k�nnen diese sich seine Daten teilen.  Auf diese Weise wird ein und
das selbe Objekt mit den entsprechenden Transformationen der
verschiedenen Instanzen in den verschiedemem Kontexten gezeichnet.  In
solch einem Fall kann jede Instanz der Raumklasse als eigenst�ndige
Sicht auf ein und denselben Raum betrachtet werden.  Wie bereits
erw�hnt muss das Teilen der Container der Zeichenobjekte durch die
Anwendung selbst realisiert werden, da die Kontexte sich nur
\fachw{Renderer}\hyp{}interne Daten teilen k�nnen.


\subsection{Die Basisklasse f�r Zeichenobjekte \code{SGLObj}}

\code{SGLObj} ist die Basisklasse aller Container f�r Zeichenobjekte.
Sie h�lt s�mtliche allgemeine Parameter des Objektes wie
Farbinformationen und die Transformationsmatrizen zur Positionierung
des Objektes im Raum.  Sie stellt auch die Schnittstelle zum Zeichnen
dieses Objektes dar, und verwaltet seinen Operationspuffer.

\inclfigure{hbt}{classSGLObj__inherit__graph}{15cm}{Klassendiagramm f�r SGLObj}{SGLObj:Class}

\inclfigure{hbt}{classSGLObj__coll__graph}{8cm}{SGLObj und ihre Beziehungen zu Anderen Klassen}{SGLObj:Bez}

\subsubsection{Puffern von Zeichenoperationen in \fachw{OpenGL}}
\label{displist}

\fachw{OpenGL} ist in der Lage, sich durch API\hyp{}Aufrufe ausgel�ste
interne Operationen des \fachw{Renderers} zu merken und im
\fachw{Renderer} zu puffern.  Muss das Objekt neu gezeichnet werden,
entscheidet die Zeichenschnittstelle von \code{SGLObj}, ob es n�tig
ist, das Objekt mittels API\hyp{}Aufrufen komplett neu zu generieren.
Dies wird im Folgenden als Kompilierung des Objektes bezeichnet. Hat
sich an einem Objekt seit der letzten Kompilierung Nichts ge�ndert,
w�rde sich auch an den API\hyp{}Aufrufen zum Zeichnen Nichts im
Vergleich zum vorherigen Aufruf dieser Zeichenfunktionen �ndern. In
solch einem Fall reicht es aus, nur den entsprechenden
Operationspuffer aufzurufen.  Der Effekt dieses Vorgehens entspricht
dem eines Caches f�r Zeichenoperationen und vereinfacht nebenbei das
gleichzeitige Anzeigen eines Objektes in mehreren Sichten, denn die
entsprechenden Operationspuffer liegen im \fachw{Renderer}. Sie stehen
somit allen Kontexten gleicherma�en zur Verf�gung.

Tritt ein Ereignis auf, welches das Objekt m�glicherweise ver�ndert,
muss der Operationspuffer dieses Objektes als ung�ltig markiert
werden.  Gleichzeitig werden alle Sichten, in denen es angezeigt wird,
aufgefordert neu zu zeichnen. Ruft eine solche Sicht (Instanz der
Raumklasse) �ber ihre Objektliste die Zeichenfunktion dieses Objekts
auf, kompiliert diese das Objekt neu, da sie erkennt, dass der
Operationspuffer nicht mehr aktuell ist.  Danach markiert die
Zeichenfunktion den Operationspuffer des Objektes wieder als g�ltig
und zeichnet das Objekt. Wenn m�glich, werden das Zeichnen und das
Kompilieren zusammengefasst, im Ergebnis unterscheidet es sich aber
nicht von der Hintereinanderausf�hrung.

\subsubsection{Operationspuffer beim Zeichnen in mehreren Sichten}

Wie beschrieben, teilen sich alle Sichten sowohl interne Daten des
\fachw{Renderers}, in diesem Fall den Operationspuffer des Objektes,
als auch die dazugeh�rigen anwendungsseitigen Informationen.  Sie
erkennen daher bei einem k�rzlich kompilierten Objekt, dass der
Operationspuffer g�ltig ist. Sie k�nnen das Objekt also direkt
zeichnen, ohne es erneut kompilieren zu m�ssen.  In einer
\fachw{Multithreading}\hyp{}Umgebung kann es theoretisch vorkommen,
dass mehrere Instanzen gleichzeitig die Ung�ltigkeit der
Operationspuffer feststellen. Das hat zur Folge, dass das Objekt unter
Umst�nden unn�tigerweise mehrfach kompiliert wird. Dies kostet zwar
etwas Zeit, hat aber keine weiteren Folgen, da der \fachw{Renderer}
daf�r sorgt, dass der Operationspuffer auch unter diesen Umst�nden
korrekt geschrieben wird.  Es lohnt es sich daher nicht, ein System zu
implementieren, das parallele Kompilierungen verhindert.  Ein solches
System w�rde selbst zus�tzlich Rechenzeit beanspruchen, und w�re nicht
zuletzt eine weitere Fehlerquelle.

Die Funktion, die den Operationspuffer eines Objektes als ung�ltig
markiert, ist als \fachw{Slot} implementiert
(\code{CompilerMerker compileNextTime} zu sehen auf
\abb{SGLObj:Class} ). Dadurch ist es m�glich, den Operationspuffer
eines Objektes mittels eines \fachw{Signals}, z. B. von einem anderen
Objekt aus, als ung�ltig zu markieren. Gleichzeitig ist es auch
m�glich den \fachw{Slot} konventionell wie eine Funktion aufzurufen
(``\code{compileNextTime()}''), da der \fachw{Slot} ein
Funktionsobjekt ist.

\subsubsection{SGLObj als abstrakte Klasse}

Welche API\hyp{}Funktionen beim Zeichnen bzw. Kompilieren des Objektes in
welcher Form aufgerufen werden, h�ngt davon ab, was gezeichnet werden
soll. Die eigentliche Zeichenfunktion, die beim Kompilieren des
Operationspuffers aufgerufen wird, ist deshalb rein virtuell
(\code{SGLObj::generate()}). Sie wird erst durch die eigentliche
Objektklasse implementiert.  So haben die verschieden Objekte eine
gemeinsame Verwaltung und trotzdem volle Kontrolle �ber ihre
Zeichenoperationen.  Neue Objektklassen k�nnen somit auch au�erhalb
der Basisbibliothek implementiert bzw. spezialisiert werden.


\subsubsection{Positionierung mittels Transformationsmatrix}

Neben der Schnittstelle zum Zeichnen des Objektes und zur Verwaltung
des Operationspuffers bietet \code{SGLObj} au�erdem von
\code{SGLMatrixObj} geerbte Funktionen zur Manipulation der
Transformationsmatrix des Objektes. In Kombination mit den Matrizen
der Sicht bestimmen sie Lage und Skalierung des Objektes. Es stehen
Funktionen zum Rotieren, Verschieben und Skalieren zur Verf�gung.  Sie
nehmen die entsprechenden �nderungen an der Transformationsmatrix vor.
Bei Aufruf dieser Funktionen muss auf die Reihenfolge geachtet werden,
da sie sich auch gegenseitig beeinflussen.  Jegliche Operationen sind
als nicht umkehrbar definiert. Der Aufruf der Transformationsmatrix
wird mit in den Operationspuffer aufgenommen, er wird daher bei allen
Zeichenaufrufen von allen Sichten aus implizit zusammen mit den
restlichen Zeichenoperationen aufgerufen. Das Objekt hat dadurch f�r
alle Sichten die selbe Lage im Raum, w�hrend jede Sicht ihre eigenen
Sichttransformationen dar�berlegt. Das entspricht dem entworfenen
Konzept von mehreren Sichten die aus unabh�ngigen Richtungen den
selben Raum betrachten. Sie ``sehen'' das Objekt zwar an der gleichen
Stelle im Raum, aber aus verschiedenen Perspektiven.

\subsection{Basisklasse zur Manipulation der Sicht (\code{SGLBaseCam})}

Die Instanzen der Raumklasse verwalteten ihre Sichttransformationen
nicht selbst, sondern �berlassen dies dem Kameraobjekt, das ihnen
zugeordnet ist. Dabei bestimmt die Kamera ob und wie sie auf
Ereignisse wie zu Beispiel einen Mausklick reagiert. 

\inclfigure{hbt}{classSGLBaseCam__coll__graph}{10cm}{\code{SGLBaseCam} und ihre Beziehungen zu anderen Klassen}{SGLBaseCam:Bez}

Durch das Austauschen des Kameraobjektes l�sst sich somit die
Interaktion der Sicht mit dem Anwender beeinflussen. Gleichzeitig erbt
\code{SGLBaseCam} von \code{SGLObj}, Kameras lassen sich also auch
zeichnen. Das wurde z.B. in den Beispielbildern der \fachw{GUI}
genutzt, um die Lage und die Parameter der Kameras anzudeuten.


\subsubsection{Abstraktion der Sichttransformationen}

Die Lage der Kamera im Raum wird im Gegensatz zur Lage der anderen
Objekte nicht durch ihre Transformationsmatrix bestimmt.  Wird eine
Kamera bewegt, �ndert sie nicht ihre Transformationsmatrix, sondern
die Transformationsmatrizen der Sicht und bestimmt so zentral wie der
Raum in der Sicht gezeichnet wird, der die Kamera zugeordnet ist.
Wird die Kamera virtuell nach rechts bewegt, �ndert dies die
Transformationsmatrizen derart, dass alles im Raum entsprechend weiter
links gezeichnet wird. Das gleiche gilt auch f�r alle anderen
virtuellen Bewegungen der Kamera im Raum.  Die meisten dazu n�tigen
Matrixoperationen werden wieder vom \fachw{OpenGL\hyp{}Renderer}
ausgef�hrt. Die \fachw{OpenGL-API} bietet Funktionen, die auf der
Basis direkter Angaben zur Lage der Kamera und des Punktes an den sie
``blickt'' entsprechende �nderungen an den Transformationsmatrizen der
Sicht vornimmt. Die Kamera ist somit das einzige Objekt, dessen Lage
im Raum nicht durch seine Transformationsmatrix, sondern durch direkte
Koordinaten bestimmt wird.

Das Zoomen kann auf mehrere Arten erfolgen. Die einfachste und
robusteste Methode ist, die Kamera einfach n�her an das betrachtete
Objekt heran zu bewegen. Die Zoomoperation w�re damit auch nur eine
Bewegung der Kamera. Die zweite M�glichkeit besteht in einer Verengung
des Sichtfeldes der Kamera. Da so weniger dargestellt wird, aber die
Gr��e der Anzeige selbst nicht ver�ndert wird, erscheint das
Angezeigte gr��er.  Dies ist vergleichbar mit dem �ndern der
Brennweite einer realen Kamera.  Die Dritte M�glichkeit besteht darin,
den Zoom als zweidimensionale Skalierung direkt auf den Anzeigepuffer
anzuwenden.  Der Zoomfaktor ist dabei allerdings auf ganze Zahlen
beschr�nkt, und darf nicht kleiner als eins sein. Diese dritte Methode
wird in dieser Arbeit nicht zur Anwendung kommen.


\subsubsection{Anpassung an Form und Gr��e des Sichfensters}

Eine �nderung von Form und Gr��e der Anzeige hat tiefgreifende Folgen
auf die Darstellung. Der \fachw{Renderer} muss den Anzeigepuffer
entsprechend anpassen und auch die Kamera selbst muss unter Umst�nden
einige Anpassungen vornehmen.  Die Information, �ber eine �nderung des
Anzeigefensters kommt vom Widgetadapter und wird von der Raumklasse an
die aktuelle Kamera weitergegeben.  Wie diese darauf reagiert, h�ngt
von ihrem Verhaltensprofil ab.  In jeden Fall informiert sie den
Renderer �ber die �nderung der Anzeige. Unternimmt sie nichts
weiteres, muss z. B.  bei einer Verkleinerung der Anzeige das in
seiner Gr��e unver�nderte Sichtfeld der Kamera auf eine kleinere
Fl�che projiziert werden. Das Angezeigte wird dadurch unweigerlich
verkleinert. Die Kamera kann diesem Effekt durch Zoomen
entgegenwirken.  Welche der drei m�glichen Methoden des Zoomens sie
dabei anwendet h�ngt wiederum von ihrem Profil ab. Die dritte Methode
steht dabei aber nicht zur Wahl, da sich mit ihr nicht stufenlos
zoomen l�sst. F�r die beiden Anderen muss der Abstand der Kamera zum
betrachteten Punkt bzw. der Winkel des Sichtfeldes bestimmt werden. Zu
diesem Zweck werden die Eckpunkte des Fensters in den Raum projiziert.
Dazu wird die in Abschnitt \vref{cursor:pos} beschriebene
Positionsbestimmung angewendet, wobei die aktuelle Entfernung der
Kamera zu dem betrachteten Punkt als Tiefeninformation fungiert.  Die
neue Entfernung bzw. der Winkel des Sichtfeldes der Kamera l�sst sich
dann aus der H�he dieses projizierten ``Fensters'' wie folgt
bestimmen.

\[\alpha=\arctan \left( \frac{h}{|\vec{s}|} \right)*2\]
\[\vec{s'}=  \vec{s}* \frac{h}{\tan(\alpha/2)|\vec{s}|}\]

Dabei sind \eRaum{\vec{s}} der Vektor von der Kamera zu dem
betrachteten Punkt, \eR{h} die H�he des projizierten Fensters und
\eR{\alpha} der Winkel des Sichtfeldes der Kamera. Es ist zu beachten,
dass der betrachtete Punkt in beiden F�llen gleich bleibt. W�hrend im
ersten Fall das Sichtfeld erweitert wird, �ndert die Kamera im zweiten
Fall ihre Position entlang der Sichtlinie. In beiden F�llen wird nur
die H�he, nicht die Breite des Anzeigefensters ber�cksichtigt.


\subsection{Basisklassen f�r Fl�chenobjekte}
\label{sec:SGLFlObj}

\subsubsection{Allgemeine geometrische Objekte (\code{SGLFlObj})}

\inclfigure{hbt}{classSGLFlObj__inherit__graph}{15cm}{Klassendiagramm f�r \code{SGLFlObj}}{SGLFlObj:Class}

Ein Gro�teil der zu zeichnenden Objekte sind geometrische Objekte, die
eine Oberfl�che haben. F�r sie werden beim Zeichnen weitere
Informationen dar�ber ben�tigt, wie ihre einzelne Grenzfl�chen zu
zeichnen sind. Die Informationen die das Zeichnen der Fl�chen direkt
bestimmen, werden in der eigenst�ndigen Objektklasse
\code{SGLMaterial} gehalten. Allgemeinere Informationen, wie z.
B., ob das betreffende Objektes �berhaupt mit Oberfl�chen, oder
nur als Drahtgittermodell gezeichnet werden soll, liegen im Objekt
selbst.  

\subsubsection{Polygonobjekte und Polygone (\code{SGLPolygonObj} / \code{SGLPolygon})}

\inclfigures{classSGLPolygon__inherit__graph}{classSGLPolygon__coll__graph}{9cm}{Klassendiagramm
  und Beziehungen von \code{SGLPolygon}}{SGLPolygon:Class}

Eine Spezialisierung der Fl�chenobjekte stellen Polygonobjekte dar.
Polygonobjekte unterscheiden sich untereinander nur in Lage und Anzahl
der Polygone, die ihre Grenzfl�chen darstellen. Gezeichnet werden
Polygonobjekte, indem alle ihre Grenzfl�chen gezeichnet werden. Diese
Grenzfl�chen sind eingenst�ndige Instanzen der Klasse
\code{SGLPolygon}, die auch einzeln gezeichnet werden k�nnen. 


Sie erben von \code{SGLObj} und k�nnen daher wie alle anderen Objekte
unter Einbeziehung des Operationspuffers gezeichnet werden. Wie schon
beschrieben, sind die allgemeinem Zeichnenfunktionen
(Pufferverwaltung, Setzen globaler Parameter, Laden der
Transformationsmatrix) und das eigentliche Zeichnen (Kompilieren) in
gesonderten Funktionen implementiert. Die Polygonobjekte k�nnen ihre
allgemeinen Zeichenfunktionen inklusive dem Verwalten des
Operationspuffers wie andere Objekte auch ausf�hren.  Statt der
eigentlichen Zeichnenoperation, d.h der Kompilierung rufen sie aber
sequentiell die Kompilierung ihrer Grenzfl�chen\hyp{}Polygone auf. Die
Lage des Polygonobjektes im Raum wird nur durch die
Transformationsmatrix des Polygonobjektes selbst bestimmt. Die
Transformationsmatrizen der Polygone werden ignoriert, da der Aufruf
der Transformationsmatrix nicht Teil der eigentlichen Zeichenoperation
ist.  Das ist auch sinnvoll, da sich die Lage der Grenzfl�chen
einfacher und zuverl�ssiger durch die Bestimmung ihrer Eckpunkte
relativ zum Mittelpunkt des Polygonobjektes festlegen l�sst. Mit Hilfe
intelligenter Zeiger k�nnen sich diese Polygone auch programmtechnisch
Eckpunkte teilen, die sie geometrisch gemeinsam haben.  Verschiebt
sich der Eckpunkt eines Polygons, verschieben sich damit automatisch
auch alle Eckpunkte anliegender Polygone. Eine normalerweise
Kommunikation zwischen den Polygonen zur ``Meldung'' der �nderung ist
bei Polygonen innerhalb von Polygonobjekten nicht erforderlich, da das
Polygonobjekt sie ohnehin nur als Verbund kompiliert. Bei einer
�nderung muss daher nur das �bergeordnete Polygonobjekt ``informiert''
werden. Diese �nderungen werden aber ohnehin in der Regel von diesem
Objekt ausgef�hrt, somit er�brigt sich auch diese Kommunikation.

\section{Die Widgetadapter}
\label{sec:xxglue}

F�r die Basisbibliothek ist es unerheblich, welches Widgetsystem f�r
das eigentliche Programm verwendet wird. Zwischen beiden existieren
nur zwei Ber�hrungspunkte, das Fenstermanagement und die
Eingabebehandlung.  Das Fenstermanagement umfasst das Anlegen von
Fenstern inklusive eines dazugeh�rigen \fachw{OpenGL}\hyp{}Kontextes,
sowie den Umgang mit dieses Fenster betreffenden Ereignissen.  Zum
Beispiel eine �nderung der Fenstergr��e.  W�hrenddessen ist die
Eingabebehandlung f�r s�mtliche Eingabeereignisse, sowohl gedr�ckte
Tasten, als auch Mausbewegungen zust�ndig.  Das Zeichnen an sich ist
unabh�ngig vom verwendeten Widgetsystem, da der
\fachw{OpenGL\hyp{}Renderer} direkt in den Anzeigepuffer, den
\fachw{OpenGL\hyp{}Kontext} schreibt und so das Widgetsystem umgeht.

Widgetadapter sind gesonderte Bibliotheken, die als Vermittler
zwischen dem jeweiligen Widgetsystem und der Basisbiliothek dienen.
Sie leiten die vom Widgetsystem (und damit vom der Plattform)
kommenden Ereignisse an die entsprechenden Stellen in der
Basisbibliothek weiter.  Implementiert werden Adapter mittels
Ableitungen der entsprechenden Objektklassen aus der Basisbibliothek.
Sie sind damit das erste Beispiel einer Spezialisierung aus den
allgemeinen Klassen der Basisbibliothek.

Im Rahmen dieser Arbeit werden Adapter f�r \fachw{QT} und \fachw{SDL}
implementiert. Die \fachw{QT}-Bibliothek bietet ein m�chtiges
Widgetsystem, das f�r zahlreiche Plattformen verf�gbar ist. Es hat in
jeder Umgebung die gleiche API und reduziert so den
Portierungsaufwand.  Im Gegensatz dazu ist \fachw{SDL} eigentlich eine
auf Spiele spezialisierte Multimediaschnittstelle. Es ist daher kein
Widgetsystem in eigentlichen Sinne, sondern eher mit \fachw{DirectX}
vergleichbar.  Wie auch \fachw{Qt} ist es, im Gegensatz zu
\fachw{DirectX}, Plattformunabh�ngig.  \fachw{SDL} verf�gt �ber
einfache Funktionen zum Anlegen und Verwalten von Fenstern, sowie �ber
eine Eingabebehandlung und erf�llt damit die gestellten Anforderungen.

Bei der Entwicklung der Basisbibliothek dient \fachw{SDL} als
alternative Testumgebung, da es viel einfacher als \fachw{Qt} und
damit leichter zu �berblicken ist.  Das eigentliche
Visualisierungstool wird jedoch ausschlie�lich auf \fachw{Qt}
ausgelegt sein.


\chapter{Die \fachw{watershed}\hyp{}Transformation}
\label{cha:watershed}

Voraussetzungen f�r eine interaktive Objekterkennung in einem
Grauwertbild ist die Aufteilung des Bildes in Unterobjekte.  Diese
Vorsegmentierung gruppiert nach dem \fachw{watershed}\hyp{}Verfahren
\cite{digabel-lantuejoul78,lantuejoul78phd} Punkte, die mit gr��ter
Wahrscheinlichkeit dem gleichen Objekt angeh�ren. Das hei�t, dass die
Wahrscheinlichkeit, dass diese Punkte einem anderen beliebig kleinen
Objekt angeh�ren kleiner ist. 

Die \fachw{watershed}\hyp{}Transformation wurde von
\citet{digabel-lantuejoul78} eingef�hrt, und sp�ter von
\citet{beucher-lantuejoul79a} verbessert. Sie ist die �blichste
Methode zur Segmentierung zweidimensionaler Grauwertbilder. Sowohl die
theoretischen �berlegungen in diesem Kapitel, als auch die
Implementationen im n�chsten Kapitel basieren auf der weiterf�hrenden
Ausarbeitung von \citet{Roer:wshed} zu diesem Thema.

Ist die Vorsegmentierung abgeschlossen, kann der Anwender die
entstehenden Unterobjekte daraufhin interaktiv zusammenf�gen, um
daraus die korrekten Objekte zu formen. Der Vorteil besteht darin,
dass der Nutzer die Kontrolle �ber die Objekterkennung beh�lt. Er kann
au�erdem bei einem Irrtum seinerseits jederzeit ein beliebiges
Unterobjekt entfernen, denn die Hauptobjekte sind lediglich Listen der
Unterobjekte die sie bilden.

\section{Mathematische Grundlagen}

\subsection{Grauwertbilder als Graphen}

\subsubsection{Allgemeine Graphen}

Ein Graph $G=(V,E)$ besteht aus einer Menge von Knoten ($V \subset
\menge{D}$), sowie einer Menge von Knotenpaaren ($E \subseteq V \times V$).
Dabei ist $\menge{D}$ ein n-dimensionaler Datenraum in dem die Punkte $V$
liegen. Die Knotenpaare $E$ werden im Folgenden als Kanten bezeichnet.
In gerichteten Graphen besteht $E$ aus geordneten Paaren, w�hrend in
ungerichteten Graphen die Reihenfolge der Knoten in den Paaren nicht
festgelegt ist.  Die Menge der Nachbarn eines Knotens $p$ $N_g(p)$
ergibt sich aus der Menge aller Knoten $q$, f�r die es eine Kante
$(p,q)$ oder $(q,p)$ gibt. Es gilt:
\[
N_g(p)=\{\forall q \in V | \exists k \in E : k = (p,q) \vee k= (q,p)\} 
\]


Ein Pfad $\pi(p,q)$ der L�nge $\ell$, in einem Graphen $G=(V,E)$ von einem
Knoten $p$ zu einem weiteren Knoten $q$ ist eine Sequenz von Knoten
$(p_0,p_1,\pp,p_{\ell-1},p_\ell)$, f�r die $p_0=p$, $p_\ell=q$
sowie $\forall i \in [0,\ell): (p_i,p_{i+1}) \in E$ gilt.
Existiert ein Pfad $\pi(p,q)$, wird $q$ als von $p$ erreichbar
bezeichnet ($p \rightsquigarrow q$). Ist dabei $p=q$, ist $\pi(p,q)$
ein Kreis.


\subsubsection{Geod�tischer Abstand}

Der Geod�tische Abstand zwischen zwei Knoten $p,q \in V$ ist die L�nge
$d_V(p,q)$ des k�rzesten Pfades zwischen $p$ und $q$ innerhalb der
Menge $V$. Der Geod�tische Abstand zwischen einem Knoten $p$ und einer
Knotengruppe $P \subset V$ l�sst sich daraus ableiten als
$d_V(p,P)=min_{q \elem{P}}(d_V(p,q))$.

%Bild zur Veranschaulichung???

\subsubsection{Geod�tische Einflusszone}
\label{IZ:B}

$B \subseteq V$ sei unterteilt in k Gruppen zusammenh�ngender Knoten
$B_i$ wobei $(i=1 \pp k)$. Dann ist die Geod�tische Einflusszone $iz_V
\subseteq{V}$ wie folgt definiert.
\[iz_V(B_i) = \{p \elem{V} | \forall i \in [1\pp k ]\backslash \{i\}:d_V(p,B_i)<d_V(p,B_j)\}\]
$iz_V(B_i)$ sind also alle Punkte $p$ f�r die gilt, dass alle
$d_V(p,B_j)$ gr��er sind als $d_V(p,B_i)$.

Die Menge $IZ_V(B) \subset V$ ist die Vereinigungsmenge aller
$iz_V(B_i)$ f�r $i=1...k$, d.h.
\[IZ_V(B)=\overset{k}{\underset{i = 1}{\bigcup}} iz_V(B_i)\]

Punkte, die zwar Element von $B$, aber nicht von $IZ_V(B)$ sind,
konnten folglich keiner Einflusszone innerhalb von $B$ eindeutig
zugeordnet werden. Dies tritt ein, wenn geod�tischer Abstand zu
mindestens zwei verschiedenen $B_i \subseteq B$ gleich ist.

\subsubsection{Wertegraphen}

Die Funktion $f:V \mapsto \menge{X}$ des Wertegraphen $G=(V,E,f)$
bestimmt f�r jeden Knoten $p \in V$ einen Wert $f(p) \in \menge{X}$.
Dabei ist $\menge{X}$ ein meist durch die Anwendung bestimmter Werteraum.

Als Pegelgruppe auf dem Pegel $h$ in einem Wertegraph $G=(V,E,f)$ wird
eine Gruppe von, durch Kanten verbundenen, Knoten ($p \in V$) mit dem
selben Wert $f(p)$ bezeichnet. Die Grenze einer Pegelgruppe setzt sich
aus den Knoten der Pegelgruppe zusammen, deren Nachbarn nicht zur
Pegelgruppe geh�ren. Die abfallende Grenze einer Pegelgruppe sind
Grenzknoten deren Nachbarn niedrigere Werte $f(p)$ haben, w�hrend
Nachbarn von Knoten der Ansteigenden Grenze h�here $f(p)$, als die
Knoten der Pegelgruppe aufweisen. Als innere Pegelgruppe werden alle
Knoten der Pegelgruppe bezeichnet, die nicht zur Grenze geh�ren.

Abfallende Pfade sind Pfade f�r deren Knoten $\forall i \in [0,\ell):
f(p_i) \geq f(p_{i+1})$ gilt, w�hrend f�r ansteigende Pfade $\forall
i \in [0,\ell): (f(p_i) \leq f(p_{i+1}))$ gilt.  $\Pi^\downarrow _f
(p)$ bezeichnet die Menge aller abfallenden Pfade, die in $p$
beginnen. Analog dazu ist $\Pi^\uparrow _f (p)$ die Menge aller
ansteigenden Pfade, die in $p$ beginnen.

Als lokales Minimum wird eine Pegelgruppe $P$ bezeichnet, die keine
abfallende Grenze hat, d.h. $\forall p \in P : \Pi^\downarrow _f (p)
= \emptyset$ .

Ein Wertegraph ist Plateaufrei, wenn jeder Knoten, der nicht Teil
eines lokalen Minima ist, einen ``niedrigeren'' Nachbarn hat. In einem
solchen Graphen $G=(V,E,f)$ gilt daher $\forall p \in V :
\Pi^\downarrow _f (p) \neq \emptyset$.

\subsubsection{digitale Grauwertbilder als Wertegraphen}

Ein digitales Grauwertbild ist ein Graph $G=(\menge{D},E,f)$, bei dem
$\menge{D} \subseteq \menge{Z}^n$ ein Satz von Knoten ist, welche
jeweils zu ihren Nachbarn Kanten haben k�nnen. $n \in \menge{Z}$ ist
dabei Anzahl der Dimensionen des Bildes. Diese Knoten sollen im
Folgenden als Punkte bezeichnet werden.  Es sei $E \in [\menge{Z}^n
\times \menge{Z}^n]$ die Menge aller Kanten zwischen benachbarten
Punkten.  Die Funktion $f: \menge{D} \mapsto \menge{N}$ weist dabei
jedem Punkt $p \in \menge{D}$ einen Ganzzahlwert $f(p)$ als Grauwert
zu.


\section{Definitionen der \fachw{watershed}\hyp{}Transformation}

\subsection{Topografischer Abstand in stetigen Bildern}

In einem stetigen Bild existieren keine diskreten Punkte. Der ``Pfad''
von einer Position, zu einer Anderen innerhalb dieses Wertegebirges
ist daher eine stetige Kurve. Sind die Punkte $p$ und $q$ Anfang bzw.
Ende eines solchen Pfades, dann beschreibt die Funktion $\gamma:
\menge{R} \mapsto \menge{D}$ einen Pfad zwischen beiden wenn $\gamma(0)=p$
und $\gamma(1)=q$ sind. F�r die \fachw{watershed}\hyp{}Transformation sind
nur stetig absteigende Pfade interessant. Es gilt somit:
\[
\forall i \in R : 0 \leq i \leq 1 \rightarrow \int(\gamma(i)) < 0
\]

Das Minimum aller Ableitungen der Pfade $\gamma$ innerhalb des
Wertegebirges $\menge{D}$, f�r die $\gamma(0)=p$ und $\gamma(1)=q$ gilt,
beschreibt den steilste Absteig von $p$ nach $q$. Die Topografische
Distanz zwischen $p$ und $q$ ist also definiert als 
\[
T_f(p,q)=\underset{\gamma} \min \int_\gamma \| \nabla f(\gamma(s))\|ds
\]


\subsection{\fachw{watershed}\hyp{}Transformation in stetigen Bildern}

Das Bild $G=(\menge{D},E,f)$ habe einen Satz lokaler Minima $\{m_k\}_{k \in
  I}$ ($I$ sei Menge von Indexen). Der Einzugsbereich $CB(m_i)
\subseteq \menge{D}$ des lokalen Minimums $m_i$ ist definiert als die Punkte,
die topografisch dichter an $m_i$ sind als zu jedem anderen lokalen
Minimum $m_j$ (vergl. Abschnitt \vref{IZ:B} )

\begin{center}
  $CB(m_i)=\{x \in \menge{D} | \forall j \in I \backslash
    \{i\}:f(m_i)+T_f(x,m_i)< f(m_j)+T_f(x,m_j)\}$ 
\end{center}

Eine Wasserscheide $W_{shed}$ ist die Menge der Punkte, die zu keinem
$CB(m_i)$ geh�ren. Sie besteht aus Punkten, welche die gleiche
Entfernung zu mindestens zwei $m_i$ haben. (vergl. Abschnitt \vref{IZ:B} )

\[
W_{shed}(G)=\menge{D} \cap \left({\underset{i \in I}{\bigcup} CB(m_i)} \right)
\]

Die \fachw{watershed}\hyp{}Transformation des Bildes $G$ ist eine Abbildung
$\lambda : \menge{D} \mapsto I \cup \{W\}$, sodass $\lambda(p)=i$, wenn $p \in
CB(m_i)$ und $\lambda(p)=W$ mit $p \in W_{shed}(f)$ wobei $W \notin I$
ein frei bestimmter Index ist. Die \fachw{watershed}\hyp{}Transformation
eines Bildes $G$ weist daher allen Punkten dieses Bildes so Indexe zu,
dass Punkte eines Einzugsbereiches den gleichen Index bekommen w�hrend
in zwei verschiedenen Einzugsbereichen nie der gleiche Index
vorkommt.  Punkte, die nicht eindeutig einem Einzugsbereich zugeordnet
werden k�nnen, werden gesondert behandelt werden. Diesen
``Bergkuppen'' wird ein Index $W \notin I$ zugewiesen.

\subsection{\fachw{watershed}\hyp{}Transformation in diskreten Bildern}
\label{wshed:diskret}


Wird die \fachw{watershed}\hyp{}Transformation f�r stetige Bilder auf
diskrete Bilder �bertragen, tritt folgendes Problem auf: In diskreten
Bildern k�nnen Gebiete mit konstantem $f(p)$, d.h. mit konstantem
Grauwert auftreten.  Sind diese Pegelgruppen von ``h�heren'' Punkten
umgeben, lassen sich ihre Punkte noch einfach dem jeweiligen lokalen
Minima zuordnen.  Jedoch lassen sich Pegelgruppen, die ausschlie�lich
von ``niedrigeren'' Punkten umgeben sind, nicht eindeutig zuordnen
(die \fachw{watershed}\hyp{}Transformation kennt keine ``lokalen
Maxima'').  Diese Punkte m�ssen gesondert behandelt werden, was im
Abschnitt \vref{plateauprob:entw} genauer besprochen wird.

\subsubsection{\fachw{watershed}\hyp{}Transformation durch Absenken}

Der klassische Algorithmus zur \fachw{watershed}\hyp{}Transformation
durch Absenken nach \citet{vincent} setzt ein digitales Grauwertbild
mit $f: \menge{D} \mapsto \menge{N}$ vorraus. Das Maximum $h_{max}$
und das Minimum $h_{min}$ der Grauwerte $f(p)$ aller Punkte $p$
seien bekannt und es gelte $h_{max}<\infty$ bzw. $h_{min}>-\infty$. Da
das Bild aus diskreten Punkten besteht, und da $f: \menge{D} \mapsto
\menge{N}$ eine diskrete Funktion ist, gibt es eine endliche
Anzahl Werte f�r $h=f(p)$. Im Bild kommen endlich viele Grauwerte vor.
Es l�sst sich somit �ber diese Werte iterieren.  $P_h \subseteq
\menge{D}$ sei die Menge der Punkte deren Grauwert kleiner oder gleich
$h$ ist, d.h. $P_h=\{\forall p \in \menge{D}:f(p)\leq h\}$. Jede
Iteration betrachtet lediglich die Menge der Punkte aus ihrem $P_h$.
In jeder Iteration werden die betrachteten Punkte die innerhalb der
geod�tischen Einflusszone der in der vorherigen Iteration entstandenen
Einflusszone liegen dieser zugerechnet. Tritt ein neues lokales
Minimum auf, wird es als neue Geod�tische Einflusszone vermerkt.
Lokale Minima sind immer vollst�ndig Teil eines $P_h$, da alle Punkte
eines lokalen Minima den selben Grauwert haben. Beginnt die Iteration
bei $h_{min}$ l�sst sich zudem ausschlie�en, dass Punkte aus dem
Einzugsgebiet eines noch nicht bekannten Minima betrachtet werden.
%NOCHMAL CHECKEN
F�r den Algorithmus l�sst sich folgende Rekursion definieren.

\[
\left\{ 
\begin{matrix}
X_{h_{min}} &=& \{p \in \menge{D}|f(p) = h_{min} \} = P_{h_{min}}& \\
X_{h+1} &=& MIN_{h+1} \cup IZ_{P_{h+1}}(X_h),& h \in [h_{min},h_{max}) \\
\end{matrix} 
\right.
\]

Eine Geod�tische Einflusszone in $h+1$ kann eine komplett Neue sein,
oder sie ist ein eine Stufe h�her liegender Teil einer Geod�tische
Einflusszone aus $X_h$. In beiden F�llen flie�en ihre Punkte in
$X_{h+1}$ ein.  $MIN_h$ ist dabei die Vereinigung aller lokalen Minima
mit $f(p)=h$. So sammeln sich alle Punkte, die entweder einem lokalen
Minima angeh�ren, oder in der Geod�tische Einflusszone der einen Pegel
tiefer liegenden ``Sammlung'' liegen.

Die Wasserscheide $W_{sched}(G)$ ist die Menge der Punkte, die am Ende
der Iteration ($h=h_{max}$) weder einem bekannten lokalem Minimum
angeh�ren, noch einer Geod�tischen Einflusszone zugeordnet wurden. Es
gilt $W_{shed}(G)=\menge{D}\backslash X_{h_{max}}$


\section{\fachw{watershed}\hyp{}Transformation durch Wurzelsuche}

$G=(\menge{D},E,f)$ sei ein diskretes plateaufreies Grauwertbild. Jeder Punkt
$p$, der nicht Teil eines lokalen Minimums ist, muss daher mindestens
einen Nachbarn $q$ haben, so dass $f(p)>f(q)$ gilt.  Der steilste Abhang
von $p$ ist definiert als
\[LS(p)= \underset{q \in N_G(p) \cup \{p\}}{\max} \left( \frac{f(p)-f(q)}{d(p,q)} \right)\]
$N_G(p)$ ist dabei die Menge der Nachbarn von p in $G$. Wenn $p=q$ gilt,
oder $p$ Teil eines lokales Minima ist, ist $LS(p)$ als $0$ definiert.

\subsection{Topografischer Abstand in diskreten Bildern}
\label{updownstream}

Die Menge der Nachbarn $q$ von $p$, deren ``H�hendifferenz'' maximal,
also gleich $LS(p)$ ist, wird als $\Gamma(p)$ bezeichnet. Analog dazu
hei�t die Menge der Punkte $q$, f�r die $p \in \Gamma(q)$ gilt,
$\Gamma^{-1}(p)$.
%vereinfachen

Die Kosten f�r den Weg von $p$ zu einem Nachbarn $q$ sind definiert als:

\[
cost(p,q)= \left\{
\begin{matrix}
LS(p)*d(p,q) & f(p)>f(q) \\
LS(q)*d(p,q) & f(p)<f(q) \\
\frac{1}{2}(LS(p) + LS(q))*d(p,q) & f(p)=f(q) \\
\end{matrix}
\right.
\]

Darauf aufbauend ist die topografische L�nge des Pfades $\pi=(p_0,\pp,p_\ell)$ zwischen
$p_0=p$ und $p_\ell=q$ definiert als:

\[
T^h _f(p,q) = \underset{i=0}{\overset{l-1}{\sum}}d(p_i,p_{i+1})cost(p_i,p_{i+1})
\]

Der topografische Abstand zwischen $p$ und $q$ ist das Minimum der
topografischen L�ngen aller Pfade zwischen $p$ und $q$

\[
T_f(p,q) = \underset{\pi \in [p \rightsquigarrow q]}{\min}T^h _f(p,q)
\]

und der topografische Abstand zwischen $p$ und einer Menge $A
\subseteq \menge{D}$ ist analog zur geod�tischen Distanz
$T_f(p,A)=\min_{a \in A}T_f(p,a)$

Wir nennen $(p_0,p_1,\pp,p_n)$ den steilsten Abstieg von $p_0=p$ nach
$p_n=q$ wenn $p_{i+1} \in \Gamma(p_i)$ f�r alle $i=0,\pp,n-1$ ist. Ein
Punkt $q$ geh�rt zum \fachw{downstream} von $p$ wenn es einen steilsten
Abstieg von $p$ nach $q$ gibt. Ein Punkt $q$ geh�rt zum \fachw{upstream}
von $p$, wenn $p$ zum \fachw{downstream} von $q$ geh�rt.

\subsubsection{Definition}
\label{wshed_discret:def}

Es sei $f(p)>f(q)$. Der Pfad $\pi(p,q)$ ist genau dann der steilste
``Abhang'', wenn $T^\pi _f(p,q)=f(p)-f(q)$ ist. Ansonsten gilt $T^\pi
_f(p,q)>f(p)-f(q)$.

\[
\forall \pi \in [p \rightsquigarrow q]:\pi = Abstieg_{\max}
\gdw T^\pi _f(p,q)=f(p)-f(q)
\]
\[
\forall \pi \in [p \rightsquigarrow q]:\pi \neq Abstieg_{\max}
\rightarrow T^\pi _f(p,q)>f(p)-f(q)
\]

%Steilste Pfade sind  die topografisch k�rzesten Pfade.

Mit der Einf�hrung der topografischen Distanz f�r diskrete Bilder ist
die restliche Definition f�r Einzugsgebiete von lokalen Minima analog
zum stetigen Fall.

Es folgt, dass $CB(m_i)$ eine Menge von Punkten im \fachw{upstream} des
lokalen Minima $m_i$ ist. Die Wasserscheide besteht aus den Punkten
$p$, die im \fachw{upstream} von mindestens zwei lokalen Minima
liegen. Es gibt also mindestens zwei $\pi_i$, f�r die
$T^{\pi_i}_f(p,q_i)=f(p)-f(q_i)$ gilt, wobei $q_i$ lokale Minima sind.

Jeder Punkt $p$ im \fachw{upstream} eines Wasserscheide\hyp{}Punktes $q$ ist
durch die obige Definition selbst ein Wasserscheide\hyp{}Punkt, denn $q$
liegt gezwungenerma�en im \fachw{downstream} von $p$, und mit ihm die
lokalen Minima von $q$.

\subsubsection{Breite Wasserscheiden}

``Breite'' Wasserscheiden treten bei Anwendung der
\fachw{watershed}\hyp{}Transformation mittels Wurzelsuche immer dann
auf, wenn sich der am steilsten absteigende Pfad von einem Punkt in
seiner ``Steilheit'' nur geringf�gig von von anderen absteigenden
Pfaden unterscheidet. Der Grund daf�r liegt darin, dass aufgrund der
endlichen Zahl der Nachbarn eines Punktes nicht die Optimale
(steilste) Richtung f�r den Abstieg gew�hlt werden kann. Auf Ursachen
und Eind�mmung dieses Problem wird sp�ter noch genauer eingegangen
werden. Obwohl dies kein f�r die Methode der Wurzelsuche spezifisches
Problem ist, kommt es hier h�ufiger als z.B. bei der
Absenken\hyp{}Methode vor.


\subsection{Das Plateauproblem}
\label{plateauprob:entw}
Bis jetzt wurden Plateaus explizit ausgeschlossen, denn mit der
bisherigen Definition kann die topologische Distanz zwischen zwei
Punkten $p$ und $q$ eines Plateaus nicht korrekt bestimmt werden. Es
w�ren $f(p)=f(q)$, was zu $LS(p,q)=0$, $cost(p,q)=0$ und schlie�lich
zu $T^{\pi}_f(p,q)=0$ f�hrt. Daher muss die Kostenfunktion entweder
erweitert, oder die Plateaus entfernt werden. F�r die Wurzelsuche
werden Plateaus explizit vor der eigentlichen
\fachw{watershed}-Transformation entfernt. Die Methode des Absenkens
behandelt Plateaus direkter. Im Abschnitt \vref{vincent:impl} wird
genauer darauf eingegangen werden.

\subsubsection{Beseitigung von Plateaus}

Es sei $\Pi^\downarrow _f$ die Menge aller absteigenden Pfade von $p$
$( p \in \menge{D}[\exists q \in \menge{D}|\forall \pi(p,q) \in \Pi^\downarrow
_f(p):f(q) < f(p)])$ und $len(\pi)$ sei die geod�tische L�nge des
Pfades $\pi$. $G=(\menge{D},E,f)$ sei ein diskretes Bild, und die Funktion
$d:\menge{D} \mapsto \menge{N}$ sei wie folgt definiert:

\[
d(p) = \left\{ 
\begin{matrix}
0 & \Pi^\downarrow _f(p) = \emptyset \\
\min_{\pi \in \Pi^\downarrow _f(p)}len(\pi) & sonst\\
\end{matrix}
\right.
\]

Unter der Annahme $L_C = \max_{p\in \menge{D}}d(p)$ wird die Wertefunktion %sicher max ? nochmal pr�fen
$f_{LC}$ des plateaubereinigten Bildes $G_{LC}=(\menge{D},E_{LC},f_{LC})$ basierend
auf der Wertefunktion $f$ des Originalbildes bestimmt durch:

\[
f_{LC}(p) = \left\{ 
\begin{matrix}
L_C * f(p) & d(p)=0 \\
L_C * f(p)+d(p)-1 & sonst\\
\end{matrix}
\right.
\]

Die Funktion $d$ hat bei lokalen Minima den Wert $0$, bei allen
Anderen entspricht $d(p)$ der L�nge des Pfades zu Punkten mit
geringerem Grauwert. Die Relation zwischen den Punkten ist definiert
durch $x \sqsubset y \gdw f_{LC}(x)< f_{LC}(y)$. Desweiteren muss sich
auch  der Graph des Bildes �ndern. $G_{LC}$ stellt einen
gerichteten Graphen dar, in dem nur B�gen von ``h�heren'' Punkten zu
``niedrigeren'' Punkten f�hren. F�r $E_{LC} \subseteq \menge{D} \times \menge{D}$ gilt
also

\[
(p,q) \in E_{LC} \gdw q \in \Gamma(p)
\]

%Nach der Transformation nach unterer Grenze $f^* = f_{LC}$ liefert die
%Funktion $T_{f*}$ eine Distanz in $D' x D'$ wobei $D'=D$ gilt.

%Def:
%Es sei $G=(D,E,f)$ ein digitales Grauwertbild, der $LC$-Graph ist
%def. als $(p,p') \in E' \gdw p' \in \Gamma(p)$

Auf dem Gebiet eines Plateaus wird eine Kante von $p$ nach $q$
erzeugt, wenn die geod�tische Distanz zur Kante des Plateaus f�r $p$
gr��er ist, als f�r $q$. Wenn also $q \sqsubset p$ gilt. Der Graph
$G_{LC}$ ist durch diese Definition implizit azyklisch.

%\subsubsection{Def: Watershed\hyp{}Transformation nach topologischer  Distanz}

%Es sei $f$ ein Grauwertbild mit $f^* = f_{LC}$ als Untere Grenze von
%$f$. Es sei $(m_i)_{i \in I}$ die Menge aller Minima von $f$. Das
%Sammelbecken $CB(m_i)$ von $f$ ist def als das Sammelbecken von $f^*$:
%
Die Bestimmung des Einzugsbereichs verh�lt sich in einem solchen Bild
wie folgt:
\[
CB(m_i)=\left\{ p \in \menge{D} | \forall j \in I \backslash
  \{i\}:f_{LC}(m_i)+T_{f_{LC}}(p,m_i) < f_{LC}(m_j) + T_{f_{LC}}(p,m_j) \right\}
\]

%In der Praxis wird der Schritt der ``Unteren Grenze'' oft nicht
%vollst�ndig ausgef�hrt. Plateau\hyp{}Pixel werden dann anderweitig einem
%bestimmten Sammelbecken zugeordnet. Das ist z.B. bei auf sogennanten
%``geordneten Ketten'' beruhenden Algorithmen der Fall.



\chapter{Implementierungen der \fachw{watershed}\hyp{}Transformation}
\label{cha:watershed_impl}

\section{Allgemeine Datenstrukturen}

Die Struktur des Graphen eines Bildes, d.h. die Liste der Punkte ($D$)
und die Liste ihrer Kanten bzw. B�gen untereinander ($E$) sind relativ
unabh�ngig von den Grauwerten dieses Bildes ($f$). Die Menge der
Punkte hat sich in den bisherigen Ausf�hrungen nie ge�ndert. Es wurden
stets die selben Punkte verwendet. Lediglich ihre Beziehungen
zueinander haben sich einmal ge�ndert. Aus Kanten zwischen
geod�tischen Nachbarn wurden gerichtete B�gen zu Punkten mit
niedrigerem Funktionswert. Dem stehen mehrere Werte entgegen, die eine
Funktion $f$ f�r einen bestimmten Punkt aus $D$ liefern kann. Es ist
daher sinnvoll, Daten und Struktur eines Bildes in der Implementation
zu trennen.


\subsection{Der Container f�r Bilddaten \code{Bild}}

Die Bilddaten sind die, jedem Punkt aus der Menge $D$ zugeordneten
Grauwertwerte in den verwendeten digitalen Bildern. Ist $F$ das
Grauwertgebirge eines Bildes, so bildet die Funktion $f$ aus der Menge
der Punkte in die Menge der Grauwerte dieses Bildes ab ($f:D\mapsto
F$). Die Klasse \code{Bild} bietet einen Container f�r die Bilddaten
und gleichzeitig eine Funktion, die f�r einen �bergeben Punkt $p \in
D$ aus diesen Daten das entsprechende von $f(p)$ liefert. Im
Nachfolgenden werden f�r verschiedene Funktionen $f$ zur Abbildung in
verschiedene Grauwertgebirge verschiedene Instanzen dieser Klasse
verwendet werden.  Diese Instanzen werden vereinfachend als ``Bild X''
bezeichnet werden, wobei ``X'' der Bezeichnung der Funktion $f$
entspricht (z.B. ``Bild f'' oder ``Bild $f_{LC}$'').

\lstinputlisting{Bild.cpp}

Der Codeausschnitt zeigt, dass \code{Bild} eine parametrisierte Klasse
ist, deren Instanzen an den Wertebereich angepasst werden k�nnen, in
den die verwendete Funktion abbilden soll. Der Konstruktor legt mit
den Informationen �ber die Dimensionen des Bildes einen entsprechend
gro�en Speicherbereich an. Dabei greift er je nach Voraussetzungen auf
die g�nstigste Systemfunktion zur�ck. Der �berladene \code{[]}-Operator
stellt die Funktionalit�t von $f(p)$ zur Verf�gung. Das hei�t er
bildet die �bergebenen Punkte in den Wertebereich des Bildes ab. Dabei
gibt er Referenzen zur�ck Es ist somit auch m�glich Werte zu
schreiben ($f(p) \leftarrow x$).

\subsection{Der Container f�r Graphendaten \code{Punkt}}
\label{class:Punkt}

Die Struktur des Graphen eines Bildes wird indirekt durch die einzelnen
Knoten des Graphen gespeichert. Jeder dieser Knoten stellt einen Punkt
des Bildes dar. Neben Informationen �ber seine Position innerhalb des
Bildes h�lt jede Punkt eine Liste mit Referenzen auf Punkte, zu denen
er B�gen bzw. Kanten hat.

\lstinputlisting{Punkt.cpp}

Die Punktlisten \code{neighboursA} und \code{neighboursB} halten
Referenzen auf senkrechte und waagerechte bzw. auf diagonale Nachbarn
der Punkte. Diese zwei Listen bilden somit die Kantenmenge $E$ des
Graphen des urspr�nglichen Bildes. Die B�gen aus $E_{LC}$ werden durch
die Liste \code{sortetNeighbours} gehalten. Diese Liste speichert
zus�tzlich zu der Referenz des Punktes, zu dem der jeweilige Bogen geht,
den Wert von $T^\pi _f(p,q)$. Sie sortiert die Eintr�ge nach diesem
Wert, so dass die steilsten B�gen am Ende der Liste stehen. Ist die
Liste vollst�ndig, ergibt \code{sortetNeighbours.rbegin()->first}
dadurch immer den maximalen Abstieg und alle Eintr�ge ab
\code{sortetNeighbours.lower\_bound(LS)} haben mindestens den Abstieg
$LS$.

\section{\fachw{Watershed}\hyp{}Transformation mittels Wurzelsuche}
\label{wurzel:impl}

Die auf der Wurzelsuche basierende Implementation der
\fachw{watershed}\hyp{}Transformation ist rekursiv und jede
Verzweigung der Rekursion l�uft isoliert ab. Sie is somit unabh�ngig
von allen anderen Verzweigungen des selben Rekursionsschrittes. Die
Wurzelsuche l�sst sich daher bei Bedarf leicht parallelisieren.

Dieser von \citet{oai:CiteSeerPSU:114309} publizierte Algorithmus zur
Umsetzung der \fachw{watershed}-Transformation setzt ein
plateaubereinigtes Bild mit dem Graphen $G_{LC}$ vorraus.  Jeder Punkt
$p$ des Graphen $G_{LC}$ hat dabei einen Bogen zu seinem Nachbarn $q
\in N_G(p)$ ($(p,q) \in E_{LC}$), wenn $f_{LC}(q)<f_{LC}(p)$ gilt.
Diese B�gen k�nnen also nur ``von oben nach unten'' gehen, was Kreise
ausschlie�t. F�r die Wurzelsuche muss das Bild plateaufrei sein. Der
erste Schritt besteht somit darin, das urspr�ngliche Bild in ein
plateaufreies Bild zu transformieren.

\subsection{Von $G=(D,E,f)$ zum plateaufreien $G_{LC}'=(D,E,f_{LC})$ }

Den �berlegungen in Abschnitt \vref{wshed:diskret} folgend erzeugt die
folgende Funktion aus den Strukturinformationen ($D$ und $E$) und den
Grauwerten ($f$) des originalen Bildes einen neuen Satz Grauwerte
$f_{LC}$, indem sie f�r alle Punkte $p \in D$ den Wert von $f_{LC}(p)$
bestimmt.

\lstinputlisting{LC.cpp}

Die Variable \code{dist} wird mit $1$ initialisiert.  Die erste
Schleife [Zeilen 6-14] tr�gt f�r alle Punkte $p$ f�r die $\exists q
\in N_{G_E}(p):f(q)<f(p)$ gilt in $f_{LC}$ einen ung�ltigen Wert ein,
und h�ngt diese Punkte hinten an eine FIFO\hyp{}Liste an. Alle anderen
Punkte sind entweder lokale Minima oder Teil eines Plateaus. F�r sie
wird in $f_{LC}(p) \leftarrow 0$ gesetzt. Nach Abschluss der Schleife
wird abschlie�end eine Trennmarkierung an die Liste angef�gt.

Die zweite Schleife [Zeilen 17-41] pr�ft, ob einer der Punkte aus der
FIFO\hyp{}Liste gleichwertige Nachbarn hat. In der Liste k�nnen zwar
ausschlie�lich Punkte vorkommen, die mindestens einen niederwertigeren
Nachbarn haben.  Dennoch kann ein weiterer Nachbar eines dieser Punkte
gleichwertig sein.  Ein solcher Punkt, der sowohl gleichwertige als
auch niederwertigere Nachbarn besitzt, ist der Rand eines Plateaus.
All seine gleichwertigen Nachbarn, die selbst keinen niederwertigeren
Nachbarn haben, werden am Ende der FIFO\hyp{}Liste angef�gt (hinter
der Markierung). Es ist zu beachten, dass aufgrund der ersten Schleife
an dem, f�r einen Punkt $p$ in $f_{LC}$ eingetragenen Wert leicht zu
erkennen ist, ob dieser Punkt niederwertigere Nachbarn hat. Dies muss
nicht wiederholt gepr�ft werden. Nachdem sich die gefundenen
Plateaupunkte in der Liste befinden, wird f�r sie in dem Bild $f_{LC}$
ein ung�ltiger Wert eingetragen.  Das verhindert, dass sie ein zweites
mal von einem anderen Randpunkt des Plateaus aus an die Liste
angeh�ngt werden k�nnen. Die Schleife setzt zudem f�r jeden anderen
Punkt der Liste den Wert in $f_{LC}$ auf die Entfernung, die dieser
Punkt zum n�chsten niederwertigeren Punkt hat. Die Entfernung betr�gt
anfangs $1$, da sich nur Punkte mit niederwertigeren Nachbarn in der
Liste befinden.  St��t die Schleife jedoch zum ersten Mal auf die
Trennmarkierung, m�ssen alle folgenden Punkte Plateaupunkte sein. Sie
sind daher mindestens zweit Schritte von einem niederwertigeren Punkt
entfernt.  Diese Punkte m�ssen einen Schritt �ber den Nachbarn
vornehmen, der sie in die Liste eingetragen hat, um zu einem
niederwertigeren Punkt zu gelangen. Die Nachbarn dieser Punkte k�nnen
nur noch gleichwertig sein, da der Rand des Plateaus bereits
bearbeitet wurde. Die Pr�fung auf Gleichwertigkeit ist aus diesem
Grund eigentlich nicht mehr notwendig. Alle Nachbarn, die noch nicht
untersucht wurden, werden wiederum hinter einer neuen Trennmarkierung
in die Liste aufgenommen.  Ihre Entfernung zu einem niederwertigeren
Punkt ist abermals um einen Schritt gr��er geworden. Auf diese Weise
``wachsen'' alle Plateaus von au�en zu. Nach dem Ende der Schleife
entspricht der f�r jeden Punkt $p$ in $f_{LC}$ eingetragene Wert
$d(p)$.

Die letzte Schleife [Zeilen 42-52] ermittelt nur noch f�r jeden Punkt
$p$ $f_{LC}(p)$ aus seinem nun bekannten $d(p)$ und tr�gt diesen Wert
in $f_{LC}$ ein. An dieser Stelle lassen sich auch leicht lokale
Minima identifizieren. Alle Punkte $p$, f�r die noch immer
$f_{LC}(p)=0$ gilt k�nnen nur zu einem Minimum geh�ren.  Diese
nebenbei erhaltene Information wird in dem Bild $idx$ abgelegt, da sie
noch beim Aufbau des Graphen gebraucht wird. Der entstandene Graph
$G_{LC}'=(D,E,f_{LC})$ ist plateaufrei. Es gibt neben den lokalen
Minima keinen Punkt ohne nach $f_{LC}$ niederwertigeren Nachbarn mehr.
F�r alle Punkte $p \in D \cap {\underset{i \in I}{\bigcup}m_i} $ ($I$
sei die Menge der Minima\hyp{}Indexe) gilt $\Pi^\downarrow
_{f_{LC}(p)} \neq \emptyset$. Das Bild ist plateaufrei.

\subsection{Von $G_{LC}'=(D,E,f_{LC})$ nach $G_{LC}=(D,E_{LC},f_{LC})$}

Die B�gen f�r $E_{LC}$ k�nnen nun relativ einfach erzeugt werden.  Der
Abstieg von einem Punkt $p$ zu einem anderen Punkt $q$ wird durch
$\frac{f_{LC}(p)-f_{LC}(q)}{d(p,q)}$ bestimmt. Es handelt sich um ein
diskretes Bild in dem der geod�tische Abstand zweier Nachbarn immer
$1$ ist. Daher gilt somit $\forall p \in D[ \forall q \in N_{G_{LC}} :
d(p,q)=1]$, denn $p$ und $q$ sind Nachbarn. Es wird also ein Bogen von
einen Punkt $p$ auf seinen Nachbarn $q$ in $E$ eingetragen, wenn
$f_{LC}(p)-f_{LC}=LS(p)$ maximal ist. Werden diagonale Nachbarn mit
einbezogen, muss f�r diese $d(p,q)=\sqrt{2}$ bzw.  $d(p,q)=\sqrt{3}$
bei Volumenbildern angenommen werden.

\lstinputlisting{build_graph.cpp}

Da $LS(p)$ unbekannt ist, weicht diese Implementierung leicht von der
obigen Definition ab. Um $LS(p)$ zu bestimmen m�ssten alle Nachbarn,
die mindestens um $1$ niederwertiger sind untersucht werden um das
Maximum, also $LS(p)$ zu finden. Stattdessen nimmt die Schleife alle
Nachbarn $q$ mit $f_{LC}(p)-f_{LC} \geq 1$ zusammen mit diesem Wert
($f_{LC}(p)-f_{LC}$) in eine Liste auf.  Die Liste wird dabei nach
diesem Wert sortiert, daher sammeln sich die Punkte, f�r die
$f_{LC}(p)-f_{LC}$ maximal, und damit gleich $LS(p)$ ist automatisch
am Ende der Liste.

Eine Eintragung in dieser Liste wird als Bogen von dem Besitzer dieser
Liste zu dem eingetragenen Punkt betrachtet. Nach Abschluss dieser
Funktion gibt es f�r jeden Punkt, der nicht zu einem lokalen Minimum
geh�rt, einen Pfad zu einem solchen.  Da ein Punkt eine konstante
maximale Anzahl von Nachbarn hat, bleibt maximale L�nge der sortierten
Liste konstant. Der Aufwand f�r das Einf�gen in diese Liste ist daher
ebenfalls konstant bzw. unabh�ngig von der Gr��e des Bildes.
Gleichzeitig hat jeder Punkt mindestens ein Bogen, der zu einen
niederwertigeren Punkt f�hrt, so dass in jeder dieser Listen
mindestens ein Eintrag steht.

Die einzigen Punkte, f�r die das nicht zutrifft, sind Punkte, die zu
einem lokalen Minimum geh�ren. F�r diese Punkte $p$ gilt immernoch
$\Pi^\downarrow _{f_{LC}(p)}=\emptyset$. Sie k�nnen somit keinen Bogen
zu einem niederwertigeren Punkt besitzen. Um dem Algorithmus
garantieren zu k�nnen, dass von jedem Punkt ein Bogen zu einen Punkt
f�hrt, m�ssen diese Punkte ein Bogen zu sich selbst besitzen. In ihrer
Liste f�r niederwertigere Punkte m�ssen sie folglich selbst als
einziger Eintrag mit dem Abstand $0$ eingetragen werden. Diese Punkte
k�nnen auf diese Weise als Wurzel erkannt werden. Jedoch muss davon
ausgegangen werden, dass es f�r jedes lokale Minimum $m$ mehr als
einen Punkt $p \in m$ gibt.  Diese Punkte $p$ mit $\Pi^\downarrow
_f(p)=\emptyset$ gelten als Wurzel des Graphen $G_{LC}$ (Der Graph
kann mehrere Wurzeln haben).  Jeder Wurzel in $G_{LC}$ wird in der
n�chsten Funktion ein eindeutiger Index zugewiesen werden.  Jeder
dieser eindeutigen Indexe steht wiederum f�r ein lokales Minimum. F�r
alle anderen Punkte wird in der eigentlichen Wurzelsuche �ber die
bereits erstellten B�gen die am n�chsten liegende Wurzel gesucht, und
ihnen der Index dieser Wurzel zugewiesen. Besteht ein lokales Minimum
aus mehreren Punkten, ist es m�glich, dass Punkten, die im
Einzugsbereich dieses Minimums liegen, verschiedene Wurzeln zugewiesen
werden. Sie werden also f�lschlicherweise als zu verschiedene Minima
geh�rend angesehen. Um das zu verhindern, darf ein Minimum nur aus
einer Wurzel bestehen. Zu diesem Zweck wird f�r jedes Minimum $m$ ein
beliebiger Punkt $p_m \in m$ ausgew�hlt, der stellvertretend als
Wurzel f�r das Minimum steht. Allen Punkte aus $m$ (auch $p_m$) wird
daraufhin statt dem Bogen auf sich selbst einen Bogen auf $p_m$
eingetragen.  Dabei bleibt die topographische L�nge dieser B�gen zu
$p_m$ $0$, da alle Punkte aus $m$ gleichwertig sind. Folglich hat der
ausgew�hlte Punkt $p_m$ als einziger Punkt des entsprechenden lokalen
Minimums einen Bogen auf sich selbst und ist somit die einzige Wurzel
des Minimums $m$.

Die Punkte eines lokalen Minimums haben per Definition den gleichen
Wert wie die Wurzel, die dieses Minimum repr�sentiert. Es wird um sie
herum immer Punkte geben, die h�her liegen, sonst w�ren sie nicht Teil
eines Minimums. Folglich k�nnen solche Punkte h�chstens einem Minimum
angeh�ren, sie sind nur von einem lokalen Minimum aus erreichbar, ohne
dass der Pfad �ber h�here Punkte verlaufen muss. Ist ein lokales Minimum
$m$ bekannt und durch einen beliebigen Punkt $p_m \in m$
repr�sentiert, lassen sich alle gleichwertigen Punkte in Nachbarschaft
von $p_m$ diesem Minimum durch einfache Wertevergleiche zuordnen.
Diese Operation wird rekursiv f�r jeden gleichwertigen Nachbarn von
$p$ wiederholt, indem dieser wiederum gleichwertige Nachbarn sucht,
die noch nicht eingeordnet wurden. Diese Rekursion kann auch genutzt
werden, um jeweils jedem Minimum bzw.  seinen Punkten einen
eindeutigen Index zuzuweisen. Dies ist m�glich, da jeder obersten
Verzweigung der Rekursion ein lokales Minimum entspricht.  Alle
Punkte, die zu diesem Minimum geh�ren, werden in dem entsprechenden
Zweig der Rekursion behandelt. Auf diesem Wege kann ihnen der Index
des Minimums zugewiesen werden. Der folgende Quelltextausschnitt zeigt
die zwei Funktionen, die diese Aufgaben erf�llen.

\lstinputlisting{label_minima.cpp} 

Die �u�ere Funktion \code{labelMinima} ruft f�r jeden Punkt die
Rekursion \code{setMinLabel} auf. Wurde eine Rekursion erfolgreich
abgeschlossen, geht die �u�ere Funktion davon aus, dass eine Menge $m$
vollst�ndig ist, und erh�ht den Index. Die Rekursion nutzt die bei der
Plateaubeseitigung gesammelte Information �ber lokale Minima, um zu
erkennen, ob f�r den behandelten Punkt $p$ $p \in m$ gilt. Ein
direkter Wertevergleich ist nicht notwendig. Es reicht zu pr�fen, ob
der behandelte Punkt teil eines Minimums ist. Ist dies der Fall,
bekommt er einen Bogen zu dem �bergebenen Ursprungspunkt. Der Vermerk,
dass er einem Minimum angeh�rt, wird durch den Index des Minimums
�berschrieben, dem er angeh�rt. Dieser Punkt wird ab diesem Zeitpunkt
also nicht mehr als Teil eines Minimums angesehen und dadurch auch
nicht wiederholt betrachtet. Danach wird die Rekursion f�r jeden
seiner Nachbarn ($N_G(p)$) mit dem gleichen Index und dem gleichen
Ursprungspunkt ausgef�hrt. Ist der behandelte Punkt jedoch kein Teil
eines Minimums bricht dieser Zweig der Rekursion erfolglos ab.

Wie bereits in diesem Abschnitt erkl�rt, kann kein Punkt mehreren
Minima angeh�ren. Es wird also kein Punkt mehrfach betrachtet, zumal
dieser nach der ersten Betrachtung ohnehin nicht mehr als Teil eines
Minimums erkannt wird.  Da bereits bekannt ist, welche Punkte einem
Minimum angeh�ren, m�ssen sie nicht mehr durch eine Suche ermittelt
werden. Der Aufwand dieser Operation �bersteigt somit nie $O(n)$ wobei
$n$ die Anzahl der Punkte im Bild ist. In der Praxis liegt der Aufwand
meist weit unter O(n), denn zusammenh�ngende Gruppen gleichwertiger
Punkte in einem lokalen Minimum stellen meist nur einen kleinen Teil
aller Punkte des Bildes dar.

Damit ist der Graph $G_{LC}=(D,E_{LC},f_{LC})$ komplett, und es kann
mit der eigentlichen Wurzelsuche begonnen werden. 


\subsection{Die Wurzelsuche in $G_{LC}$}

Die \fachw{watershed}\hyp{}Transformtion des urspr�nglichen Bildes findet �ber
eine Wurzelsuche in $G_{LC}$ statt. Sie weist jedem Punkt, der noch
keinen Index hat, den Index der Wurzel zu, zu der der Pfad am steilsten
ist. Dadurch wird eine Zuordnung aller Punkte zu einem
Einzugsbereich und damit eine sinnvolle Gruppierung erreicht. Die
Punkte, die auf diese Weise mindestens zwei Wurzeln (d.h. Minima)
zugeordnet wurden, werden wie spezifiziert als Wasserscheide markiert.
Diese Markierung erfolgt �ber einen speziellen Index, der sonst nicht
vorkommt. Wie im folgenden Ausschnitt zu sehen ist, ist auch diese
Funktion wieder zweigeteilt und rekursiv.

\lstinputlisting{root_find.cpp}

Die �u�ere Funktion \code{findRoot} verwendet die rekursive Funktion
\code{resolve}, um die eindeutige Wurzel jedes Punktes zu ermitteln.
Ist die ermittelte Wurzel g�ltig, wird in dem Bild \code{idx} den
Index seiner Wurzel vermerkt. Ist die Wurzel jedoch ung�ltig, geht die
�u�ere Funktion davon aus, dass die Wurzel nicht eindeutig ermittelt
werden konnte. In diesem Fall, wird der Punkt in \code{idx} als
Wasserscheide eingetragen. Die innere Funktion [Zeilen 1-21] sucht f�r
jeden �bergeben Punkt \code{p} die Wurzel. Dazu untersucht sie alle
Punkte \code{q} zu denen von \code{p} aus ein Bogen existiert und
deren gemerkter Grauabstand ($f_{lc}(p)-f_{lc}(q)$) maximal ist ($q \in
\Gamma(p)$) [Schleifenkopf Zeilen 6-8]. Ist \code{q} ein g�ltiger und
normaler Punkt, wird seine Wurzel rekursiv bestimmt. Das Ergebnis
ersetzt den LC-Nachbarn in der Variablen \code{q} [Zeilen 11-12].
Die folgenden Iterationen (weitere LC\hyp{}Nachbarn) laufen bis Zeile
13 gleich ab.  Es wird eine Wurzel f�r den LC\hyp{}Nachbarn in dem
Eintrag \code{i} gesucht, und in \code{q} abgelegt.  In Zeile 15 wird
nun jedoch die neu gefundene Wurzel mit der Wurzel verglichen, die der
Vorherige durchlauf ergeben hat.  Sind beide verschieden wird der
R�ckgabewert der Funktion ung�ltig.  Die Schleife bricht in
diesem Fall ab und gibt den ung�ltigen Wert zur�ck.  Daraufhin wird
die �u�ere Funktion noch den betrachteten Punkt wie beschrieben als
Wasserscheide markieren.

Im Ergebnis besitzen alle Punkte entweder einen g�ltigen Index oder
sind eine Wasserscheide.

\subsection{Zusammenfassung}

Die Umsetzung der \fachw{watershed}\hyp{}Transformation mittels
Wurzelsuche interpretiert ein plateaufreies Bild als
Graphen auf einem plateaufreien Wertegebirge. Sie f�hrt in diesem
Graphen f�r jeden Knoten eine rekursive Wurzelsuche durch, wobei jedes
lokale Minima eine Wurzel ist. 

Hat ein als Punkt interpretierter Knoten niedrigere Nachbarn, bildet
er B�gen zu den Nachbarn, zu denen der Unterschied des Grauwertes
maximal ist. Knoten ohne niedrigere Nachbarn, geh�ren einem lokalen
Minimum an. Sie bilden einen Bogen zu dem Punkt aus, der
stellvertretend f�r das Minimum steht, dem sie angeh�ren.  Dies gilt
ebenfalls f�r den stellvertretenden Punkt selbst.  Damit sind alle Punkte
eines lokalen Minimums (einer Senke) diesem Minimum zugeordnet.  In
dem nun vollst�ndigen Graphen versuchen w�hrend der Wurzelsuche die
restlichen Punkte �ber einen m�glichst steilen Pfad einen Punkt zu
erreichen, der bereits einem Minimum zugeordnet ist. Es wird nicht
gefordert, dass dieser gefundene Punkt selbst Teil eines Minimums ist.
Auf diese Weise k�nnen schon bekannte steilste Pfade wiederverwendet
werden.  Hat ein Punkt ersteinmal einen k�rzesten Pfad gefunden,
besteht eine gute Chance, dass Punkte in seiner N�he diesen Pfad
mitbenutzen k�nnen.  Dabei zeigt sich jedoch ein Problem der
rekursiven Definition des k�rzesten Pfades zur Wurzel. Sie bewirkt,
dass alle Punkte die im \fachw{upstream} einer Wasserscheide liegen
selbst eine Wasserscheide sind. Allerdings ist gerade bei falschen
Wasserscheiden die Wahrscheinlichkeit gro�, dass Punkte in ihrem
\fachw{upstream} liegen.

Eigentlich k�nnen Wasserscheiden nach ihrer Definition nur auf
``Bergspitzen'' vorkommen. Das gilt aber nur, wenn ein Punkt unendlich
viele Nachbarn hat, denn nur dann kann er wirklich den steilsten Weg
nach unten w�hlen. Je weniger Nachbarn ein Punkt hat (Konnektivit�t),
umso weiter muss er mit seinem ``Abstieg'' vom Optimalen (steilsten)
Weg abweichen.  Unter ung�nstigen Bedingungen kann das zu mehreren
Pfaden zu mehreren Minima f�hren. Keiner dieser Pfade ist dabei
wirklich der steilste, sie sind nur zuf�llig gleich weit vom
eigentlichen Optimum entfernt. Der entsprechende Punkt wird trotzdem
als Wasserscheide interpretiert. Diese falschen Wasserscheiden liegen,
wie gesagt, nicht gezwungenerma�en auf einer Bergspitze, folglich gibt
es mit gro�er Wahrscheinlichkeit Punkte in ihrem \fachw{upstream}.

Aus \vref{wshed_discret:def} ergibt sich, dass jeder Punkt, der bei
seiner Wurzelsuche auf eine Wasserscheide trifft, selbst zu einer
wird. Sein bisheriger Weg bis zu dem Wasserscheide\hyp{}Punkt, war so
steil wie m�glich.  Es ist auch bekannt, dass der von dem gefundenen
Punkt ausgehende Pfad maximal steil ist, weshalb der gesamte Pfad der
steilstm�glichste Pfad zu der entsprechenden Wurzel ist. Folglich kann
der suchende Punkt die Zuordnung des aktuellen Punktes �bernehmen,
denn aufgrund der Definition w�rde er wieder den gleichen Pfad finden.
Durch die �bernahme der Zuordnung spart er sich den weiteren Weg nach
unten, und damit Rechenzeit.  Ist der aktuelle Punkt aber eine falsche
Wasserscheide, ist er also f�lschlicherweise mindestens zwei Minima
zugeordnet, pflanzt sich dieser Fehler nun unkontrolliert fort.
Falsche Wasserscheiden ``wachsen'' so zus�tzlich an. Der Fehler dabei
liegt nicht in der ``Abk�rzung'', sondern in der rekursiven Definition
der Wurzelsuche. Er l�sst sich daher nicht vermeiden.

Die Form der Senken in einem Bild und die Konnektivit�t beeinflussen
wieviele Punkte f�lschlicherweise als Wasserscheide deklariert werden.
Die Abbildungen \vref{test_topograph_direct} und \vref{test_topograph_inv}
illustrieren dies (schwarze Bereiche sind Wasserscheiden).

\inclfiguress{test_orig}{test_topograph_direct_low}{test_topograph_direct}{4cm}{\fachw{watershed}\hyp{}Transformation
  eines ung�nstigen Bildes mittels Wurzelsuche}{test_topograph_direct}

Die \abb{test_topograph_direct} zeigt sehr deutlich, wie ein
ung�nstiges Bild zu gro�en Fehlern in den
\fachw{watershed}\hyp{}transformierten Bildern f�hren kann. In dem
gezeigten Fall sorgt die Form der Senke im Original (der schwarze Ring
im linken Bild) f�r Probleme bei der Zuordnung der Punkte.  %Sucht ein
%Punkt innerhalb der wei�en Fl�che in der Mitte eine Wurzel, variieren
%die L�ngen, und damit auch die Steilheit der Pfade zu den Wurzeln
%innerhalb des schwarzen Ringes nur gering.  Aufgrund der begrenzten
%Anzahl von Nachbarn werden dann Pfade bevorzugt, die vom optimalen
%Pfad abweichen. 
Im mittleren Bild k�nnen nur senkrechte und waagerechte Pfade
eindeutig als k�rzeste Pfade erkannt werden. Die Punkte kennen dort
nur die Nachbarn neben, und �ber ihnen. Im rechten Bild kommen noch
die diagonalen Nachbarn hinzu. Wie man sieht entsch�rft diese Erh�hung
der Konnektivit�t das Problem falscher Wasserscheiden zwar merklich,
kann ihre Entstehung aber nicht verhindern.

\inclfiguress{test_inv}{test_topograph_inv}{test_topograph_inv}{4cm}{\fachw{watershed}\hyp{}Transformation
  eines g�nstigen Bildes mittels Wurzelsuche}{test_topograph_inv}

Die \abb{test_topograph_inv} zeigt den umgekehrten Fall. Durch die
g�nstige Form der Senke in der Mitte entstehen dort keine falschen
Wasserscheiden.  Lediglich die tiefen R�nder des Bildes verursachen
waagerecht bzw.  senkrecht verlaufende Wasserscheiden wo keine sein
d�rften. Im Gegensatz zur vorherigen Abbildung kommen sie hier aber
sehr viel seltener vor. Die Erh�hung der Konnektivit�t hat bei diesem
Beispiel keinen Einfluss.

Die Vorteile der \fachw{watershed}\hyp{}Transformation mittels Wurzelsuche
liegen in ihrer relativ hohen Geschwindigkeit und ihrer einfacheren
Parallelisierbarkeit. Sie hat jedoch den Nachteil, dass sie unter
ung�nstigen Bedingungen eine hohe Fehlerrate aufweist.



\section{\fachw{Watershed}\hyp{}Transformation mittels Absenken}

Da die Implementation der \fachw{watershed}\hyp{}Transformation als
Wurzelsuche \cite{oai:CiteSeerPSU:114309}  unter bestimmten Umst�nden zu gro�en Erkennungsfehlern
f�hrt, sollte die Anwendung eine Alternative Implementierung bieten,
die dieses Problem nicht hat. Die dazu ausgew�hlte Implementierung
nach \citet{vincent} basiert statt der Wurzelsuche auf dem Versenken
des Grauwertgebirges (in Wasser). Die Pfade zu den Minima werden nicht
f�r jeden Punkt einzeln, sondern Schritt f�r Schritt f�r alle Punkte
gleichzeitig gebildet. Die Vervielfachung falsch erkannter
Wasserscheiden wird auf diese Weise vermieden. Da au�erdem Plateaus
korrekt behandelt werden, ist eine Vortransformation des Bildes nicht
n�tig.  Dieser Algorithmus besteht jedoch aus mehreren teilweise
ineinander verschachtelten Schleifen.  Sie besitzt daher theoretisch
eine h�here Zeitkomplexit�t als die
\fachw{watershed}\hyp{}Transformation mittels Wurzelsuche.

\subsection{Implementation nach Vincent\hyp{}Soille}
\label{vincent:impl}

F�r die \fachw{watershed}\hyp{}Transformation mittels Absenken m�ssen
s�mtliche Punkte des Bildes ($p \in D$) aufsteigend nach ihrem
Grauwert $f(p)$ sortiert sein. Der Vincent\hyp{}Soille\hyp{}Algorithmus
besteht haupts�chlich aus einer gro�en Schleife, die den Pegel $h \in
\menge{N}$ sukzessive von $h_{min}$ nach $h_{max}$ durchl�uft. Das
entspricht einem ``Versenken'' des Grauwertgebirges und sollte nicht
mit einem ``Volllaufen'' des selbigen verwechselt werden.  Die
``Pegelh�he'' ist im gesamten Gebirge zu jedem Zeitpunkt gleich. Der
folgende Codeausschnitt zeigt den Funktionskopf, einige n�tige
Initialisierungen und den Anfang dieser Schleife.

\lstinputlisting[firstline=1,lastline=22]{vincent.cpp}

Wie im n�chsten Ausschnitt zu sehen, werden innerhalb der Schleife
zuerst alle Punkte $p$ mit $f(p)=h$ untersucht.  Dabei bekommt jedes
neue lokale Minima einen eindeutigen Index, den es sp�ter an alle
Punkte in seinem Einzugsbereich weitergibt. Au�erdem werden alle
Punkte mit $f(p)=h$ markiert. Da die Punkte in der Liste nach $f(p)$
sortiert sind, muss die Liste nicht durchsucht werden, um alle Punkte
mit $f(p)=h$ zu finden.  Hat einer dieser Punkte einen Nachbarn, der
bereits mindestens einer Einflusszone angeh�rt, wird er an eine
FIFO\hyp{}Liste angeh�ngt. In dieser Liste stehen nun alle
``Uferpunkte'' des aktuellen Pegelstandes. Als Abschluss wird an die
FIFO\hyp{}Liste eine Trennmarkierung angeh�ngt.

\lstinputlisting[firstline=23,lastline=42]{vincent.cpp}

Diese ``Uferpunkte'', d.h. alle Punkte $p \in fifoA$ m�ssen nun
indiziert werden. Das hei�t, sie m�ssen entweder eindeutig einem
Bassin zugeordnet, oder als Wasserscheide identifiziert werden. Diese
Aufgabe wird von der im n�chsten Ausschnitt gezeigten Schleife
erf�llt, indem sie jeweils einen Eintrag aus der FIFO\hyp{}Liste der
``Uferpunkte'' herausnimmt und dessen Umgebung untersucht. Ist der
aktuelle Punkt \code{p} eine Trennmarkierung, liegen f�r die aktuelle
Ausbreitung des Wassers keine (weiteren) ``Uferpunkte'' vor.  Das
Wasser muss sich in diesem Fall einen Schritt auf Plateaus ausbreiten,
wenn in dieser Pegelh�he Plateaus existieren. Die Schleife wird dann
neu gestartet, um die neuen ``Uferpunkte'' zu bearbeiten [Zeilen
45-52]. Auf diese Weise wird das Plateauproblem gel�st. Ist der
aktuelle Punkt keine Trennmarkierung werden seine Nachbarn in den
Zeilen 53 bis 77 untersucht. Ist einer seiner Nachbarn schon indiziert
worden, und ist er selbst noch keinem Bassin zugeordnet, �bernimmt er
dessen Zuordnung.  Hat er dagegen schon einen anderen Index, ist er
folglich gleich weit von zwei Bassins entfernt und wird als
Wasserscheide markiert. Trifft nichts davon zu, �bernimmt der
untersuchte Punkt ggf. die Wasserscheide\hyp{}Markierung seines Nachbarn.
Nachbarn, f�r die zwar $f(p)=h$ gilt, die aber nicht zum aktuellen
``Ufer'' geh�ren (erkennbar an \code{dist[*q]==0}) werden mit erh�htem
Abstand hinter der Trennmarkierung an die FIFO\hyp{}Liste angef�gt. Sie
werden verarbeitet werden, sobald die Ausbreitung des Wassers
ausreichend erh�ht wurde.

\lstinputlisting[firstline=43,lastline=79]{vincent.cpp}

Alle Punkte $p$ mit $f(p)=h$ die nach dieser Schleife noch nicht
indiziert wurden, haben nur h�here Nachbarn, sie sind somit Minima.
Der folgende letzte Ausschnitt zeigt die Schleife, die nocheinmal �ber
alle Punkte $p$ mit $f(p)=h$ geht und sie daraufhin untersucht. Dabei
wird auch der Puffer f�r die Abst�nde zur�ckgesetzt. In den Zeilen 85
bis 98 bekommen jedes neue Minima und seine Nachbarn gleicher H�he
einen eindeutigen Index. Im Gegensatz zu der ersten inneren Schleife
[Zeilen 23-40] �ndert diese Schelife den globalen Iterator \code{aktP} direkt,
sodass dieser nach ende der Schleife den ersten Punkt $p$ mit $f(p)>h$
referenziert.

\lstinputlisting[firstline=80,lastline=102]{vincent.cpp}

Damit ist der Block der Hauptschleife zu Ende. Der Pegel $h$ wird um
eins erh�ht und alle Operationen beginnen von vorn. Existieren f�r
einen Pegel $h$ keine Punkte mit $f(p)=h$, wird die erste Schleife
nicht ausgef�hrt, da $f(p) \neq h$. Dadurch steht in der FIFO\hyp{}Liste
nur eine Trennmarkierung, die sofort bei der Initialisierung der
zweiten Schleife [Zeile 43] herunter genommen wird. Die Bedingung
\code{fifoA.size()} kann d.h. nicht mehr erf�llt werden, und die
Ausf�hrung der Schleife wird abgewiesen. F�r die letzte Schleife gilt
das Gleiche wie f�r die Erste, sie wird nicht ausgef�hrt, da bereits f�r
den ersten betrachteten Punkt $p$ $f(p) \neq h$ gilt. F�r dieses $h$
mit $\forall p \in D : f(p) \neq h$ f�hrt die Hauptschleife daher keine
Operationen aus, sondern erreicht sofort das Ende der Schleife und
erh�ht $h$.

Ist der Pegel $h=h_{max}$ erreicht, wurden alle Punkte $p \in D$
behandelt, alle lokalen Minima des Gebirges wurden gefunden und
indiziert sowie alle anderen Punkte ihrem n�chsten Minima zugeordnet
oder als Wasserscheide markiert.

\subsection{Erh�hung der Konnektivit�t}
\label{vincent:conn}

Im Gegensatz zur Wurzelsuche  kennt die Implementation nach
Vincent\hyp{}Soille keine Kostenfunktion.  Stattdessen wird der Geod�tische
Abstand zweier Nachbarn hier fest als $1$ angenommen. Der Abstand
diagonaler Nachbarn, wie sie bei einer Konnektivit�t von $8$
auftreten, betr�gt jedoch $\sqrt{2}$.  Daraus folgt, dass der
Algorithmus, wie er hier implementiert wurde, nur f�r eine
Konnektivit�t von $4$ geeignet ist. Das Fehlen der Kostenfunktion
zeigt sich beim Ausbreiten des Wassers in der Ebene.  Es l�sst sich
umgehen, indem diese Ausbreitung in mehreren Schritten ausgef�hrt
wird. Jeder dieser Schritte ist dann f�r je eine ``Abstandsart''
ausgelegt. Zu diesem Zweck sind gesonderte FIFO\hyp{}Listen und
\code{curdistX}\hyp{}Variablen notwendig.


\subsection{Ergebnisse}

Da auch die Implementation nach Vincent\hyp{}Soille weiterhin auf diskreten
Bildern bzw. diskreten Graphen angewendet wird, werden auch weiterhin
falsche Wasserscheiden erkannt. Es kommt auch hier vor, dass statt dem
eigentlich k�rzesten Pfad zwei suboptimale Pfade gefunden werden.
Dieses grundlegende Problem wurde schon besprochen und l�sst sich
schwer unterbinden, da die Diskretit�t eine grunds�tzliche
Eigenschaft der Eingabedaten ist. Bei Vincent\hyp{}Soille k�nnen sich diese
Fehler allerdings nicht r�ckw�rts �ber die Wurzelsuche fortpflanzen,
da es hier keine Wurzelsuche gibt. Auch Plateaus m�ssen nicht k�nstlich
in H�nge verwandelt werden, sie werden bei dieser Implementation
intuitiver �ber das Ausbreiten des Wassers in der Ebene behandelt.

Eine Transformation nach Vincent\hyp{}Soille erweist sich daher als relativ
unempfindlich gegen�ber ung�nstigen Bildern. Die
\abb{test_vincent_direct} zeigt die Ergebnisse einer Anwendung auf das
gleiche ung�nstige Bild, in dem die Wurzelsuche �ber 45000
Wasserscheiden fand. Analog zu \abb{test_topograph_direct} und
\abb{test_topograph_inv} wurde f�r das mittlere Bild eine vierfache
Konnektivit�t verwendet, w�hrend das rechte Bild das Ergebnis einer
Transformation mit achtfacher Konnektivit�t ist. Das Linke Bild ist
wie immer das Original. Vor allem am mittleren Bild erkennt man die
Robustheit von Vincent\hyp{}Soille, beziehungsweise im Vergleich dazu das
totale Versagen der Wurzelsuche.  W�hrend die Wurzelsuche fast das
gesamte Bild als Wasserscheide ``erkennt'' (ca. 45000 Pixel eines
256x256 Pixel gro�en Bildes wurden als Wasserscheide markiert), zeigt
Vincent\hyp{}Soille ein �hnliches Verhalten wie bei der achtfachen
Konnektivit�t. Es werden nur an den ``�blichen'' Punkten falsche
Wasserscheiden gefunden (ca. 13000 Wasserscheiden in einen 256x256
Pixel gro�en Bild).  Es f�llt au�erdem auf, dass die Schwarzen Linien,
die die Wasserscheiden darstellen hier nie mehr als einen Pixel breit
sind.

\inclfiguress{test_orig}{test_vincent_direct_low}{test_vincent_direct}{4cm}{\fachw{watershed}\hyp{}Transformation
  eines ung�nstigen Bildes  mittels Vincent\hyp{}Soille}{test_vincent_direct}


\abb{test_vincent_inv} zeigt zum Vergleich das Verhalten von Vincent\hyp{}Soille auf
einem g�nstigen Bild. Hier unterscheiden sich beide Implementationen
kaum. Sowohl bei einer vierfachen, als auch bei einer achtfachen
Konnektivit�t fand die Wurzelsuche ca. 1000 Wasserscheiden.
Vincent\hyp{}Soille liegt mit 989 Wasserscheiden nur knapp darunter.

\inclfiguress{test_inv}{test_vincent_inv}{test_vincent_inv}{4cm}{\fachw{watershed}\hyp{}Transformation
  eines g�nstigen Bildes mittels Vincent\hyp{}Soille}{test_vincent_inv}


\section{\fachw{Watershed} in dreidimensionalen Bildern}

Bisher wurde der Einfachheit halber in den Betrachtungen zu den
Implementationen implizit von zweidimensionalen Bildern ausgegangen.
Die Messdaten liegen dagegen dreidimensional vor. Die
Implementierungen m�ssen daher noch an dreidimensionale Bilder
angepasst werden.


\subsection{Anpassung der Algorithmen}

Die Anpassung der Definition der \fachw{watershed}\hyp{}Transformation
erweist sich als relativ unkompliziert. Die
\fachw{watershed}\hyp{}Transformation basiert grunds�tzlich auf Graphen von
Punkten als Eingabedaten. Um die \fachw{watershed}\hyp{}Transformation
auf dreidimensionale Bilder auszulegen reicht es daher f�r den
Ausgangsgraphen $G=(D,E,f)$ festzulegen, dass $D \subseteq
\menge{Z}^3$ gilt. In einem solchen Graphen hat ein Punkt sechs
direkte Nachbarn, der Abstand zu diesen Nachbarn betr�gt $1$.  Liegt
der Punkt im Zentrum eines, innerhalb dieses Graphen gedachten,
W�rfels mit der Kantenl�nge von $2$, so liegen diese Nachbarn jeweils
im Mittelpunkt einer seiner Seitenfl�chen. Weitere acht Nachbarn
kommen als diagonale Nachbarn auf den Koordinatenebenen dazu.  Sie
liegen in der Mitte der W�rfelkanten und ihre Entfernung zum zentralen
Punkt betr�gt $\sqrt{2}$. Die acht Eckpunkte des W�rfels stellen mit
einem Abstand von $\sqrt{3}$ die am weitesten vom zentralen Punkt
entfernten Nachbarn dar. Statt vier bzw.  acht Nachbarn sind bei
dreidimensionalen Graphen Folglich sechs, vierzehn oder zweiundzwanzig
Nachbarn mit drei verschiedenen Abst�nden zu ber�cksichtigen. Jede
Implementierung der \fachw{watershed}\hyp{}Transformation st�tzt sich auf
diese Datenstruktur, und ist daher theoretisch wie die mathematische
Definition der Transformation unabh�ngig von der Dimensionalit�t der
Daten.

Dementsprechend beschr�nkt sich die konkrete Anpassung der Wurzelsuche
\cite{oai:CiteSeerPSU:114309} an dreidimensionale Daten auf die
Verwendung einer, der Lage des Nachbarn entsprechenden Kostenfunktion.
Weiterhin m�ssen lediglich entsprechend mehr Nachbarn eines Punktes
ber�cksichtigt werden. Die bei der Implementierung f�r die Speicherung
der Nachbarn verwendeten Listen passen ihre L�nge automatisch den
Anforderungen an, sodass an dieser Stelle keine �nderung vorgenommen
werden muss. Es erweist sich jedoch als praktischer die Nachbarn von
vornherein entsprechend ihrer Lage in verschiedene Listen
einzusortieren. Damit ist die Art ihrer Nachbarschaft bekannt, und sie
muss bei Bedarf nicht neu ermittelt werden.  Im Ansatz ist dieses
Vorgehen auch schon in der gezeigten zweidimensionalen Implementation
der Wurzelsuche zu sehen.  F�r die diagonalen Nachbarn in einem
zweidimensionalen Graphen mit achtfacher Konnektivit�t wurde dort eine
gesonderte Liste \code{neighboursB} eingef�hrt. Die darin liegende
Information �ber die Entfernungen wurde in der Funktion
\code{buildGraph} f�r die Kostenberechnung verwendet.  F�r die
Verarbeitung dreidimensionaler Graphen mit zweiundzwanzigfacher
Konnektivit�t reicht es daher analog eine weitere Liste
\code{neighboursC} hinzuzuf�gen, und entsprechend zu verwenden.

Wie in Abschnitt \vref{vincent:conn} beschrieben sind Entfernungen
zwischen Nachbarn, die einen anderen Wert als $1$ haben f�r die
Implementation nach Vincent\hyp{}Soille \cite{vincent} problematisch.
F�r die Verarbeitung dreidimensionaler Graphen muss daher analog zu
der dort beschriebenen L�sung ein weiterer Schritt f�r die Ausbreitung
des Wassers eingef�hrt werden.  Die Alternative w�re, sich bewusst auf
eine sechsfache Konnektivit�t zu beschr�nken. Eine solche
Implementation ist merklich schneller als eine Implementation, die
zweiundzwanzig Nachbarn ber�cksichtigen muss.  Die Implementation nach
Vincent\hyp{}Soille hat sich auch bei geringer Konnektivit�t als
robust erwiesen.  Unter Umst�nden ist es daher sinnvoll einen kleinen
zus�tzlicher Fehler zugunsten der einfacheren Implementation und der
Geschwindigkeit in Kauf zu nehmen.


\subsection{Reduzierung des Speicherbedarfs}

Die in den Abschnitten \vref{wurzel:impl} und \vref{vincent:impl}
beschriebenen Implementationen sind bewusst naiv gehalten worden, sie
sind somit gleichzeitig leicht verst�ndlich, und robust. Werden sie
unver�ndert f�r die \fachw{watershed}\hyp{}Transformation
dreidimensionaler Daten �bernommen w�chst mit der Anzahl der Punkte
auch der Speicherbedarf exponentiell an. Zum Beispiel belegt die im
Abschnitt \vref{class:Punkt} beschriebene Datenstruktur zur
Repr�sentation eines Punktes $0.00015259$ MByte. Das erscheint
zun�chst wenig, jedoch besteht eine typische MRT-Aufname des
Kenrspintomografen des CBS derzeit aus $5.12$ Millionen Punkten ($160
\times 160 \times 200$ Punkte), was einen Speicherbedarf $781.26$
MByte allein f�r die Punkte ergibt. Dazu kommen der \fachw{overhead}
durch die Struktur der verwendeten Listen sowie der Speicherbedarf der
Bilddaten und Zwischenergebnisse. Der Speicherbedarf des gesamten
Systems kann so schnell mehrere GByte betragen. Dies steht klar im
Wiederspruch zu dem Anspruch des Programms auch auf einfachen
Arbeitsplatzrechnern anwendbar zu sein.  Der Speicherbedarf der
\fachw{watershed}\hyp{}Implementierungen muss also f�r den praktischen
Einsatz drastisch verringert werden.

Wie schon beschrieben ``verschwendet'' die Datenstruktur \code{Punkt}
den meisten Speicher. Instanzen dieser Datenstruktur kommen in sehr
gro�er Zahl vor. Jedes hier eingesparte Byte kann daher mehrere MByte
Gesamtersparnis bringen. Es f�llt auch auf, dass sowohl die Nachbarn,
als auch die Koordinaten eines Punktes jederzeit aus seiner
Position in der Liste der Punkte und den Dimensionen des Datensatzes
ermittelt werden kann. Ein Punkt wird somit nur noch durch seinen
Index in der Liste der Punkte repr�sentiert. Er ist daher nur noch $4$
Byte gro�, und die $5.12$ Millionen Punkte eines $160 \times 160
\times 200$\hyp{}Bildes belegen nur noch knapp $20$ MByte. Die somit
n�tigen Integeroperationen zur Bestimmung der fehlenden Daten
verlangsamen die Algorithmen zwar etwas, dies ist jedoch angesichts
einer Einsparung von fast $95\%$ vertretbar. Die Speicherung der
Punkte in dynamischen Listen ist oft unn�tig, da sich die L�nge der
meisten in den Implementationen vorhandenen Listen kaum �ndert. Ein
einfaches Array reicht in diesen F�llen aus, und bringt kaum
\fachw{overhead} mit. Die Datenstruktur \code{Bild} wird dagegen
beibehalten, denn sie ist aus Speichersicht bereits optimal. Ihre
Schnittstelle muss auch nicht angepasst werden, denn der Container
\code{Punkt} besteht weiterhin, und kann benutzt werden. Lediglich die
Nachbarlisten werden entfernt. Statt Punkten werden wie beschrieben
nur deren Indexe in den Punktlisten gespeichert. 

Auf die Implementation der \fachw{watershed}\hyp{}Transformation nach
Vincent\hyp{}Soille \cite{vincent} angewendet haben diese Anpassungen
folgende Auswirkungen. Ein einfaches Array h�lt die Indexe der Punkte
von $0$ bis $n-1$, wobei $n$ die Anzahl der Punkte ist. Diese Liste
wird nach den Grauwerten sortiert. Dabei werden aus jedem Index die
Koordinaten bestimmt. Die Koordinaten werden verwendet, um den
Grauwert des Entsprechenden Punktes mittels \fachw{vista}-Funktionen
(\cite{vista} siehe auch Abschnitt \vref{cha:virgil}) direkt aus den
Messdaten zu bestimmen. Sind die Punkt-Indexe in der Liste sortiert,
kann der eigentliche Algorithmus beginnen. Er arbeitet die Liste der
Indexe sukzessive ab, und bestimmt bei Bedarf die Koordinaten der
Punkte wie beschrieben. Die in der urspr�nglichen Implementation
verwendete FIFO-Liste bleibt als einzige in dieser Form erhalten. Sie
kann nie mehr Punkte beinhalten, als in einer Ebene der Volumendaten
vorkommen.  Au�erden werden regelm��ig Punkte zu ihr hinzugef�gt bzw.
entfernt, sodass an dieser Stelle ein einfaches Array ungeeignet ist.
Die verwendeten \fachw{vista}-Funktionen \cite{vista} k�nnen direkt
verwendet werden, um den Grauwert eines Punktes zu ermitteln. Da sie
die Volumendaten schon von sich aus puffern, ist es nicht n�tig diese
zwischen zu speichern.  Der Container f�r die Farbwerte (im Beispiel
\code{im}) kann daher entfallen. Zur Optimierung des Containers f�r
die Markierungen (im Beispiel \code{lc}) wird die Konstante
\code{INIT} zur Markierung unbearbeiteter Punkte von $-1$ auf den
gr��tm�glichen Wert des verwendeten Datentyps ge�ndert. Dies Spart das
Vorzeichenbit ein, und erm�glicht eine effektivere Ausnutzung des
Wertebereichs.  Zeigt sich in der Praxis, dass nie mehr als $2^16-2$
Minima vorkommen, kann der verwendete Datentyp auf $16$ Bit reduziert
werden. Vorerst wird jedoch darauf verzichtet. Die einzigen gr��eren
durch die \fachw{watershed}\hyp{}Transformation angelegten Objekte im
Speicher sind somit die sortierte Punktliste und das Ergebnissbild mit
den indizierten Punkten. Der Speicherbedarf der Implementation der
\fachw{watershed}\hyp{}Transformation nach Vincent\hyp{}Soille
reduziert sich durch diese Optimierungen auf weniger als $50$ MByte.
Die Voraussetzungen f�r die Anwendbarkeit auf �blichen
Desktop\hyp{}Rechnern sind daher f�r diesem Fall wieder erf�llt.

Die Implementation der Wurzelsuche \cite{oai:CiteSeerPSU:114309} kann
�hnlich angepasst werden.  Die Punktliste wird hier nicht sortiert,
die Reihenfolge der Punkte ist irrelevant. Eine sortierte Punktliste
ist daher �berfl�ssig, die Punkte k�nnen direkt �ber ihre Koordinaten
im Bild angesprochen werden. F�r jeden Punkt muss jedoch eine Liste
aller Punkte in seinem \fachw{downstream} angelegt werden. Diese
Listen k�nnen je nach verwendeter Konnektivit�t eine Maximall�nge von
$6$, $14$ oder $22$ Punkten erreichen. In der Regel werden sie aber
weit weniger Punkte beinhalten, da es nur selten vorkommt, da� ein
Punkt im \fachw{upstream} all seiner Nachbarn liegt. Eine feste Ausage
�ber die L�nge dieser Listen ist daher nicht m�glich. Ist eine solche
Liste allerdings einmal gef�llt, �ndert sich ihre L�nge nicht mehr.
Die Listen der Nachbarn im \fachw{downstream} eines Punktes k�nnen
daher als Arrays implementiert werden, die bei der Erkennung der
Nachbarn im \fachw{downstream} mit maximaler L�nge ($6$,$14$ oder
$22$) initialisiert werden. Die erkannten Nachbarn im
\fachw{downstream} des Punktes werden bei der Erkennung in diesem
Array abgelegt. Ist die n�tige L�nge der Liste am Ende der Operation
bekannt, wird das Array entsprechend gek�rzt. Desweiteren speichert
die Wurzelsuche Zwischenergebnisse einiger Operationen als Bild ab,
diese Bilder sind in den meisten F�llen tempor�r. Sie k�nnen daher
gel�scht bzw.  �berschrieben werden, wenn sie nicht mehr ben�tigt
werden. Das �berschreiben der Bilder ist bei gleichem Datentyp
problemlos m�glich, da die Anzahl der Punkte eines Bildes in jedem
Fall konstant bleibt.  Der Speicherbedarf dieser Implementaion der
\fachw{watershed}\hyp{}Transformation kann vor allem aufgrund der
\fachw{downstream}\hyp{}Listen lediglich auf unter $300$ MByte
reduziert werden. Es wird angenommen, dass �bliche
Desktop\hyp{}Rechnern �ber $512$ MByte Arbeitsspeicher verf�gen. Die
Voraussetzungen f�r die Anwendbarkeit auf solchen Rechnern kann somit
knapp erf�llt werden. Ein Speicherbedarf von unter $50$ MByte w�re
realisierbar, wenn die Nachbarn im \fachw{downstream} nicht
gespeichert w�rden. Da sie in diesem Fall bei Bedarf immer wieder neu
ermittelt werden m�ssten, w�rde diese ``Optimierung'' die
Transformation enorm verlangsamen. Die Listen bleiben deshalb vorerst
erhalten.


\chapter{Implementierung des Visualisierungstools}
\label{cha:virgil}

Bei der Implementierung des eigentlichen Visualisierungstools wird auf
die Basisbibliothek, den \fachw{QT}\hyp{}Widgetadapter sowie die, an
die Anwendung auf Volumendaten angepassten Implementierungen der
\fachw{watershed}\hyp{}Transformation zur�ckgegriffen.  Die grafische
Schnittstelle basiert, wie bereits erkl�rt auf \fachw{QT} und die f�r das
Programm verwendeten Daten liegen im \fachw{vista}\hyp{}Format vor.
Zum Lesen und Schreiben dieser Daten greift das Programm daher auf die
Funktionen der frei verf�gbaren \fachw{vista}\hyp{}Bibliothek
\cite{vista} zur�ck.

\section{Handhabung der Volumendaten}
\label{sec:volumetex}

Das \fachw{vista}\hyp{}Format organisiert grafische Daten als
sequentielle Datens�tze beliebiger Anzahl in Bl�cken innerhalb einer
\fachw{vista}\hyp{}Datei. Jeder Eintrag eines solchen Blockes entspricht
dabei einem Voxel des Datenraumes und damit einem Messpunkt im reellen
Raum. Die Kantenl�ngen des gemessenen Quaders sowie die Kantenl�ngen
der einzelnen Voxel sind in zus�tzlichen Informationsfeldern innerhalb
der entsprechenden Blockes gespeichert.

Gelesen werden diese Daten direkt von einer Instanz der Klasse
\code{GLvlVolumeTex}. Wie \abb{GLvlVolumeTex:Bez} zeigt, ist
\code{GLvlVolumeTex} abgeleitet von \code{SGLBaseTex}, der Basisklasse
f�r Texturcontainer aus der Basisbibliothek.
\inclfigure{hbt}{classGLvlVolumeTex__coll__graph}{10cm}{\code{GLvlVolumeTex} und ihre Beziehungen}{GLvlVolumeTex:Bez}

Von \code{SGLBaseTex} erbt \code{GLvlVolumeTex} die n�tigen Methoden
zur Verwaltung von \fachw{OpenGL}\hyp{}Texturen. Das Laden der
Volumendaten erfolgt durch die \fachw{vista}\hyp{}Laderoutinen.\\
\code{GLvlVolumeTex} l�dt beim Erzeugen einer Instanz der ihr
�bergeben Datei den Datensatz, der die gemessenen anatomischen Daten
enth�lt. Zus�tzlich zu diesen hoch aufgel�sten Daten der
Gewebestrukturen eines kompletten Gehirns, liegen den
\fachw{vista}\hyp{}Dateien meist noch funktionelle Daten bei. Diese
Informationen �ber die Aktivit�ten bestimmter Bereiche des selben
Gehirns �ber eine bestimmte Zeitspanne hinweg liegen in einer
geringeren Aufl�sung vor und umfassen meist auch nur einen Ausschnitt
des Gehirns. Sie werden sp�ter unter Verwendung von
\fachw{Multitexturing} beim Zeichnen der Schnittfl�che ``�ber'' die
Anatomischen Daten geblendet.

Die Volumendaten k�nnen in verschieden Datentypen vorliegen. \code{GLvlVolumeTex} 
muss dem Rechnung tragen und die jeweils optimale Laderoutine anwenden.  
W�hrend anatomische Daten aus ganzzahligen
Werten bestehen, sind funktionelle Daten in der Regel
Flie�kommazahlen. So werden anatomische Daten, die als Grauwerte
zwischen $0$ und $255$ vorliegen, als Indexwerte einer vorgegeben
Farbpalette geladen. Die Flie�kommazahlen, welche die Aktivit�ten im
funktionellen Datensatz repr�sentieren, lassen sich schlechter einem
bestimmten Index zuweisen. Sie werden deshalb als Parameter einer
vorgegebenen Funktion verwendet, die an den entsprechenden Stellen in
der \fachw{OpenGL}\hyp{}Textur Echtfarben zuweist.


Da die sp�tere Schnittfl�che beliebig im Datenraum platziert werden
kann, wird sie dort unter Umst�nden auch herausragen. In einem solchen
Fall muss der Renderer beim Zeichnen der Textur auf die Schnittfl�che
ein definiertes Verhalten zeigen. Der Renderer wird deshalb so
konfiguriert, dass er beim Zeichnen eines Pixels zu dem die
Texturinformationen fehlen, die Informationen des am n�chsten
liegenden verf�gbaren Voxels der Textur anwendet. Der Rand der Textur
wird dadurch bei einer herausragenden Schnittfl�che unendlich oft
wiederholt. Aus diesem Grunde ist die Textur an allen Seiten um
mindestens einen Voxel gr��er der entsprechende Datensatz. Dieser
Voxel hat im Alpha\hyp{}Kanal den Wert $0$. Der \fachw{Renderer} ist
dabei so konfiguriert, da� er derartige Punkte nicht zeichnet.

Einfache \fachw{OpenGL}\hyp{}Renderer k�nnen nur Texturen verarbeiten,
deren L�nge $2^n$ mit $n=\{1,2,3,\pp\}$ entspricht. Deshalb kann die
Gr��e der Textur nur eingeschr�nkt an die Gr��e der vorliegenden Daten
angepasst werden. Die Daten werden stattdessen in eine Textur mit der
n�chstgr��eren Potenz von zwei gelegt. Wie schon erw�hnt wird jeweils
an einer Seite ein Voxel ausgespart, w�hrend auf der Gegenseite alle
�brigen Voxel leer bleiben. Liegen Daten mit der Kantenl�nge 500 vor,
wird eine Textur der Kantenl�nge $2^9=512$ angelegt. Daraus folgt,
dass die eigentlichen Daten darin erst bei $1$ beginnen und bei $501$
enden. Die �brigen Voxel ($0$ und $501-511$) werden mit leeren
Eintr�gen ($Alpha \leftarrow 0$) aufgef�llt. Diese Spezifikationen
werden auf alle drei Dimensionen angewandt.

Die Eintr�ge $1$ bis $255$ der Farbpalette f�r die anatomischen Daten
werden mit grauen RGBA\hyp{}Werten
(${(1,1,1,255),\pp,(255,255,255,255)}$ gef�llt. Die eigentliche Textur
braucht daher wie bereits erw�hnt aus nur $8$ Bit gro�en Indexen
bestehen, deren Werte $1-255$ in dem entsprechenden Grauwert
dargestellt werden. Der ersten Eintrag der Farbpalette ($(0,0,0,0)$)
ist ein Sonderfall, sein Alpha\hyp{}Wert (vierter Wert) ist Null.
Voxel mit dem Index $0$ werden daher beim Zeichnen den Alpha\hyp{}Wert
$0$ ergeben. Sie werden somit nicht gezeichnet. Auf diese Weise k�nnen
Voxel, die nicht gezeichnet werden sollen, markiert werden, ohne ein
zus�tzliches Bit daf�r zu ben�tigen. Im Datensatz kommt der Wert $0$
nur im ``Hintergrund'' vor, und dieser soll ohnehin nicht gezeichnet
werden. Die anatomischen Daten stellen den mit Abstand gr��ten
Datensatz dar, diese Konzepte sparen Speicher bei der Ablage der Daten
im Texturspeicher des Renderers.

\section{Umsetzung der Nutzerschnittstelle}
\label{sec:cutview}

Die Nutzerschnittstelle des Programms besteht aus einer Anzahl
Fenstern mit jeweils einer \fachw{OpenGL}\hyp{}Instanz. Nach dem Start ist
nur das Fenster mit der �bersicht vorhanden. Es zeigt zu diesem
Zeitpunkt einen leeren Raum, denn es sind noch keine Schnitte
angelegt. F�r jeden neuen Schnitt wird ein weiteres Fenster mit einer
Schnittansicht ge�ffnet. Diese Schnitte verhalten sich wie im Entwurf
konzipiert. Wechselt der Nutzer zu einem Schnittfenster, kann er mit
Hilfe der Maus die Lage des Schnittes ver�ndern und der Cursor zeigt
die in den Raum projizierte Position seiner Maus an. Im
�bersichtsfenster wird gleichzeitig die Lage des Schnittes im Raum
verdeutlicht und es lassen sich beliebig viele weitere Schnitte
erzeugen.

Alle Schnittfenster, sowie das �bersichtsfenster teilen sich, wie im
Entwurf vereinbart, sowohl \fachw{Renderer}interne Daten als auch die
dazugeh�rigen programmseitigen Parameter.  Um das System nicht unn�tig
zu belasten, wird jede Anzeige nur dann aktualisiert, wenn dies
wirklich n�tig ist. Es ist deshalb notwendig, dass ein Schnitt, der
seine Lage �ndert, dies selbstst�ndig dem �bersichtsfenster mittels
\fachw{Signal}\hyp{}\fachw{Slot}\hyp{}System mitteilt, damit dieses
wei�, dass es neu zeichnen muss.

\section{Darstellung der Volumendaten auf der Schnittfl�che}
\label{sec:cutplane}

Die Klasse f�r die eigentliche Schnittfl�che \code{GLvlCutPlane} ist
(�ber Zwischenschritte) von \code{SGLPolygon} aus der Basisklasse
abgeleitet. Damit ist \code{GLvlCutPlane} ein schlichtes Viereck mit
einer Instanz der Klasse \code{GLvlVolumeTex} als Textur. Diese Textur
teilt sich die Schnittfl�che mit Hilfe intelligenter Zeiger mit allen
anderen Schnittfl�chen, da sie alle die selben Volumendaten darstellen
sollen. Ihre Eckpunkte teilt sich die Schnittfl�che mit der Kamera
ihrer Schnittansicht. Jede Schnittansicht hat eine eigene Instanz der
Klasse \code{GLvlPlaneCam}, einer Spezialisierung der Basisklasse
\code{SGLBaseCam}. Diese Kamera bestimmt bei jeder Verlagerung alle
vier Eckpunkte ihres Blickfeldes entsprechend den Beschreibungen aus
Abschnitt 4.2.3 neu. Diese Eckpunkte teilt sie sich mit der
Schnittfl�che. Die Schnittfl�che und die Projektionsfl�che der Sicht
sind damit identisch. �ndert sich ein Parameter der Kamera �ndern sich
auch die Eckpunkte der Schnittfl�che. Lediglich die Mittelung �ber
diese �nderung muss explitzit per \code{compileNextTime} an die
Schnittfl�che gesandt werden, damit sie beim n�chsten Neuzeichnen neu
kompiliert wird.

Die Lage der Schnittfl�che im eigentlichen Raum ist damit bestimmt,
ihre Lage im Datenraum fehlt aber noch. Auch hierf�r m�ssen die
Koordinaten der Eckpunkte bestimmt werden. Um deren Bestimmung
m�glichst einfach zu gestalten, benutzen der eigentliche Raum und der
Datenraum ein identisches Koordinatensystem.  Die Texturkoordinaten
k�nnen daher fast direkt aus den Raumkoordinaten der Schnittfl�che
�bernommen werden. Dabei ist jedoch zu beachten, dass die leeren Voxel
an den R�ndern der Textur im Renderer nicht gezeichnet werden. Da sie
dennoch Teil des Texturraumes sind, ist der Datenraum eigentlich als
eine Untermenge des Texturraumes zu betrachten. Die Texturkoordinaten
der Schnittfl�che werden beim Zeichnen auf den Texturraum und nicht
auf den Datenraum angewendet.  Sie m�ssen daher entsprechend angepasst
werden, um den Texturraum im Verh�ltnis zu der Schnittfl�che so zu
verschieben und zu skalieren, dass die Differenz zwischen Datenraum
und Texturraum ausgeglichen wird.

\section{Der Cursor}
\label{sec:cursor}

Der f�r den Cursor verwendete Objektklasse \code{GLvlPlaneCursor} ist
von der Basisklasse \code{SGLMetaObj} abgeleitet. Sie vereint eine
beliebige Anzahl von Instanzen der Basisklasse \code{SGLCube} in sich.
Der Cursor besteht daher aus einer beliebigen Anzahl von W�rfeln, die
entsprechend W�rfelf�rmig angeordnet werden. Die einzelnen W�rfel
haben eine Kantenl�nge von $1$ und werden als Drahtgittermodelle
gezeichnet. Ihre Anzahl wird von Nutzer bestimmt, der auf diese Weise
die Gr��e des Cursors anpassen kann. Dabei verdeutlichen die Einzelnen
W�rfel jederzeit den Ma�stab. Die Position des Cursors im GL-Raum wird
wie in Abschnitt \vref{cursor:pos} beschrieben, aus der Position des
Mauszeigers ermittelt. Seine Koordinaten k�nnen dabei auf ganzzahlige
Werte gerundet werden. Dieser ``Fang'' kann vom Nutzer aktiviert bzw.
deaktiviert werden. Obwohl der Cursor den Mauszeiger prinzipiell
ersetzt, wird dieser nicht ausgeblendet, denn vor allem bei
aktiviertem Fang verwirrt ein fehlender Mauszeiger mehr als er n�tzt.


\section{Integration der \fachw{watershed}\hyp{}Transformation}

Die in Kapitel \vref{cha:watershed_impl} beschriebenen
Implementationen der \fachw{watershed}\hyp{}Transformation liefern
Bilder $f_{wshed}:\menge{N}^3 \mapsto \menge{N}$ die die Gruppierung
der Punkte nach ihrere zugeh�rigkeit zu lokalen Minima beschreiben.
F�r jeden Punkt $p$ liefert $f_{wshed}(p)$ den Index des Minimas, dem
er angeh�rt.  Die Visualisierung dieser Unterteilung wird in
Kobination mit dem Cursor mittels Polygonen realisisiert. Jeder Punkt
$p$, der einen Nachbar $q \in N_G(p)$ mit anderer Zugeh�rigkeit hat
($f_{wshed}(p) \neq f_{wshed}(q)$) legt dazu ein Polygon an der Grenze
zwischen $p$ und $q$ an.  Aus Effizienzgr�nden werden diese Polygone
nicht als Instanzen der Basisklasse \code{SGLPolygon} angelegt.
Stattdessen wird die Erkennung Grenzen und das anlegen der
entsprechenden Polygone innerhalb der \code{generate()}-Routine der
Objektklasse \code{GlvlMinima} ausgef�hrt, wobei das Objekt zwar
kompiliert jedoch nicht gezeichnet wird. \code{GlvlMinima} wird zu
diesem Zweck von der Basisklasse \code{SGLFlObj} abgeleitet. Im
Zeichenpuffer des \fachw{Renderers} liegen somit durch ihre
Grenzfl�chen, definierte \fachw{Minima-K�rper} f�r jedes durch
\fachw{watershed} erkannte Unterobjekt. Das f�r
\code{GlvlMinima}-Objekte sehr aufwendige \code{generate()} muss
lediglich einmal durchgef�hrt werden, da sich die durch sie
repr�sentierten \fachw{Minima-K�rper} nicht ver�ndern. (siehe
``Caching von Zeichenoperationen'' in Abschnitt \vref{displist}). Die
Anwendungsseitigen Kontainer dieser Objekte (Instanzen der Klasse
\code{GlvlMinima}) werden in einer Liste unter dem Index des lokalen
Minimas mit dem entsprechenden Einzugsgebiet gespeichert.  Bewegt sich
der Cursor auf einen Punkt $p$ im Texturraum, kann der
\fachw{Minima-K�rper} in dem sich $p$ befindet direkt durch den Wert
von $f_{wshed}(q)$ angesprochen werden.  Ausschlie�lich dieser
ausgew�hlte \fachw{Minima-K�rper} wird gezeichnet.  Die Auswahl und
Darstellung eines \fachw{Minima-K�rpers} f�r weitere Bearbeitung ist
somit �u�erst effizient, und kann daher problemlos interaktiv
angewendet werden.

Die Aufgabe der Visualisierung nach der
\fachw{watershed}-Transformation besteht somit darin, dem Nutzer die
entsprechenden Unterobjekte anzuzeigen, und ihn beim ``Einsammeln''
dieser zu unterst�zen.



\chapter{Zusammenfassung}

\section{Fazit der Arbeit}

Im Rahmen dieser Arbeit wurde ein offenes und schlankes System zur
Visualisierung dreidimensionaler Daten entworfen und entwickelt.  Im
Entwurf wurden einfache und effiziente L�sungen gesucht und gefunden.
Dabei lag die Priorit�t prim�r auf der praktischen Anwendbarkeit im
Wissenschaftlichen Umfeld, und sekund�r auf der Flexibilit�t der
Anwendung. Das System wurde von Grund auf Modular entworfen. Ein
Basismodul, die \fachw{simpleGL}-Bibliothek, spielt in diesem Entwurf
eine zentrale Rolle. Die \fachw{qt\_glue}-Bibliothek entkoppelt als
kleinstes Modul das Hauptmodul von der, f�r die Nutzerschnittstelle
verwendeten \fachw{Qt}-Bibliothek. Die Nutzerschnittstelle greift als
viertes Modul auf das Basismodul und auf das vierte Modul, die
\fachw{watershed}\hyp{}Transformation zur�ck. Sie bildet auf diese
Weise die eigentliche Applikation. Weitere Module zur Erf�llung
weiterer Aufgaben k�nnen diesem Verbund jederzeit hinzugef�gt werden.

Der modulare Entwurf wurde konsequent objektorientiert umgesetzt.
W�hrend der Implementierung wurde gro�er Wert auf
Wartbarkeit,Erweiterbarkeit und Wiederverwendbarkeit des Codes gelegt.
Fehleranf�llige Mittel der Programmierung, wie z.B. Zeiger wurden
bewusst vermieden und durch moderne robustere Alternativen ersetzt.
Lediglich w�hrend der Implementierung der
\fachw{watershed}\hyp{}Transformation wurden ``traditionelle'' Mittel
bevorzugt, um die dort n�tige Speichereffizienz zu erreichen.

\section{Empfehlungen zur verwendeten Plattform}

Das Visualisierungssystem wurde haupts�chlich unter Debian-Linux
entwickelt. Seine Portabilit�t wurde durch die Verwendung portabler
Bibliotheken und den weitgehenden Verzicht auf hardwarenahe
Programmierung erhalten. Es sollte daher ggf. nach geringem
Portierungsaufwand allen �blichen UNIX\hyp{}Desktop\hyp{}Systemen
lauff�hig sein. Auch eine Portierung nach Windows w�re theoretisch
m�glich, scheitert aber unter Umst�nden an Lizenzkosten f�r \fachw{QT}
und der \fachw{vista}\hyp{}Bibliothek, die nicht f�r Windows verf�gbar
ist.

Eine Ausname der allgemeinen Verf�gbarkeit auf �blichen Plattformen
bilden in diesem Zusammenhang Rechner ohne kompatiblen
\fachw{OpenGL}\hyp{}Renderer. Das Visualisierungssystem kann zwar auf
Softwarerenderern wie z.B. Mesa \cite{mesa} betrieben werden, hier
treten jedoch oft Probleme mit dem Grafikspeicher auf. Die Verlagerung
der Volumendaten in den Speicher des Renderers entlasten auf Rechnern
mit separatem Grafikspeicher die CPU und den Systemspeicher. Jedoch
sind Systeme ohne echten Grafikspeicher (shared memory) aus diesem
Grund nur eingeschr�nkt nutzbar, da hier der Grafikspeicher
physikalisch im Systemspeicher liegt. Dieser ist f�r die Lagerung
dreidimensionaler Texturen nicht geeignet. Ein Einsatz des entwickelten
Visualisierungssystems auf Rechnern, bei denen die Grafikkarte
lediglich als Chip auf dem \fachw{Mainboard} realisiert ist, ist daher
zwar m�glich, aber nicht empfehlenswert. Einfache 3D-Grafikkarten mit
ausreichendem eigenen Speicher und \fachw{OpenGL}\hyp{}kompatiblen
Grafikprozessoren (GPUs) wie z.B. ``ATI Radeon 9200'', oder ``nVidia
GeForce2'' sind sind dagegen vollkommen ausreichend. Besonders
empfehlenswert sind GPUs nach Version $2.0$ des
\fachw{OpenGL}-Standards, denn dieser l�sst einen effektiveren Umgang
mit dem Texturspeicher zu. Der verf�gbare Grafikspeicher sollte jedoch
in jedem Fall gro� genug sein, um die Volumendaten aufnehmen zu
k�nnen. F�r die am CBS �blichen MRT-Aufnahmen reicht ein
Grafikspeicher von $32$ MByte.

Verf�gt das System wie empfohlen �ber separaten Grafikspeicher,
reichen f�r die Visualisierung inklusive Segmentierung $128$ MByte
Systemspeicher. An die Festplatte werden keine besonderen
Anforderungen gestellt. Ein gro�z�gig bemessener Bildschirm ist jedoch
von Vorteil. 

Da \fachw{OpenGL} ein Client\hyp{}Server\hyp{}System ist, steht auch
einer Anwendung auf einem Terminal\hyp{}Server nichts im Wege, solange
die Grafikkarten der Terminals die Anforderungen erf�llen. Dabei
sollte jedoch beachtet werden, dass die Volumendaten beim Laden von
der Anwendung (in diesem Fall auf dem Terminal\hyp{}Server) zum
Renderer (die Grafikkarte im Terminal) �bertragen werden m�ssen. Daher
wird eine ausreichend leistungsf�hige Netzwerkverbindung ben�tigt.

\section{Fortf�hrende �berlegungen}

Das im Rahmen dieser Arbeit entwickelte eigentliche Visualisierungssystem
(virgil) konnte dank der Auslagerung der meisten Funktionen in die
entsprechenden Module einfach gehalten werden. Daher bleibt der
Quellcode sehr �bersichtlich. Zuk�nftige Korrekturen, Anpassungen oder
Erweiterungen lassen sich somit leicht realisieren. Die parallel
entwickelte Basisbibliothek weist au�er \fachw{OpenGL} keine weiteren
Abh�ngigkeiten auf. Ihr Funktionsumfang deckt ein weites Spektrum von
Visualisierungsaufgaben ab. Ihre Wiederverwendbarkeit auch au�erhalb
des aktuellen Rahmens ist daher ausgesprochen hoch.

Eine weitere Entwicklung sowohl des Basismoduls, als auch des
Visualisierungssystems wird daher empfohlen.



\listoffigures

\rhead{ }
\renewcommand{\headrulewidth}{0pt}   % put no line under the header 
\printnomenclature[2.5cm]

\bibliography{ref,urls}


\end{document}

///
Echtzeit ? n�
vorhandene Wurzeln k�nnen beim n�chsten mal wieder verwendet werden => abk�rzung

s/topografisch/topographisch/i - ok
Kernspinresonanztomografen etwas lang ? ne eig nicht..
Volumendaten kennt nich jeder - sollte jetzt ok sein (in einf�hrung zusammen mit ``dreidimensional'' erw�hnt)
MRI erkl�ren / ausschreiben - is doch
Ordinaten kennt nich jeder
API kennt nich jeder
keine ``leeren'' �berschriften - lieber eine kleine Zusammenfassung
Mesa oder mesa
mesa -> eventuell technische Details etwas k�rzen
renderer ist ein Fachwort, IMMER

Aufwandsberechnung pr�fen
-6.2.2 build_graph und kurz vor Wurzelsuche

ordinate oder koordinate ?
was ist ein Operator
bei wshed am ANFANG sagen woher ich das hab
simpleGL.tex ``Die Polygonobjekte k�nnen ihre allgemeinen Zeichenfunktionen inklusive dem Verwalten des Operationspuffers wie andere Objekte auch ausf�hren.'' Zeile 507
Zoomen - is das OK ??
Zeilenangaben in () oder []
watershed-Verfahren oder Algorithmus
vincent wohl doch nich komplexer (O(X)) - aussage raus
$1$ ohne Einheit - ok ??
nach leehren \ref{} suchen
vista f�r Win ?
cns cbs und MPI suchen - durch was Einheitliches (CBS = Max-Planck-Institut
f�r Kognitions- und Neurowissenschaften) ersetzen
new erkl? OO erkl ?
Firmen/Produknamen lieber ins Glossar
GPL ?
die sache mit den Voxeln
``Bild'' erkl

%This technique is the basis of current MRI techniques. A few years later, in 1977, Raymond Damadian demonstrated MRI called field-focusing nuclear magnetic resonance. In this same year, Peter Mansfield developed the echo-planar imaging (EPI) technique.  This technique will be developed in later years to produce images at video rates (30 ms / image).


\[
\begin{pmatrix}
 x_{scale} & 0 & 0 & 0 \\ 
 0 & y_{scale} & 0 & 0 \\ 
 0 & 0 & z_{scale} & 0 \\ 
 0 & 0 & 0 & 1
\end{pmatrix} 
\begin{pmatrix}
 1 & 0 & 0 & x_{move} \\ 
 0 & 1 & 0 & y_{move} \\ 
 0 & 0 & 1 & z_{move} \\ 
 0 & 0 & 0 & 1
\end{pmatrix} 
=
\begin{pmatrix}
 x_{scale} & 0 & 0 & x_{scale}*x_{move} \\ 
 0 & y_{scale} & 0 & y_{scale}*y_{move} \\ 
 0 & 0 & z_{scale} & z_{scale}*z_{move} \\ 
 0 & 0 & 0 & 1
\end{pmatrix} 
\]

\[
\begin{pmatrix}
 1 & 0 & 0 & -x_{move} \\ 
 0 & 1 & 0 & -y_{move} \\ 
 0 & 0 & 1 & -z_{move} \\ 
 0 & 0 & 0 & 1
\end{pmatrix} 
\begin{pmatrix}
 \frac{1}{x_{scale}} & 0 & 0 & 0 \\ 
 0 & \frac{1}{y _{scale}} & 0 & 0 \\ 
 0 & 0 & \frac{1}{z _{scale}} & 0 \\ 
 0 & 0 & 0 & 1
\end{pmatrix} 
=
\begin{pmatrix}
 \frac{1}{x_{scale}} & 0 & 0 & -x_{move} \\ 
 0 & \frac{1}{y_{scale}} & 0 & -y_{move} \\ 
 0 & 0 & \frac{1}{z_{scale}} & -z_{move} \\ 
 0 & 0 & 0 & 1
\end{pmatrix} 
\]
