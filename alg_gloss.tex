\nomenclature{GL-Raum}{virtueller, durch eine \fachw{OpenGL}-Sicht
  dargestellter Raum, in den zu zeichnende Objekte platziert werden.
  Der GL-Raum ist theoretisch unendlich, es wird jedoch nur ein, durch
  die Sicht definierter Teil, gezeichnet. }

\nomenclature{Texturraum}{virtueller Raum innerhalb des GL-Raumes, der
  durch eine dreidimensionale Textur geformt wird. Hier auch
  vereinfachend als Datenraum bezeichnet.}

\nomenclature{GUI}{``graphical user interface'' bzw. ``Grafische Benutzungsschnittstelle''\\
  System das grafisch mit dem Nutzer �ber Fenster, Schaltfl�chen,
  Dialogboxen u.�. interagiert. Wird allgemein als intuitiver als die
  Interaktion mittels Kommandos angesehen. }


\nomenclature{Kontext}{Hier meist \fachw{OpenGL}-Kontext.\\
  Identifiziert eine Instanz des \fachw{OpenGL}-Renderers. Der Kontext
  ist fest mit reservierten Ressourcen des Systems, sowie dem Fenster
  in das der Renderer zeichnet, verbunden.}



\nomenclature{Widget}{Kurzform von ``Window Gadget''\\
  Frei �bersetzt: ``Dingsbums f�r Fenster''.  Bezeichnet in der
  Programmierung allgemein Funktionen oder Objekte, die zum Darstellen
  in Grafischen Nutzerschnittstellen verwendet werden.}

%\nomenclature{WM}{Window Manager bzw. Fenstermanager\\
%System, das die Darstellung der Fenster einer Grafischen
%Nutzerschnittstelle beeinflusst und die Eingabeger�te (Tastatur, Maus
%usw.) dem jeweils aktiven Fenster zuordnet. Sowohl der �u�ere Rahmen
%als auch die Position der prim�rer Fenster werden vom WM kontrolliert.}

\nomenclature{voxel}{Kurzform f�r Volumenpixel\\
  Dreidimensionale Version des Pixels. Ein Voxel repr�sentiert ein
  Datum in einem Volumendatensatz. Voxel sind hier nicht
  gezwungenerma�en kubisch. }

\nomenclature{CBS}{Max-Planck-Institut f�r Kognitions- und
Neurowissenschaften\\
(engl.: Max Planck Institute for Human Cognitive and Brain Sciences ) }


\nomenclature{vista}{Die Vista Bibliothek\\
  Eine Funktionsbibliothek f�r die wissenschaftliche Arbeit mit
  graphischen Daten im weitesten Sinne. Im CBS vor allem f�r den
  Datenaustausch verwendet. (siehe auch Abschnitt \vref{sec:volumetex}
  und Quellenverweis \cite{vista})}


\nomenclature{mesa}{auch Mesa3D\\
  Vor allem im UNIX-Bereich verbreiteter \fachw{Softwarerenderer} mit
  modularer Architektur. (siehe auch Abschnitt \vref{def:renderer} und
  Quellenverweis \cite{mesa}) }

\nomenclature{OpenGL}{Open Graphics Library\\
  Urspr�nglich von \cite{sgi} als IrisGL entwickelte
  Softwareschnittstelle f�r Vektorgrafik. Inzwischen als OpenGL
  standardisiert.\\
  (siehe auch Kapitel \vref{opengl})}

\nomenclature{API}{Application Programming Interface\\
  Die Schnittstelle, die ein Softwaresystem anderen Programmen zur
  Verf�gung stellt.}

\nomenclature{ARB}{Architecture Review Board\\
  Hier das Standardisierungsgremium des \fachw{OpenGL}\hyp{}Standards. \\
  (siehe auch Kapitel \vref{opengl})}

\nomenclature{DVR}{direct volume rendering\\
  Renderingmethode bei der Volumendaten direkt mittels
  \fachw{Raytracing} abgebildet werden.  (siehe auch Abschnitt
  \vref{desing:volume}) } 

\nomenclature{SF}{``surface fitting'' bzw.
  Oberfl�chenrendering\\
  Renderingmethode bei der Volumendaten indirekt mittels
  Polygonisierung dargestellt werden.  (siehe auch Abschnitt
  \vref{desing:volume})}

\nomenclature{Raytracing}{auch Strahlverfolgung\\
  Ein Rendering-Verfahren zur Erzeugung meist fotorealistisch
  wirkender Bilder. F�r jedes Pixel der Zielbildes wird mindestens ein
  Strahl in die Szene gesandt. Die ``Bedingungen'', die der Strahl
  dort ``vorfindet'' bestimmen die Farbe des Zielpixels.}

\nomenclature{MRI}{Magnetresonanztomografie, auch
  Kernspintomografie oder Kernspin\\
  Ein bildgebendes Verfahren zur Darstellung von Strukturen im Inneren
  des K�rpers, das auf dem magnetischen Moment von Atomkernen beruht.\\
  (siehe auch Abschnitt \vref{MRI}) }

\nomenclature{fMRI}{funktionelle Magnetresonanztomografie
  oder Kernspintomografie\\
  kernspintomografisches Verfahren das unter Verwendung des
  BOLD\hyp{}Verfahrens �ber die Sauerstoffanreicherung R�ckschl�sse auf die
  Aktivit�t des entsprechenden Gebietes zieht.\\ (siehe auch Abschnitt
  \vref{MRI}) }

\nomenclature{BOLD}{Blood Oxygen Level Dependency\\
  Von \citet{Ogawa:bold} entwickeltes kernspintomografisches Verfahren,
  das die 1935 von Linus Pauling und Charles D. Coryell entdeckte
  paramagnetische Eigenschaft von H�moglobin ausnutzt, um die
  Sauerstoffanreicherung im Blut zum messen.\\ (siehe auch Abschnitt
  \vref{MRI}) } 

\nomenclature{Renderer}{vom englischen ``to render''
  d.h. ``machen''\\
  Hier ist damit die Generierung von Pixelbildern aus abstrakten
  Eingabedaten gemeint. Als \fachw{Renderer} werden Systeme
  bezeichnet, diese Funktion ausf�hren. }

\nomenclature{rastern}{Abbilden beliebiger Bilddaten in
  diskreten Puffern\\
  Bezeichnet eine Operation, die meist stetige zweidimensionale Bilder
  in zweidimensionale Pixelpuffer (z.B. Bildpuffer einer Grafikkarte)
  abbildet, in diesem Fall ist $f:\menge{R}^2 \mapsto \menge{N}^2$.\\
  (siehe auch Abschnitt \vref{opengl:arch}) }

\nomenclature{MRT}{Magnetresonanztomografie\\
  (siehe auch MRI)}

\nomenclature{overhead}{Daten, die nicht prim�r zu den Nutzdaten
  z�hlen, sondern als Hilfsdaten zur �bermittlung oder Speicherung
  ben�tigt werden.}

\nomenclature{dual-core}{Prozessoren mit zwei Rechenwerken\\
Dual-Core-Prozessoren werden von vielen Herstellern als Weg gesehen,
die Leistung von Desktopsystemen ohne Erh�hung der Taktrate zu
steigern. Sowohl AMD, als auch Intel haben Dual-Core-Prozessoren angek�ndigt oder  bereits
 im Angebot.}

\nomenclature{GPU}{Graphic Processing Unit oder Visual Processing Unit (VPU)\\
Prozessor, der auf graphische Operationen wie
z.B. \fachw{OpenGL}\hyp{}Operationen spezialisiert ist. 
(siehe auch Abschnitt \vref{opengl})
}

\nomenclature{Direct3D}{3D-Modul von DirectX\\
\fachw{OpenGL} nachempfundene, aber objektorientierte Schnittstelle
f�r die Programmierung dreidimensionaler Systeme. Stark an MS-Windows
gebunden, und daher nur dort verf�gbar.
}

\nomenclature{Qt}{``The Qt application development framework'' \\
Eine umfangreiche und sehr m�chtige Systembibliothek mit
Widgetsystem, die f�r zahlreiche Plattformen verf�gbar ist. 
Die API ist plattformunabh�ngig.
}